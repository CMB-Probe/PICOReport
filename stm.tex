\begin{table}[]
\caption{\textbf{Mission Specifications}}\label{tab:specs}
%\small
\footnotesize
% sensitivity table
\hrule
\hspace{0.25in}
\begin{tabular}{ccccccccccccccccccccccc}
%\hline
\noalign{\vskip 1mm}
Bands & GHz    & 21   & 25   & 30   & 36   & 43   & 52   & 62   & 75   & 90  & 108 & 129 & 155 & 186 & 223 & 268 & 321 & 385 & 462 & 555  & 666   & 799  \\
%\cline{1-2}
FWHM & arcmin   & 38.4 & 32.0 & 28.3 & 23.6 & 22.2 & 18.4 & 12.8 & 10.7 & 9.5 & 7.9 & 7.4 & 6.2 & 4.3 & 3.6 & 3.2 & 2.6 & 2.5 & 2.1 & 1.5  & 1.3   & 1.1  \\
%\cline{1-2}
\noalign{\vskip8pt}
\multicolumn{2}{c}{Polarization map depth} & \multicolumn{21}{c}{} \\ % just filling line. 
Baseline  & $\mu$K$_{\rm CMB}$\,arcmin & 23.9 & 18.4 & 12.4 & 7.9  & 7.9  & 5.7  & 5.4  & 4.2  & 2.8 & 2.3 & 2.1 & 1.8 & 4.0 & 4.5 & 3.1 & 4.2 & 4.5 & 9.1 & 45.8 & 177.2 & 1047 \\
Baseline  & Jy/sr                 & 8.3  & 10.9 & 11.8 & 12.9  & 19.5  & 23.8  & 45.4  & 58.3  & 59.3 & 77.3 & 96.0 & 119.1 & 433.1 & 604.2 & 433.4 & 577.8 & 429.1 & 551.1 & 1580 & 2075 & 2884 \\
CBE$^a$  & $\mu$K$_{\rm CMB}$\,arcmin & 16.9 & 13.0 & 8.7 & 5.6  & 5.6  & 4.0  & 3.8  & 3.0  & 2.0 & 1.6 & 1.5 & 1.3 & 2.8 & 3.2 & 2.2 & 3.0 & 3.2 & 6.4 & 32.4 & 125.3 & 740.3 \\
CBE$^a$  & Jy/s                       & 5.9 & 7.7 & 8.3 & 9.2  & 13.8  & 16.8  & 32.1  & 41.3  & 41.8 & 53.5 & 69.3 & 83.7 & 301.5 & 436.3 & 303.5 & 411.1 & 303.1 & 387.3 & 1117 & 1467 & 2040 \\
\noalign{\vskip 1mm}
%\hline
\end{tabular}
%
%\hfill
\enskip
%\vrule % vertical rule
\quad
%
%\begin{tabular}{|r@{\hskip 0.06in}l|r@{\hskip 0.06in}l|}
\begin{tabular}{l@{\hskip 0.25in}lc}
\multicolumn{1}{|@{\hskip 0.22in}l}{Cost:{\:\;}\$\,958M  } & \multicolumn{1}{l}{Launch mass:{\:\;}2147~kg } \\
\multicolumn{1}{|@{\hskip 0.22in}l}{Mission length:{\:\;}5 years }  & \multicolumn{1}{l}{Total power:{\:\;}1526~W }\\ 
\noalign{\vskip8pt}
\multicolumn{2}{l}{\hspace{0.4in} Combined polarization map depth} & \\
\multicolumn{2}{l}{\multirow{2}{3in}{0.87~$\mu$K$_{\rm CMB}$ equivalent to 3250 {\it Planck} Missions} } & \\
 & \\
\multicolumn{2}{l}{\multirow{2}{3in}{0.61~$\mu$K$_{\rm CMB}$ equivalent to 6400 {\it Planck} Missions} } & \\ 
 & 
%
%& \multicolumn{3}{@{\hskip 0.0in}l}{0.87~$\mu$K$_{CMB}$\, ' } 
\end{tabular}
\hrule
\vspace{2pt}
$^a$ Current best estimate  \\ % footnote to table
%
%
\vspace{8mm}
%
%
%   -----------------------------------------------------------------------------------------------------------------
%
%
\caption{\textbf{Science Traceability Matrix (STM) }}\label{tab:STM}
%\small
\footnotesize
\begin{tabular}{cccccccc}
%\noalign{\vskip 2mm}
\hline
\noalign{\vskip 2mm}    
% Header line 1
%\rowcolor[HTML]{EFEFEF} 
\multicolumn{1}{c}{\multirow{2}{1in}{\centering \bf Science Goals from NASA Science Plan}}&
\multicolumn{1}{c}{\multirow{2}{2in}{\centering \bf Science Objectives}}& 
\multicolumn{3}{c}{\bf Scientific Measurement Requirements}&
\multicolumn{2}{c}{\bf Instrument (single instrument, single mode)}&
\multicolumn{1}{c}{\multirow{2}{1.75in}{\centering \bf Mission Functional Requirements}} 
\\
%Header line 2
\noalign{\vskip 2mm}    
\cline{3-7}
\noalign{\vskip 2mm}    
%\rowcolor[HTML]{EFEFEF} 
\multicolumn{1}{c}{} &
\multicolumn{1}{c}{} &
\multicolumn{1}{c}{Model Parameters} &
\multicolumn{1}{c}{Physical Parameters} & 
\multicolumn{1}{c}{Observables} &
\multicolumn{1}{c}{Functional Requirements} &
\multicolumn{1}{c}{Projected Performance} & 
\\
% Line SO1
\noalign{\vskip 2mm}    
\hline
\multicolumn{1}{l}{\multirow{2}{1in}{\vskip5pt \textbf{\textit{Explore how the Universe began (Inflation)}}}}&
\multicolumn{1}{l}{\parbox[t]{2in}{SO1. Probe the physics of the big bang by detecting the energy scale at which inflation occurred if it is above $4\times10^{15}$\,GeV, or place an upper limit if it is below (\S\,\ref{sec:fundamentalsci})}}&
\multicolumn{1}{l}{\parbox[t]{2in}{Tensor-to-scalar ratio $r$: $\sigma(r) = 1\times10^{-4}$ at $r = 0$; $r < 5 \times 10^{-4}$ at $5\sigma$ confidence level$^a$}} &
\multicolumn{1}{l}{\parbox[t]{2in}{CMB polarization $BB$ power spectrum for modes $2<\ell<300$ to cosmic-variance limit, and CMB lensing power spectrum for modes $2<\ell<1000$ to cosmic-variance limit}}&
\multicolumn{1}{l}{\parbox[t]{2in}{Linear polarization across $60 < \nu < 300$\,GHz over entire sky; foreground separation requires $21 < \nu < 799$\,GHz}}& 
\multicolumn{1}{l}{\multirow{5}{1.75in}{%
\vskip15pt
Frequency coverage: central frequencies $\nu_c$ from 21 to 799\,GHz.
\vskip5pt
Frequency resolution: $\Delta\nu/\nu_c = 25\%$.
\vskip5pt
Sensitivity: See Table~\ref{tab:specs}.
Combined instrument noise:  $< 0.61\,\mu{\rm K}_{\rm CMB}\sqrt{s}$.
\vskip5pt
Angular resolution [for delensing and foreground separation]: ${\rm FWHM} =  6.2' \times ( 155\,{\rm GHz} / \nu_c )$.
\vskip5pt
Effective aperture: 1.4~m.
\vskip5pt
Sampling rate: $( 3 / {\rm Beam FWHM} ) \times ( 336' / {\rm s})$.
}}& 
\multicolumn{1}{l}{\parbox[t]{1.5in}{}}& 
\multicolumn{1}{l}{\multirow{7}{1.75in}{%
\vskip10pt
Sun-Earth L2 orbit with Sun-Probe-Earth $< 15^\circ$.
\vskip5pt
5 yr survey with $\ge 95\%$ survey efficiency.
\vskip5pt
Full sky survey: Spin instrument at 1 rpm; boresight $69^\circ$ off spin axis;
spin axis $26^\circ$ off anti-Sun line, precessing $360^\circ$ / 10hr.
\vskip5pt
Pointing control: Spin axis $60'$ ($3\sigma$, radial). Spin \@ $1 \pm 0.1$ rpm ($3\sigma$)
\vskip5pt
Pointing stability: Drift of spin axis $< 1'$/1min ($3\sigma$, radial);
jitter $< 20''$/20 ms ($3\sigma$, radial).
\vskip5pt
Pointing knowledge
(telescope boresight):
$10'' \, (3\sigma$, each axis) from spacecraft attitude;
$1'' \, (1\sigma$, total) final reconstructed;
\vskip5pt
Return and process instrument data:
1.5 Tbits/day (after 4$\times$ compression)
\vskip5pt
Thermally isolate instrument from solar radiation and from spacecraft bus
}}\\
% Line SO2
\noalign{\vskip 1mm}
\cline{2-5}
\noalign{\vskip 1mm}
\multicolumn{1}{l}{}&
\multicolumn{1}{l}{\parbox[t]{2in}{SO2. Probe the physics of the big bang by excluding classes of potentials as the driving force of inflation (\S\,\ref{sec:fundamentalsci}, Fig.~\ref{fig:nsr})}}&
\multicolumn{1}{l}{\parbox[t]{2in}{Spectral index ($n_s$) and its derivative ($n_{\rm run}$): $\sigma(n_s) < 0.0015$; $\sigma(n_{\rm run}) < 0.002$}}&
\multicolumn{1}{l}{\parbox[t]{2in}{CMB polarization $BB$ power spectrum for modes $2<\ell<1000$ to cosmic-variance limit}}&
\multicolumn{1}{l}{\multirow{3}{2in}{
\vskip15pt
Intensity and linear polarization across $60 < \nu < 220$\,GHz over the entire sky}}& 
\multicolumn{1}{l}{\parbox[t]{1.75in}{}}& 
\multicolumn{1}{l}{\multirow{7}{1.5in}{%
Frequency coverage: See Table~\ref{tab:specs}. %\ref{tab:freq}.
\vskip 2pt 
21 bands with $\nu_c$ from 21 to 799\,GHz.
\vskip5pt
Frequency resolution: $\Delta\nu/\nu_c = 25\%$.
\vskip5pt
Sensitivity: See Table~\ref{tab:specs}. %\ref{tab:sensitivity}.
\vskip2pt
Combined instrument noise: $0.43\,\mu{\rm K}_{\rm CMB}\sqrt{\rm s}$.
\vskip5pt
Angular resolution: See Table~\ref{tab:specs}. %~\ref{tab:resolution}.
\vskip2pt
${\rm FWHM} = 6.2' \times (155\,{\rm GHz} / \nu_c )$;
$1.1'$ for $\nu_c = 799\,$GHz.
\vskip5pt
Sampling rate: See Table~\ref{tab:focal_plane}. %\ref{Sampling}.
$( 3 / {\rm Beam FWHM}) \times ( 336' / {\rm s})$ 
}}&
\multicolumn{1}{l}{\parbox[t]{1in}{}}\\
% Line SO3
\noalign{\vskip 1mm}
\cline{1-4}
\noalign{\vskip 1mm}
\multicolumn{1}{l}{\multirow{2}{1in}{\vskip5pt \textbf{\textit{Discover how the Universe works (neutrino mass and $N_{\rm eff)}$}}}}&
\multicolumn{1}{l}{\parbox[t]{2in}{SO3. Determine the sum of neutrino masses. (\S\,\ref{sec:fundamentalsci}, Fig.~\ref{fig:DM_baryons})}}&
\multicolumn{1}{l}{\parbox[t]{2in}{Sum of neutrino masses ($\Sigma m_\nu$): $\Sigma m_\nu < 15$\,meV with DESI or Euclid$^b$ }}& %; \comred{$\Sigma m_\nu < ??$\,meV alone}}}&
\multicolumn{1}{l}{\parbox[t]{2in}{CMB polarization $BB$ power spectrum for modes $2<\ell<4000$ to cosmic-variance limit; CMB intensity maps (to give Compton $Y$ map from which we extract clusters)}}&
\multicolumn{1}{l}{\parbox[t]{2in}{}}& 
\multicolumn{1}{l}{\parbox[t]{1.75in}{}}& 
\multicolumn{1}{l}{\parbox[t]{1.5in}{}}& 
\multicolumn{1}{l}{\parbox[t]{1in}{}}
\\
% Line SO4
\noalign{\vskip 1mm}
\cline{2-4}
\noalign{\vskip 1mm}
&
\multicolumn{1}{l}{\parbox[t]{2in}{SO4. Tightly constrain the thermalized fundamental particle content of the early Universe (\S\,\ref{sec:fundamentalsci}, Fig.~\ref{fig:Neff_future})}}&
\multicolumn{1}{l}{\parbox[t]{2in}{Number of neutrino effective relativistic degrees of freedom ($N_{\rm eff}$): $\sigma(N_{\rm eff}) < 0.03$}}&
\multicolumn{1}{l}{\parbox[t]{2in}{CMB temperature and $EE$ polarization power spectra $2<\ell<4000$ to cosmic-variance limit}}&
\multicolumn{1}{l}{\parbox[t]{2in}{}}& 
\multicolumn{1}{l}{\parbox[t]{1.75in}{}}& 
\multicolumn{1}{l}{\parbox[t]{1.5in}{}}& 
\multicolumn{1}{l}{\parbox[t]{1in}{}}
\\
% Line SO5
\noalign{\vskip 1mm}
\cline{1-5}
\noalign{\vskip 1mm}
\multicolumn{1}{l}{\multirow{1}{1in}{\textbf{\textit{Explore how the Universe evolved (reionization)}}}}&
\multicolumn{1}{l}{\parbox[t]{2in}{SO5. Distinguish between models that describe the formation of the earliest stars in the Universe (\S\,\ref{sec:extragalacticsci}, Fig.~\ref{fig:ReionizationPICO})}}&
\multicolumn{1}{l}{\parbox[t]{2in}{Optical depth to reionization ($\tau$): $\sigma(\tau) < 0.002$}}&
\multicolumn{1}{l}{\parbox[t]{2in}{CMB polarization $EE$ power spectrum for modes $2<\ell<20$ to cosmic-variance limit}}&
\multicolumn{1}{l}{\parbox[t]{2in}{Linear polarization across $60 < \nu < 300$\,GHz over entire sky; foreground separation enveloped by SO1}}& 
\multicolumn{1}{l}{\parbox[t]{1.75in}{}}& 
\multicolumn{1}{l}{\parbox[t]{1.5in}{}}& 
\multicolumn{1}{l}{\parbox[t]{1in}{}}
\\
% Line SO6
\noalign{\vskip 1mm}
\cline{1-6}
\noalign{\vskip 1mm}
\multicolumn{1}{l}{\multirow{2}{1in}{{\vskip5pt \textbf{\textit{Explore how the Universe evolved (Galactic structure and dynamics)}}}}}&
\multicolumn{1}{l}{\parbox[t]{2in}{SO6. Constrain the temperatures and emissivities characterizing the Milky Way's interstellar diffuse dust (\S\,\ref{sec:galacticsci})}}&
\multicolumn{1}{l}{\parbox[t]{2in}{Intrinsic polarization fractions of the components of the diffuse interstellar medium to accuracy better than 3\% when averaged over $10'$ pixels }}&
\multicolumn{1}{l}{\parbox[t]{2in}{Fractional polarization and intensity as a function of frequency}}&
\multicolumn{1}{l}{\parbox[t]{2in}{Intensity and linear polarization maps in 12 frequency bands between 108 and 799\,GHz.}}& 
\multicolumn{1}{l}{\multirow{2}{1.75in}{
\vskip 15pt
Enveloped by SO1--4, except:
\vskip4pt
Angular resolution: $\le 1.1'$ (at highest frequency)
\vskip4pt
Sensitivity at 799\,GHz: 27.4\, kJy/sr
}}& 
\multicolumn{1}{l}{\parbox[t]{1.5in}{}}& 
\multicolumn{1}{l}{\parbox[t]{1in}{}}
\\
% Line SO7
\noalign{\vskip 1mm}
\cline{2-5}
\noalign{\vskip 1mm}
\multicolumn{1}{l}{}&
\multicolumn{1}{l}{\parbox[t]{2in}{SO7. Determine if magnetic fields are the dominant cause of low Galactic star-formation efficiency (\S\,\ref{sec:galacticsci})}}&
\multicolumn{1}{l}{\parbox[t]{2in}{Ratio of cloud mass to maximum mass that can be supported by magnetic field ("Mass to flux ratio" $\mu$); %\comred{$\sigma(\mu) < ??$};
ratio of turbulent energy to magnetic energy (Alfv\'{e}n Mach number $\mathcal{M}_A$) on scales 0.05--100\,pc  }}&%\comred{$\sigma(\mathcal{M}_A) < ??$}}}&
\multicolumn{1}{l}{\parbox[t]{2in}{The turbulence power spectrum on scales 0.05--100\,pc; magnetic field strength ($B$) as a function of spatial scale and density; hydrogen column density; gas velocity dispersion
}}&
\multicolumn{1}{l}{\parbox[t]{2in}{Intensity and linear polarization with $< 1$\,pc resolution for thousands of molecular clouds and with $< 0.05$\,pc for the 10 nearest molecular clouds; maps of polarization with 1' resolution over the entire sky}}& 
\multicolumn{1}{l}{\parbox[t]{1.75in}{}}& 
\multicolumn{1}{l}{\parbox[t]{1.5in}{}}& 
\multicolumn{1}{l}{\parbox[t]{1in}{}}
\\
\noalign{\vskip 1mm}
\hline
\noalign{\vskip 1mm}
\end{tabular}
{\footnotesize
$^a$ The values predicted include delensing and foreground subtraction; see Section~\ref{sec:fundamentalsci}. \\
$^b$ Using the PICO $\tau$ and $BB$ lensing power spectrum, and BAO from DESI or Euclid. The same strength constraint can be derived independently using clusters detected by PICO, if their redshift becomes available, and using  $\sigma_{8}$ from LSST.
}
\end{table}
 
