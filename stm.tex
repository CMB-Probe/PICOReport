\begin{table}[]
\caption{\textbf{Science Traceability Matrix}}\label{tab:STM}
\footnotesize
\begin{tabular}{cccccccc}
\noalign{\vskip 2mm}
\hline
\noalign{\vskip 2mm}    
% Header line 1
%\rowcolor[HTML]{EFEFEF} 
\multicolumn{1}{c}{\multirow{2}{1in}{\bf Science Goals from NASA Science Plan}}&
\multicolumn{1}{c}{\multirow{2}{2in}{\bf Science Objectives}}& 
\multicolumn{3}{c}{\bf Scientific Measurement Requirements}&
\multicolumn{2}{l}{\bf Instrument (single instrument, single mode)}&
\multicolumn{1}{c}{\multirow{2}{1.5in}{\bf Mission Functional Requirements}} 
\\
%Header line 2
\noalign{\vskip 2mm}    
\cline{3-7}
\noalign{\vskip 2mm}    
%\rowcolor[HTML]{EFEFEF} 
\multicolumn{1}{c}{} &
\multicolumn{1}{c}{} &
\multicolumn{1}{c}{Model Parameters} &
\multicolumn{1}{c}{Physical Parameters} & 
\multicolumn{1}{c}{Observables} &
\multicolumn{1}{c}{Functional Requirements} &
\multicolumn{1}{c}{Projected Performance} & 
\\
% Line SO1
\noalign{\vskip 2mm}    
\hline
\multicolumn{1}{l}{\multirow{2}{1in}{\vskip5pt Explore how the universe began (Inflation)}}&
\multicolumn{1}{l}{\parbox[t]{2in}{SO1. Probe the physics of the big bang by detecting the energy scale at which inflation occurred if it is above $4\times10^{15}$\,GeV, or place an upper limit if it is below (\S\,2.2.1,Figure TBD)}}&
\multicolumn{1}{l}{\parbox[t]{2in}{Tensor-to-scalar ratio $r$: $\sigma(r) < 5\times10^{-5}$ at $r = 0$; $r < 10^{-4}$ at 95\% confidence level}} &
\multicolumn{1}{l}{\parbox[t]{2in}{CMB polarization $B$-mode power spectrum for modes $2<l<300$ to cosmic variance limit, and CMB lensing power spectrum for modes $2<l<1000$ to cosmic variance limit}}&
\multicolumn{1}{l}{\parbox[t]{2in}{Linear polarization across $60 < \nu < 300$\,GHz over entire sky}}& 
\multicolumn{1}{l}{\multirow{4}{2in}{%
Frequency coverage [for foreground separation]: $\nu_c$ from 30 to 500\,GHz.
\vskip5pt
Frequency resolution: $\Delta\nu/\nu_c = 25\%$.
\vskip5pt
Sensitivity: See Table~\ref{tab:tbd}.
Combined instrument weight of $< 0.7\,\mu{\rm K}_{\rm CMB}\sqrt{\rm s}$.
\vskip5pt
Angular resolution [for delensing and foreground separation]: ${\rm FWHM} =  6.2' \times ( 155\,{\rm GHz} / \nu_c )$.
\vskip5pt
Sampling rate: $( 3 / {\rm Beam FWHM} ) \times ( 336' / {\rm s})$.
\vskip5pt
 Polarization systematics?
}}& 
\multicolumn{1}{l}{\parbox[t]{2in}{}}& 
\multicolumn{1}{l}{\parbox[t]{1in}{}}\\
% Line SO2
\noalign{\vskip 1mm}
\cline{2-5}
\noalign{\vskip 1mm}
\multicolumn{1}{l}{}&
\multicolumn{1}{l}{\parbox[t]{2in}{SO2. Probe the physics of the big bang by excluding classes of potentials as the driving force of inflation (\S\,2.2.1, Figure TBD)}}&
\multicolumn{1}{l}{\parbox[t]{2in}{Spectral index ($n_s$) and its derivative ($n_{\rm run}$): $\sigma(n_s) < 0.0015$; $\sigma(n_{\rm run}) < 0.002$}}&
\multicolumn{1}{l}{\parbox[t]{2in}{CMB polarization $B$-mode power spectrum for modes $2<l<1000$ to cosmic variance limit}}&
\multicolumn{1}{l}{\parbox[t]{2in}{}}& 
\multicolumn{1}{l}{\parbox[t]{2in}{}}& 
\multicolumn{1}{l}{\multirow{8}{1.5in}{%
Frequency coverage: See Table~\ref{tab:freq}.
\vskip 2pt 
21 bands with $\nu_c$ from 21 to 799\,GHz.
\vskip5pt
Frequency resolution: $\Delta\nu/\nu_c = 25\%$.
\vskip5pt
Sensitivity: See Table~\ref{tab:sensitivity}.
\vskip2pt
Combined instrument weight of $0.46\,\mu{\rm K}_{\rm CMB}\sqrt{\rm s}$.
\vskip5pt
Angular resolution: See Table~\ref{tab:resolution}.
\vskip2pt
${\rm FWHM} = 6.2' \times (155\,{\rm GHz} / \nu_c )$;
$1.1'$ for $\nu_c = 799\,$GHz.
\vskip5pt
Sampling rate: See Table~\ref{Sampling}.
$( 3 / {\rm Beam FWHM}) \times ( 336' / {\rm s})$ 
}}& 
\multicolumn{1}{l}{\multirow{8}{1.5in}{%
Sun-Earth L2 orbit with Sun-Probe-Earth $< 15^\circ$.
\vskip5pt
5 yr survey with $\ge 95\%$ survey efficiency.
\vskip5pt
Full sky survey: Spin instrument \@ 1 rpm; Boresight $69^\circ$ off spin axis;
Spin axis $26^\circ$ off anti-Sun line, precessing $360^\circ$ / 10hr.
\vskip5pt
Pointing control: Spin axis $60'$ ($3\sigma$, radial). Spin \@ $1 \pm 0.1$ rpm ($3\sigma$)
\vskip5pt
Pointing stability: Drift of spin axis $< 1'$/1min ($3\sigma$, radial);
Jitter $< 20"$/20 ms ($3\sigma$, radial).
\vskip5pt
Pointing knowledge
(telescope boresight):
$10"$ ($3\sigma$, each axis) from spacecraft attitude
$1"$ ($3\sigma$, each axis) final reconstructed
\vskip5pt
Return and process instrument data:
1.5 Tbits/day (after 4x compression)
\vskip5pt
Thermally isolate instrument from solar radiation and from spacecraft bus
}}\\
% Line SO3
\noalign{\vskip 1mm}
\cline{1-5}
\noalign{\vskip 1mm}
\multicolumn{1}{l}{\multirow{2}{1in}{\vskip5pt Discover how the universe works (Neutrino Mass and Neff)}}&
\multicolumn{1}{l}{\parbox[t]{2in}{SO3. Determine the sum of neutrino masses, and distinguish between inverted and normal neutrino mass hierarchies (\S\,2.2.1, Figure TBD)}}&
\multicolumn{1}{l}{\parbox[t]{2in}{Sum of neutrino masses ($\Sigma m_\nu$): $\Sigma m_\nu < 15$\,meV with DESI or Euclid; $\Sigma m_\nu < X$\,meV alone}}&
\multicolumn{1}{l}{\parbox[t]{2in}{CMB polarization $B$-mode power spectrum for modes $2<l<4000$ to cosmic variance limit; CMB intensity maps (to give Compton $Y$ map from which we extract clusters)}}&
\multicolumn{1}{l}{\parbox[t]{2in}{Intensity and linear polarization across 60--400\,GHz over entire sky}}& 
\multicolumn{1}{l}{\parbox[t]{2in}{}}& 
\multicolumn{1}{l}{\parbox[t]{2in}{}}& 
\multicolumn{1}{l}{\parbox[t]{1in}{}}
\\
% Line SO4
\noalign{\vskip 1mm}
\cline{2-5}
\noalign{\vskip 1mm}
&
\multicolumn{1}{l}{\parbox[t]{2in}{SO4. Tightly constrain the thermalized fundamental particle content of the early Universe (\S\,2.2.1, Figure TBD)}}&
\multicolumn{1}{l}{\parbox[t]{2in}{Number of neutrino effective relativistic degrees of freedom ($N_{\rm eff}$): $\sigma(N_{\rm eff}) < 0.03$}}&
\multicolumn{1}{l}{\parbox[t]{2in}{CMB temperature and $E$-mode polarization power spectra $2<l<4000$ to cosmic variance limit}}&
\multicolumn{1}{l}{\parbox[t]{2in}{Intensity and linear polarization across 60--300\,GHz over entire sky }}& 
\multicolumn{1}{l}{\parbox[t]{2in}{}}& 
\multicolumn{1}{l}{\parbox[t]{2in}{}}& 
\multicolumn{1}{l}{\parbox[t]{1in}{}}
\\
% Line SO5
\noalign{\vskip 1mm}
\cline{1-6}
\noalign{\vskip 1mm}
\multicolumn{1}{l}{\parbox[t]{1in}{\vskip5pt Explore how the universe evolved (reionization)}}&
\multicolumn{1}{l}{\parbox[t]{2in}{SO5. Distinguish between models that describe the formation of the earliest stars in the universe (\S\,2.2.2, Figure TBD)}}&
\multicolumn{1}{l}{\parbox[t]{2in}{Optical depth to reionization ($\tau$): $\sigma(\tau) < 0.002$}}&
\multicolumn{1}{l}{\parbox[t]{2in}{CMB polarization $E$-mode power spectrum for modes $2<l<20$ to cosmic variance limit; $T$ power spectrum and Compton $Y$ maps.}}&
\multicolumn{1}{l}{\parbox[t]{2in}{Intensity and linear polarization across 60--300\,GHz over entire sky (role of intensity maps at high $\ell$ to be clarified)}}& 
\multicolumn{1}{l}{\parbox[t]{2in}{Enveloped by SO1--4, and less driving: Angular resolution $< 1^\circ$ at XX\,GHz (role of intensity maps at high $\ell$ to be clarified). Combined instrument weight of  $< 0.86$\, $\mu$K\,arcmin}}& 
\multicolumn{1}{l}{\parbox[t]{2in}{}}& 
\multicolumn{1}{l}{\parbox[t]{1in}{}}
\\
% Line SO6
\noalign{\vskip 1mm}
\cline{1-6}
\noalign{\vskip 1mm}
\multicolumn{1}{l}{\multirow{3}{1in}{\parbox[t]{1in}{\vskip5pt Explore how the universe evolved (Galactic structure and dynamics)}}}&
\multicolumn{1}{l}{\parbox[t]{2in}{SO6. Determine if magnetic fields are the dominant cause of low star formation efficiency in our Galaxy. (\S\,2.2.3, Figure TBD)}}&
\multicolumn{1}{l}{\parbox[t]{2in}{}}&
\multicolumn{1}{l}{\parbox[t]{2in}{The turbulence power spectrum on scales 0.05--100 pc (from cores to diffuse cloud envelopes). Magnetic field strength ($B$) as a function of spatial scale and density. Hydrogen column density. Gas velocity dispersion.
}}&
\multicolumn{1}{l}{\parbox[t]{2in}{Intensity and linear polarization with $< 1$\,pc resolution for thousands of molecular clouds and $< 0.05$\, pc for the 10 nearest molecular clouds.}}& 
\multicolumn{1}{l}{\multirow{3}{2in}{%
Enveloped by SO1--4, except:
\vskip4pt
Angular resolution: $\le 1.1'$ (at highest frequency)
\vskip4pt
Sensitivity at 800\,GHz: 27.4\, kJy/sr
\vskip4pt
Saturation/Dynamic range?
}}& 
\multicolumn{1}{l}{\parbox[t]{2in}{}}& 
\multicolumn{1}{l}{\parbox[t]{1in}{}}
\\
% Line SO7
\noalign{\vskip 1mm}
\cline{2-5}
\noalign{\vskip 1mm}
\multicolumn{1}{l}{}&
\multicolumn{1}{l}{\parbox[t]{2in}{SO7. Constrain the temperatures and emissivities characterizing Milky Way's interstellar diffuse dust.}}&
\multicolumn{1}{l}{\parbox[t]{2in}{Intrinsic polarization fractions of the warm and cold components of the diffuse interstellar medium to accuracy better than 2\% when averaged over 10 arcmin pixels. Temperatures and spectral indices of the two dust components to an accuracy better than??\% }}&
\multicolumn{1}{l}{\parbox[t]{2in}{Fractional polarization and intensity as a function of frequency}}&
\multicolumn{1}{l}{\parbox[t]{2in}{Intensity and linear polarization maps in 12 frequency bands between 108 and 800\,GHz.}}& 
\multicolumn{1}{l}{\parbox[t]{2in}{}}& 
\multicolumn{1}{l}{\parbox[t]{2in}{}}& 
\multicolumn{1}{l}{\parbox[t]{1in}{}}
\\
% Line SO8
\noalign{\vskip 1mm}
\cline{2-5}
\noalign{\vskip 1mm}
\multicolumn{1}{l}{}&
\multicolumn{1}{l}{\parbox[t]{2in}{SO8. Determine the role of energy feedback in the evolution of Milky Way's interstellar medium.}}&
\multicolumn{1}{l}{\parbox[t]{2in}{Ratio of turbulent energy to magnetic energy (Alfv\'en Mach number Ma) on scales $0.03$--$100$\,pc   sigma(Ma)<??}}&
\multicolumn{1}{l}{\parbox[t]{2in}{The turbulence power spectrum on scales 0.03--100 pc in the neutral ISM. 
Magnetic field strength ($B$) as a function of spatial scale and density.
Neutral hydrogen velocity dispersion.
}}&
\multicolumn{1}{l}{\parbox[t]{2in}{Maps of polarization with $1'$ resolution over the entire sky.}}& 
& 
& 
\\
\noalign{\vskip 1mm}
\hline
\end{tabular}
\end{table}
 
