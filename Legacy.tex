\documentclass[PICOReport.tex]{subfiles}

\begin{document}

\definecolor{mygray}{gray}{0.6}

PICO was designed to respond to requirements posed by the seven \ac{SOs} listed in Table~\ref{tab:STM}. It will also generate a rich catalog of hundreds of thousands of new sources, consisting of proto-clusters, strongly lensed galaxies, and polarized radio and dusty galaxies. An abundance of information about galaxy and cluster evolution, dark matter, the physics of jets of active galactic nuclei, and magnetic fields of dusty galaxies will be stored in this catalog (Table~\ref{tab:STM2}). The catalog will be mined in future years through subsequent analysis and follow-up observations. 
 \begin{table}[h]
\caption{\textbf{Legacy Surveys } }\label{tab:STM2}
\footnotesize
\vspace{-0.1in}
\begin{tabular*}{\textwidth}{@{}l@{\extracolsep{\fill}}ll@{}}
\noalign{\vskip 2mm}
\hline
\noalign{\vskip 1mm}
\hline
\noalign{\vskip 2mm}    
% Header line 1
{\bf \hfil Catalog\hfil}&
{\bf \hfil Impact\hfil}&
{\bf \hfil Science\hfil}\\
% Line 1
\noalign{\vskip 2mm}    
\hline
\noalign{\vskip 1mm}    

\parbox[t]{0.8in}{Strongly\\ lensed galaxies}&
\parbox[t]{2.55in}{Discover 4500$^a$ strongly lensed and highly magnified dusty galaxies across redshift. 
\vspace{1mm}
{\color{mygray}\hrule}
\vspace{1mm}
Current knowledge: 13 sources confirmed in \planck\ data; a few hundred candidates in \textit{Herschel}, SPT and ACT data.}&
\parbox[t]{2.7in}{Gain information about the physics governing early, $z\simeq5$, galaxy evolution, taking advantage of magnification and extra resolution enabled by gravitational lensing;  learn about dark matter sub-structure in the lensing galaxies.}\\
% Line 3
\noalign{\vskip 1mm}    
\cline{1-3}
\noalign{\vskip 1mm}    

\parbox[t]{0.8in}{Proto-clusters}&
\parbox[t]{2.55in}{Discover 50,000$^{a}$ mm/sub-mm proto-clusters distributed over the sky out to $z\sim4.5$.  
\vspace{1mm}
{\color{mygray}\hrule}
\vspace{1mm}
Current knowledge: \planck\ + ACT/SPT data expected to yield a few tens.}&
\parbox[t]{2.7in}{Probe the earliest phases of cluster evolution, well beyond the reach of other instruments; test the formation history of the most massive virialized halos; investigate galaxy evolution in dense environments.}\\
% Line 2
\noalign{\vskip 1mm}    
\hline
\noalign{\vskip 1mm}    

\parbox[t]{0.8in}{Nearby galaxies}&
\parbox[t]{2.55in}{Detect 30,000 galaxies at $z\simlt 0.1$ at frequencies above 300~GHz.  
\vspace{1mm}
{\color{mygray}\hrule}
\vspace{1mm}
Current knowledge: 3400 (280) source candidates in the \planck\ 857 (353)~GHz  band. }&
\parbox[t]{2.7in}{Using frequencies that match cold ($15-25$~K) dust emission, give its spectral energy distribution as a function of galaxy properties to enable correlations with star-formation activity.} \\
\noalign{\vskip 1mm}
\hline 
\noalign{\vskip 1mm}

\parbox[t]{0.8in}{Polarized point\\ sources}&
\parbox[t]{2.55in}{Detect 2000$^{b}$ radio and several thousand dusty galaxies in polarization. 
\vspace{1mm}
{\color{mygray}\hrule}
\vspace{1mm}
Current knowledge:  about 200 radio sources up to 100~GHz; one polarization measurement of a dusty galaxy. }&
\parbox[t]{2.7in}{Study the physics of jets of extragalactic sources, close to their active nuclei; determine the large-scale structure of magnetic fields in dusty galaxies; determine the importance of polarized sources as a foreground for CMB polarization science.}\\
\noalign{\vskip 1mm}
\hline
\noalign{\vskip 1mm}

\parbox[t]{0.8in}{Cosmic infrared \\ background}&
\parbox[t]{2.55in}{Provide eight maps of the anisotropy from dusty star-forming galaxies for frequencies $\nu>200$~GHz, and with 1\arcmin\ resolution at 800~GHz.
\vspace{1mm}
{\color{mygray}\hrule}
\vspace{1mm}
Current knowledge:  Three \planck\ (higher noise) maps between 300 and 900~GHz with 5\arcmin\,~resolution. }&
\parbox[t]{2.7in}{Improve constraints on the parameters describing universal star-formation history. Construct a tracer of large-scale structure for CMB de-lensing. Cross-correlate with galaxy surveys and CMB lensing map.}\\
\noalign{\vskip 1mm}
\hline
\noalign{\vskip 1mm}

\end{tabular*}
{\footnotesize
$^a$ Confusion (not noise) limited\qquad
$^b$ Noise and confusion limited }
\end{table}


%Here we focus on two specific science deliverables that are enriched by PICO's unique capabilities. %catalog. \comor{Explain Why?}

%\comor{Kathy Romer says: Table 2 - I like the way this is broken down with ?Current knowledge? summaries. But it only talks about Planck. What about ACT, SPT, SO, balloons etc.? Gianfranco: SPT and ACT are already mentioned in connection to strongly lensed galaxies. So far there is not much from them, in the published literature, on source polarization. On proto-clusters there is the recent discovery of one at z=4.3 (Miller et al. 2018). Perhaps we could add that. }

\subsubsection{Early Phases of Galaxy Evolution}

%\comor{Kathy Romer: Section 2.3.1 - page 22 - I noted ?why do we care about lensed high-z galaxies? in the margin. So maybe you need to stress the motivation for this section more? Gianfranco: It is already said that strong lensing provides a unique possibility to look into the structure and kinematics of high-z dusty star-forming galaxies, i.e. to get crucial information on how they form and evolve. I don?t know what to say more.}

PICO's catalog of high-$z$ strongly-lensed galaxies will provide answers to major open issues in galaxy formation and evolution. What are the main physical mechanisms shaping the properties of galaxies~\citep{SilkMamon2012, SomervilleDave2015}: in situ processes, interactions, mergers, or cold flows  from the intergalactic medium? And how do feedback processes work? To settle these issues we need direct information on the structure and dynamics of high-$z$ galaxies. But these are compact, with typical sizes of 1--2~kpc~\cite{Fujimoto2018}), corresponding to angular sizes of 0.1--$0.2''$ at $z\simeq 2$--3. Thus they are hardly resolved, even by ALMA or by HST. If they {\it are} resolved, high enough \ac{SNR}s per resolution element are only achieved for the brightest galaxies, which are probably not representative of the general population.



Strong gravitational lensing provides a solution to these problems. Since lensing conserves the surface brightness, the effective angular size is stretched on average by a factor of $\mu^{1/2}$, where $\mu$ is the gravitational magnification, thus substantially increasing the resolving power. A spectacular example is ALMA observations of the \planck-discovered, strongly lensed galaxy PLCK\_G244.8\-+54.9 at $z \simeq 3.0$  with $\mu \simeq 30$~\citep{Canameras2017ALMA}. ALMA observations with a $0.1''$ resolution reached an astounding spatial resolution of 60~pc, substantially smaller than the size of Milky Way giant molecular clouds. CO spectroscopy of this object, measuring the kinematics of the molecular gas, gave an uncertainty of 40--50~km\,s$^{-1}$. Such precision allows a high \ac{SNR} detection of the predicted $\sim$1000~km\,s$^{-1}$ outflows capable of sweeping the galaxy clear of gas that would otherwise be available for star formation~\citep{KingPounds2015}. In this specific case, there were no clear indications that mergers or cold flows shaped the galaxy, but similar spectroscopy of another strongly lensed galaxy at $z=5.3$ detected a fast (800 km\,s$^{-1}$) molecular outflow due to feedback~\citep{spilker2018}. 


%The outflow carries mass at a rate close to the star-formation rate, and can thus remove a large fraction of the gas that would otherwise be available for star formation.  
% Ca\~{n}ameras et al.~\citep{Canameras2017ALMA} have obtained CO spectroscopy for this object, measuring the kinematics of the molecular gas with an uncertainty of 40--50~km/s. This spectral resolution makes possible a direct investigation of massive outflows driven by AGN feedback at high $z$. Using this technique \citet{Spilker2018} detected a fast (800 km/s) molecular outflow due to feedback in a strongly lensed galaxy at $z=5.3$. They found that the outflow carries mass at a rate close to the star formation rate, and can thus remove a large fraction of the gas that would otherwise be available for star formation.

PICO will detect thousands of early forming galaxies whose flux densities are boosted by large factors due to strong lensing~(Fig.~\ref{fig:SED3}, right). Currently there are reports of just a few other high-$z$ galaxies that are spatially resolved thanks to gravitational lensing, albeit with less extreme magnifications~\citep{Dye2018, Lamarche2018, Sharda2018}. PICO's catalog will be transformative as it will probe the \ac{SED} of the lensed galaxies at their peaks. Two examples of known sources are shown in the left panel of Figure~\ref{fig:SED3}. While nearly all ground-based instruments observe at frequencies up to $\nu = 10^{11.45}$~Hz, PICO's data will extend to the peak of the \ac{SED}, up to $\nu = 10^{11.9}$~Hz.

An extrapolation of the \textit{Herschel} counts to the 70\% non-Galactic sky gives a detection of 4,500 strongly-lensed galaxies with a redshift distribution peaking at $2\simlt z \simlt 3$~\cite{Negrello2017lensed}, but extending up to $z> 5$ (Fig.~\ref{fig:SED3}, left panel).
%\footnote{Extrapolating from achieved performance by the South Pole Telescope~\cite{??}, we estimate that S3 experiments will detect $\sim$1,600 such sources, and only in the RJ part of the spectrum.} 
If objects like the $z=5.2$ strongly lensed galaxy HLS\,J091828.6+514223 exist at higher redshifts, they will be detectable by PICO out to $z>10$. At the 600~GHz detection limit, about 25\% of all detected extragalactic sources will be strongly lensed; for comparison, at optical/near-IR and radio wavelengths, where intensive searches have been carried out for many years, the yield is only about 0.1\%, more than two orders of magnitude lower~\cite{Treu2010}. To add to the extraordinary sub-mm lensing bonanza, the selection of PICO-detected strongly lensed galaxies will be easy because of their unique sub-mm colors (Fig.~\ref{fig:SED3}, left), resulting in a selection efficiency close to 100\% \citep{Negrello2010}. The survey will find the brightest objects over the entire sky, maximizing the efficiency of selecting sources for follow-up observations. 
%\comor{need to compare to SO, extrapolated from SPT}

The intensive high spectral and spatial resolution follow-up campaign of this large sample will enable a leap forward in our understanding of the processes driving early galaxy evolution and open up other exciting prospects, both on the astrophysical and cosmological sides \cite[e.g.,][]{Treu2010}.
\begin{figure*}[t]
%\vskip-3cm
\begin{center}
\includegraphics[width=0.416\columnwidth, trim={0 0 0 0cm}, clip]{images/fig_SED_PICO.pdf}
\hspace{0.75cm}
\includegraphics[width=0.4\columnwidth, trim={0 0 0 0cm}, clip]{images/NgtF_pico_NEWNEW.pdf}
\vskip-0.3cm
\caption{ \captiontext {\bf Left:} PICO will detect thousands of new strongly lensed galaxies near the peak of their spectral energy distributions (SEDs), such as SMM\,J2133$-$0102 (blue)  at $z=2.3$~\cite{Swinbank2010} and HLS\,J$091828.6{+}514223$ (orange) at $z=5.2$~\cite{Combes2012}. The dashed lines are the SEDs before magnification by lensing. PICO's higher resolution gives point-source detection limits (black line) that are up to 10 times fainter than \planck 's 90\% completeness limits (red line~\cite{PCCS2}). High-frequency measurements ($\nu>300$~GHz) of 30,000 low-$z$ galaxies, like M61 (magenta, SED was scaled down by a factor of ten), will give a census of their cold dust.  {\bf Right:} Integral counts of unlensed (black) and strongly lensed, high-$z$ (orange) star-forming galaxies for 70\% of the sky away from the Galactic plane at 600~GHz based on fits of \textit{Herschel} counts over 1000 deg$^2$ (inset~\citep{Negrello2017lensed}). The PICO detection region (right of vertical red line) will yield a factor of 1000 increase in strongly lensed galaxies relative to \planck~(yellow square), as well as about 50,000 proto-clusters (blue) and 2,000 radio sources (green)~\citep{Negrello2017protocl}.}
%Also shown are predicted radio source counts (green). }
\label{fig:SED3}
\end{center}
\vspace{-0.25in}
\end{figure*}

\subsubsection{Early Phases of Cluster Evolution}

PICO will open a new window for the investigation of early phases of cluster evolution, when their member galaxies were actively star forming (and dusty), but the hot \ac{IGM} was not necessarily in place. In this phase, traditional approaches to cluster detection (X-ray and SZ surveys, and searches for galaxy red sequences) work only for the more evolved clusters, which do include a hot \ac{IGM}; indeed these methods have yielded only a handful of confirmed proto-clusters at $z\simgt 1.5$ \cite{Overzier2016}.\footnote{More high-$z$ proto-clusters have been found by targeting the environment of tracers of very massive halos, such as radio-galaxies, QSOs, and sub-mm galaxies. These searches are, however, obviously biased.} \planck~has demonstrated the power of low-resolution surveys for the study of large-scale structure~\cite{Planck2016high_z}, but its resolution was too poor to detect individual proto-clusters \cite{Negrello2017protocl}.  Studies of the high-$z$ two-point correlation function \cite{Chen2016, Negrello2017protocl} and \textit{Herschel} images of the few sub-mm bright protoclusters detected so far, at $z \le 4$ \cite{Ivison2013, Wang2016, Oteo2018}, all of which will be detected by PICO, indicate sizes of $\simeq 1'$ for the proto-cluster cores, nicely matching the PICO FWHM at the highest frequencies.

PICO will detect 50,000 proto-clusters as peaks in the high-frequency maps, which are not available for ground-based instruments (Table~\ref{tab:STM2}; blue line in the right-hand panel of Fig.~\ref{fig:SED3}).
%%% FULL SKY MAPS ARE NOT AVAILABLE FOR GROUND-BASED INSTRUMENTS %%%
%\footnote{Extrapolating from achieved performance by the South Pole Telescope~\cite{??}, we estimate that S3 experiments will detect few hundred sources.} 
The redshift distribution will extend out to $z\sim4.5$. This catalog will be augmented by 150,000 evolved clusters, detected by the SZ effect. This will constitute a breakthrough in the observational validation of the formation history of the most massive dark-matter halos, traced by clusters, representing a crucial test of models for structure formation. Follow-up observations will characterize the properties of member galaxies, probing galaxy evolution in dense environments and shedding light on the complex physical processes driving it.

\subsubsection{Additional Products of PICO Surveys}

PICO will yield a complete census of cold ($15$--$25$~K) dust, available to sustain star formation in the nearby Universe, by detecting tens of thousands of galaxies mostly at $z\simlt 0.1$; the \ac{SED} of M61 is a typical example (Fig.~\ref{fig:SED3}, left). With a statistical population, and information only available using data at frequencies above 300~GHz, we will investigate the spectral energy distribution of the dust as a function of galaxy properties, such as morphology and stellar mass. 

PICO will increase by an order of magnitude the number of blazars selected at sub-mm wavelengths and will determine the SEDs of many hundreds of them up to 800\,GHz and up to $z> 5$. Blazar searches are the most effective way to sample the most massive black holes at high $z$ because of the Doppler boosting of their flux densities. PICO's surveys of the largely unexplored mm/sub-mm spectral region will also offer the possibility to discover new transient sources or events, such as blazar outbursts~\cite{Metzger2015}.

PICO will make a leap forward in the determination of the polarization properties of both radio sources and dusty galaxies over a frequency range where ground-based surveys are impractical or impossible.
It will find  1,200 radio sources and 350 dusty galaxies above a flux density limit of 4~mJy at 320~GHz, and 500 radio sources and 15,000 dusty galaxies above 6~mJy at 800~GHz.
% At 320 (800)~GHz it will find 1,200~(500) radio sources and 350~(15,000) dusty galaxies above a flux limit of 4~(6)~mJy.  
These data will give information on the structure and ordering of large-scale magnetic fields in  dusty galaxies. In the case of radio sources, emission at higher frequencies comes from regions closer to the central engine, providing information on the innermost regions of the jets, close to the active nucleus. 

The anisotropy of the \ac{CIB}, produced by dusty star-forming galaxies over a wide redshift range $0 < z \lesssim 5$, is an excellent probe of the history of star formation across time. The \planck\ collaboration derived values for parameters describing the rate of star formation out to $z\sim4$~\cite{2014A&A...571A..30P,2014A&A...571A..18P,madau2014}. PICO's lower noise and twice the number of frequency bands will give an order of magnitude improvement on the statistical errors for these parameters~\cite{Wu:2016hej}. Similar improvement will be achieved in constraining $M_{\mathrm{eff}}$, the galaxy halo mass that is most efficient in producing star-formation activity. PICO's increased sensitivity to Galactic dust polarization will enhance the separation of signals coming from the largely unpolarized \ac{CIB} and polarized Galactic dust; an effective separation of signals currently limits making reliable, legacy-quality \ac{CIB} maps. 
By providing a nearly full-sky map of matter fluctuations traced by dusty star-forming galaxies, such a set of maps could be used for delensing the CMB~\cite{Sherwin/Schmittfull}, for measuring local primordial non-Gaussianity from \ac{CIB} auto-correlations~\cite{tucci}, or for cross-correlations with CMB lensing maps and with galaxy surveys~\cite{Schmittfull/Seljak}.

\end{document}

%%%%%%%%%%%%%%%%%%%%%%%%%%%%%%%
