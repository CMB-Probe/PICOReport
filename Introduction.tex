
\documentclass[PICOReport.tex]{subfiles}

\begin{document}

%\vskip 10pt
 
 
The Probe of Inflation and Cosmic Origins (PICO) has seven \ac{SOs}. They derive from three strategic goals that are well-articulated by NASA's science plan~\citep{latest_nasa_science_plan,latest_nasa_strategic_plan}: to explore how the Universe began; to discover how the Universe works; and to explore how the Universe evolved. The \ac{SOs}, which include probing the inflationary epoch after the big bang, constraining the properties of fundamental particles and potentially detecting new ones, probing the structure and evolution of the Universe, and understanding the structure of our own Galaxy, require measurements in and around frequency bands in which the cosmic microwave background is most intense. The \ac{SOs} and the measurement requirements derived from them are given in Table~\ref{tab:STM} and define the PICO `baseline design.' 
This report focuses on the primary \ac{SOs} listed in the Table but also describes the much broader set of science deliverables that the mission design enables.
%This design gives rise to a mission that delivers a much broader set of science deliverables.  This report describes the broad array of science deliverables, but focuses primarily on the primary \ac{SOs} listed in the Table. 

The PICO mission consists of a single instrument: an imaging polarimeter that surveys the entire sky at 21 frequency bands spread between 21 and 799~GHz.  The telescope has an aperture of 1.4~m, giving diffraction-limited resolution between 38\arcmin\ and 1\arcmin . The instrument incorporates a 0.1~K cooled focal plane that hosts 12,996 \ac{TES} bolometric detectors. The baseline design contains a margin of 40\% in detector noise (\S\,\ref{sec:focal_plane}). We include throughout this report performance estimates that are based also on our current estimate for the actual performance. Those are labeled `\ac{CBE}' (\S\,\ref{sec:focal_plane}). Table~\ref{tab:specs} gives key mission parameters and Table~\ref{tab:spec_bands} gives the frequency bands, resolution, and both baseline and \ac{CBE} noise levels. Experience with past space missions, most recently with \planck , shows that pre-mission calculated detector performance is in fact achieved in space~\citep{planck1101.2038,planck1101.2039,Jarosik}.

Mission operations throughout the 5-year duration of the survey are simple and are optimized for polarimetric measurements as each sky pixel is scanned along multiple orientations. The spacecraft spins around its symmetry axis at 1~rpm and the symmetry axis precesses around the anti-Sun direction with a period of 10~hours. With this repetitive scan pattern, the entire sky is scanned every 6~months, giving ten independent full-sky maps of the intensity and polarization Stokes parameters $T$, $Q$, and $U$.  
%Each sky pixel is scanned along multiple orientations and therefore independent full-sky maps of all Stokes parameters can be reconstructed from the data of {\it each} of the detectors. \comor{give more explicit contrast to Planck and detector differencing in systematics?}
%This is in contrast to e.g. \planck ,  which relied on differencing 

% \comor{words about 1/f?}. 

Some of the PICO polarization science goals are more appropriately described in terms of $E$ and $B$ polarization maps rather than $Q$ and $U$~\cite{seljak97,kamionkowski97a, zaldarriaga97b,kamionkowski97b}. This is because sources of polarization signatures that are scalar in nature, such as primordial density perturbations, can only produce $E$-mode polarization. Sources that are tensor in nature, such as gravitational waves, can produce both $E$- and $B$-mode polarization. The angular power spectra of $E$ and $B$ maps will be denoted as `$EE$' and `$BB$'.

This report assumes that PICO's Phase~A will start in 2023. The science outcomes are expected to break new ground, and to be complementary to data sets available at the end of 2020s and the beginning of the following decade. Where appropriate we highlight complementarities with funded projects that are in implementation phases, such as LSST, Euclid, and WFIRST. We also include performance comparisons to funded CMB projects that are in implementation and for which final design specifications and projections exist in the literature. Such next-generation US-based sub-orbital CMB experiments are collectively denoted as `Stage-3 (S3)'~\citep{advancedact,spt3g,so,class_overview,biceparray,spider,piper}. 

%Therefore we are including performance comparisons to funded projects that are in implementation and for which final design specifications and projections exist in the literature. Such next-generation US-based CMB experiments are collectively denoted as Stage-3 (S3)~\citep{advancedact,spt3g,so,class_overview,biceparray,spider,piper}. 

This section describes PICO's science objectives, places them in the context of current knowledge, and provides performance forecasts (\S~\ref{sec:fundamentalsci}--\ref{sec:legacy}). It gives our estimates of: the efficacy of separating the detected radiation into the several astrophysical sources of emission~(\S~\ref{sec:signal_separation}); an assessment of anticipated systematic uncertainties~(\S~\ref{sec:systematics}); a discussion of PICO's complementarity with sub-orbital measurements~(\S~\ref{sec:complementarity}); and the measurement requirements that derive from the combination of these topics~(\S~\ref{sec:requirements}).  
%\S~\ref{sec:instrument} describes the instrument, which consists of the telescope, the focal plane, the detector readout, and shielding and cooling hardware. The section also describes plans for integration and testing. \S~\ref{sec:design_reference} describes the operations, including the instrument's survey of the sky, and the spacecraft. In \S~\ref{sec:technology_maturation} we discuss the path to maturing the few technologies that are not yet at \ac{TRL}~6, and potential descopes. Project management, assessment of risk, and costs are presented in \S~\ref{sec:project_management}. 



%between with a polarization sensitivity that is 57 or 82 times that of the \planck\ mission for the PICO baseline and current best estimate (\ac{CBE}) configurations, respectively. 

%PICO's data will enrich and complement other astrophysical surveys in the next decade. 
%The \ac{CBE} is our current estimate for the actual performance of the mission. \comor{where are we outlining the assumptions for the CBE}. 

%Emission within our own Galaxy is a source of confusion that must be separated with high fidelity before definitive discovery of non-zero $r$, or stronger upper limits, can be claimed~\citep{2016A&A...586A.133P}. For the levels of $r$ targeted in the next decade, PICO has both the frequency coverage and sensitivity to measure and separate sources of foreground confusion and is thus poised to detect or place unprecedented constraints on the physics of inflation. Its measurements of the spectral index of primordial fluctuations will give the strongest constraints yet on specific models of inflation. 

%PICO has the frequency coverage and sensitivity to measure and separate sources of foreground confusion and is thus poised to detect or place unprecedented constraints on the physics of inflation. \comor{Its measurements of the spectral index of primordial fluctuations will give the strongest constraints yet on specific models of inflation. }


%These same experimental features are advantageous not only for $\Neff$ but for any new physics with signatures on the CMB. Of particular interest is the nature of dark matter and its interactions. PICO will place constraints that are more than one order of magnitude stronger than \planck\ for a dark matter particle of MeV mass range, which can not be probed by direct detection experiments. PICO will thus reveal important clues to the nature of the fundamental physical laws and our cosmic origins. 


%Secondary anisotropies in the CMB\footnote{Secondary anisotropies arise from sources other than primordial density and \ac{IGW} fluctuations} provide a wealth of information on the growth and evolution of structure in our Universe. CMB lensing, the thermal and kinematic \ac{SZ} effects, and extragalactic point sources all contribute significantly to the CMB intensity fluctuations on small angular scales (note that lensing is also present in polarization fluctuations). Immense progress in mapping these sources is enabled by PICO's depth, broad frequency coverage, and relatively high resolution. The all-sky, projected mass map reconstructed from CMB lensing that PICO will provide can be correlated with tracers of large-scale structure to tomographically probe the growth of structure at unprecedented \ac{SNR} levels. The thermal SZ effect provides a map of the integrated free-electron pressure along the line of sight, and the peaks in this map trace the locations of all galaxy clusters in the Universe. PICO will find all the massive, virialized, galaxy clusters at any redshift.  The epoch of reionization imprints information on the statistical moments of the kinematic SZ signal. The combination of these kSZ statistical moments with the cosmic-variance-limited $\tau$ measurement from PICO will provide tight constriants on the global properties of the sources responsible for reionization of the Universe.
%Combining thermal and kinematic SZ measurements of galaxies will yield important information about the thermodynamics of galaxy formation, as well as the precise location of the ``missing baryons''. 

%Our understanding of magnetic fields is rooted in observations of the very local Universe: the Milky Way and nearby galaxies. Magnetic fields are observed to be a foremost agent of the Milky Way's ecology. Understanding the Milky Way's magnetic field is crucial for making progress on important issues in the astrophysics of galaxies: the dynamics and energetics of the multiphase interstellar medium; the efficiency of star formation; the acceleration and propagation of cosmic rays; and the impact of feedback on galaxy evolution. Through its detailed high-resolution polarization measurements of Galactic dust emission, PICO will produce an unprecedented data set, mapping Galactic magnetic fields and providing answers to these questions (SO7). 

%Learning about magnetic fields has impact beyond understanding the dynamics and evolution of galaxies. The very origin of magnetic fields in galaxies, and their possible evolution from primordial, early Universe cosmic magnetic fields is a topic of intense debate. PICO is poised to provide definitive answer as to whether early Universe magnetic fields provide the seeds for the magnetic fields observed in most current galaxies. 

%The magnetized ISM in the solar neighborhood presents a challenge for the investigation of cosmological signals. The signals of interest, such as CMB B-mode polarization, CMB spectral distortions, and 21cm line emission from the cosmic dawn and the reionization epoch, are obscured by Galactic dust and synchrotron emission that can be orders of magnitude brighter. PICOs detailed mapping of these signals will strongly constrain the physical properties of the ISM and thus models of dust grain composition, temperature, and emissivities (SO6). In addition, this significantly improved understanding of the physics of the ISM will feed back into improving foreground removal that is essential for supporting other PICO science goals.

%The PICO deep and high resolution maps will yield a treasure trove of point sources that will be mined for years. The mission will provide a full-sky catalog of tens of thousands of extragalactic millimeter and sub-millimeter point sources, which are beacons for active galactic nuclei (in the radio) and dust emission from vigorously star-forming galaxies at $z \sim 2$ and earlier (in the far-IR). 

%Cosmic magnetism is an outstanding puzzle of fundamental importance to astrophysics. Magnetic fields are ubiquitous, and their evolution is critically interwoven with the dynamics of the Universe. Hence, it is crucial to understand their origin and the dynamo processes that must have amplified weak  primordial seed fields and maintained their strength across cosmic time \citep{Brandenburg2005}. 

%\comor{need to add words about galactic science} 

\end{document}

%\begin{figure}[!htb]
%\centering
%\includegraphics[width=4cm]{images/example}
%\caption{example}
%\label{fig:im_4}
%\end{figure}
