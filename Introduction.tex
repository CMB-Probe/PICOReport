
\documentclass[PICOReport.tex]{subfiles}

\begin{document}

%\vskip 10pt

The angular power spectra of sky-based $Q$ and $U$ polarization Stokes parameters are commonly recast in terms of curl-free $E$ mode and gradient-free $B$ mode patterns. $E$ modes are generated by either scalar fluctuations, such as density perturbations in the early Universe or by tensor perturbations, such as gravitational waves. $B$ modes are only generated through tensor perturbations. The Probe of Inflation and Cosmic Origins (PICO) is an imaging polarimeter designed to survey the entire sky at frequencies between 21 and 800 GHz with a polarization sensitivity that is 57 or 82 times that of the \planck\ mission for the PICO baseline and \ac{CBE} configurations, respectively. This sensitivity surpasses any other current or planned CMB instrument. 

Fluctuations of the space-time metric during the epoch of inflation, near the Planck time, have generated gravitational waves that embed a unique $B$-mode signature on the polarization of the CMB. A detection of this \ac{IGW} signal "would be a watershed discovery", a quote from the 2010 decadal panel report~\citep{blandford2010}, as it would be our first signal from the epoch of quantum gravity at the beginning of the Universe. The signal would also give strong clues about the nature of inflation, as the $B$-mode signal is proportional to the energy scale of inflation through a parameter commonly labeled $r$, the tensor-to-scalar ratio. The combination of data from \planck\ and the BICEP/Keck Array give the strongest constraint to date $r<0.06\,\, (95\%)$~\citep{2018arXiv181005216A}.
%This limit has already ruled out several models for the inflaton potential~\citep{planck2018inflation}. 
%An intense campaign is ongoing by sub-orbital instruments to reduce the limit on $r$ and perhaps detect the 
%inflationary B-mode signal.  
But the measurements have also revealed that emission within our own galaxy is a source of confusion that must be separated with high fidelity before definitive discovery, or stronger upper limits, can be claimed~\citep{whichplanck}. For the levels of $r$ targeted in the next decade, PICO has the frequency coverage and sensitivity to measure and separate sources of foreground confusion and is thus poised to detect or place unprecedented constraints on the physics of inflation. \comor{Its measurements of the spectral index of primordial fluctuations will give the strongest constraints yet on specific models of inflation. }

A few hundred million years after the Big Bang, the neutral hydrogen gas permeating the Universe was reionized by photons emitted by the first luminous sources to have formed.  The nature of these sources (e.g., star-forming galaxies or high-redshift quasars) and the exact history of this epoch are key missing links in our understanding of structure formation.  Various measurements, including \planck 's measurement of the optical depth to reionization $\tau = 0.054 \pm 0.007$, have indicated that reionization concluded by $z \approx 6$, but its onset at higher redshift is poorly constrained. PICO will yield a breakthrough in this context via a cosmic-variance-limited\footnote{The cosmic variance limit is the statistical limit arising from observing a single universe.}measurement of $\tau$, with $\sigma{\tau}=0.002$, which can only be directly measured in large-scale CMB polarization fluctuations.  The only proven method to date for measuring this signal, which requires exquisite control of systematics and foreground contamination, is a space-based platform.

Lensing of the CMB photons by structures as they traverse the Universe provides a projected map of all the matter in the universe from the epoch of decoupling until today.  The non-zero mass of neutrinos affects the clustering of matter and thus can be inferred from maps of the projected matter distribution. The quantity that can specifically be inferred is the sum of the neutrino masses.  The current constraint from the combination of \planck\ and large-scale structure data is $\sum m_{\nu} < 0.12$ eV (95\%).  This is approaching the minimum summed mass allowed in the inverted neutrino hierarchy of $\approx 0.1$ eV and is within a factor of two of the minimal mass allowed in the normal hierarchy of $\approx 0.06$ eV.  A detection thus appears imminent.  However, the precision of determining the neutrino mass scale, using the CMB or {\it any} other cosmological probe, is limited by knowledge of $\tau$, due to the strong degeneracy between $\tau$ and the amplitude of matter fluctuations.  PICO's map of the projected matter, with \ac{SNR} exceeding 500, {\it and} its own cosmic-variance-limited measurement of $\tau$ will give a $4\sigma$ detection of $\sum m_{\nu}$ in the normal hierarchy, rising to $\sim7\sigma$ for the inverted hierarchy. 

%A direct measurement of $\tau$ via the large-scale $E$-mode polarization signal is thus required in order to break this degeneracy and enable a detection of the sum of the neutrino masses.  The current uncertainty from \planck, $\sigma(\tau) \approx 0.007$, will already limit neutrino mass constraints from cosmological experiments in the next five years; in order to go beyond this, a cosmic-variance-limited measurement with $\sigma(\tau) = 0.002$ must be achieved.  Due to its multifrequency capabilities, all-sky coverage, and excellent control of systematics, PICO is the ideal experiment to achieve this goal.

The CMB also offers a unique window into the {\it thermal} history of the universe, from the time of reheating through today.  
It is during these eras that the matter and radiation that fill the universe were produced and evolved to form the structures 
observed at low redshifts.  Measurements of the CMB on small angular scales are sensitive to the many components 
that make up the universe including the baryons, cosmic neutrinos, dark matter, and a wide variety of particles 
motived by extensions of the Standard Model.  

The Standard Model of particle physics posits three neutrino families, but it also allows for additional light, relativistic particles, if 
they existed early enough during the evolution of the Universe.   We count the total number 
light particles thermalized in the early universe using $\Neff$. Light particles thermalized in the early universe leave a universal contribution to $\Neff$ that is sensitive to the freeze-out temperature and then spin of the particle.  A mission like PICO holds the promise to reach 
back to times when then temperature of the universe was orders of magnitude hotter than we have probe today.  
Such a measurement would shed light on the history of the universe at those very early times and can address 
important questions about the particles and forces in the Standard Model and Beyond.  The history of the universe prior to a few seconds is still largely unexplored observationally and an PICO could reveal important clues to the nature of the fundamental laws and our cosmic origins.

The current measurement of $\Neff = 2.99 \pm 0.17$ from Planck is sensitive to particles thermalized after the QCD phase transitions.  Reaching much earlier time is possible with PICO because of much lower noise levels in polarization.  Larger sky coverage further improves the statistics and compensates for the lower resolution compared to ground based measurements.  These features are advantageous not only for $\Neff$ for any new physics present in the primary CMB and/or lensing potential.  Of particular interest is the nature of dark matter and its interactions, which can be manifest itself in any or all of these probes, depending on the details physics of the dark sector. 


%Measuring CMB polarization at small angular scales will bring many of the questions about the universe into sharp focus.  Of particular interest is the projected improvements to the number of relativistic species, $\Neff$.  Light particles thermalized in the early universe leave a universal contribution to $\Neff$ that is sensitive to the freeze-out temperature and then spin of the particle.  A mission like PICO holds the promise to reach back to times when then temperature of the universe was orders of magnitude hotter than we have probe today.  Such a measurement would shed light on the history of the universe at those very early times and can address important questions about the particles and forces in the Standard Model and Beyond.



\comor{the paragraph below covers a lot of ground, but the anticipated
  impact is not clear. do we want to mention cluster counts? the
  impact of source counts? Gianfranco thinks pico is unique, but this
  is not clear.  Need to say what PICO will do for these topics -
  Nick, thoughts? NB: I gave this a shot}  Secondary anisotropies in the CMB provide
a wealth of information on the growth and evolution of structure in our universe.
CMB lensing, the thermal and kinematic Sunyaev-Zel'dovich (SZ)
effects, and extragalactic point sources all contribute significantly
to the CMB intensity fluctuations on small angular scales (note that
lensing is also present in polarization fluctuations). The all-sky,
projected mass map reconstructed from CMB lensing that PICO will
provide can be correlated with tracers of large-scale structure to
tomographically probe the growth of structure at unprecendented
signal-to-noise. The thermal SZ effect provides a map of the
integrated free electron pressure along the line of sight, and the
peaks of this map trace the locations of all galaxy clusters in the
universe. PICO will find all the massive, virialized, galaxy clusters
at any redshift.  The epoch of reionizaton imprints information in the
statistical moments of the kinematic SZ signal.  The combination of
these statistical moments with its cosmic variance limited $\tau$
measurement, PICO will provide information on the nature of the
sources responsible for reionization.
%Combining thermal and kinematic SZ measurements of galaxies will yield important information about the thermodynamics of galaxy formation, as well as the precise location of the ``missing baryons''. 

%Galsci intro moved from Galsci section (DC 10/30/18) - does this work here?
Pico will provide a full sky catalog of tens of thousands of
extragalactic millimeter and sub-millimeter point sources, which are
beacons for active galactic nuclei (in the radio) and dust emission
from vigorously star-forming galaxies at $z \sim 2$ and earlier (in
the far-IR).
Cosmic magnetism is an outstanding puzzle of fundamental importance to astrophysics. Magnetic fields are ubiquitous, and their evolution is critically interwoven with the dynamics of the universe. Hence, it is crucial to understand
their origin and the dynamo processes that must have amplified weak  primordial seed fields 
and maintained their strength across cosmic time \citep{Brandenburg2005}. 

As often in astrophysics, our understanding is rooted in 
observations of the very local universe: the Milky Way and nearby galaxies. Magnetic fields are observed to be a foremost agent of the 
Milky Way's ecology. They hold keys for making progress on some exciting issues in the astrophysics of galaxies: the dynamics and 
energetics of their multiphase interstellar medium (ISM), the efficiency of star formation, the acceleration and propagation of cosmic rays and the 
impact of feedback on their evolution. Magnetic fields are not only critical for understanding galaxies. 
The magnetized ISM in the Solar Neighborhood presents a challenge for the investigation of cosmological signals. 
Cosmological signals of interest, such as CMB B-mode polarization, CMB spectral distortions, and 21cm line emission from cosmic dawn and reionization are obscured by Galactic dust and synchrotron emission that can be orders of magnitude brighter. Discovery of these signals will require understanding and correcting for the magnetized ISM with unprecedented precision.  

A broad range of science topics call for progress in our modeling of Galactic magnetic fields, which in turn 
motivates ambitious efforts to obtain relevant data. 
As a result, today Galactic magnetism is a dynamic research field, driven by major advances in observational capabilities. PICO will provide a transformative data set to our understanding of this exciting area of exploration.
%\comor{need to add words about galactic science} 

\end{document}

%\begin{figure}[!htb]
%\centering
%\includegraphics[width=4cm]{images/example}
%\caption{example}
%\label{fig:im_4}
%\end{figure}
