\documentclass[PICOReport.tex]{subfiles}

\begin{document}

%To cover: Galaxy Formation, Clusters, Reionization, point sources (probably moves to a new section called 'Legacy Science')

{\bf Physics of Reionization \comor{the formation of the first stars in the universe?}}\\
The reionization of the Universe, which according to current measurements takes place near a Universe age of $\sim$700 million 
years~\citep{planckreion}, imprints multiple signals in the temperature and polarization of the CMB.  In polarization, the most 
important signal is an enhancement of power of the E-mode spectrum at large angular scales $\ell \simlt 20$. 
%scale $E$-mode power sourced by the scattering of the temperature quadrupole during this epoch.  
This signal gives a direct measurement of the optical depth to the reionization epoch $\tau$ with very little degeneracy with other 
cosmological parameters (and, to some extent, a measurement of the reionization history itself).  \comor{not 
sure about the comment in the parenthesis. $\tau$ is in the STM to 'distinguish between models that describe
the formation of the earliest stars' , but that aspect is not highlighted here ...(??). the paragraph also doesn't reference Figure 3, should it?} 
In contrast, inference of $\tau$ from the $TT$ power spectrum is hindered by its direct degeneracy with the scalar fluctuation amplitude.  
Thus, large-scale $EE$ power spectrum measurements are a unique and crucial observable to constrain this parameter.  
If measurements of $\tau$ are not improved beyond the current uncertainties from {\em Planck}, inference of several new aspects 
of cosmological physics (e.g., neutrino mass) will be severely hindered.  PICO is the ideal experiment to make this measurement.  
Its noise level and frequency coverage permit a cosmic-variance-limited constraint on $\tau$, i.e., $\sigma(\tau) \approx 0.002$.  
% We verify this expectation via an explicit forecast following the methodology described in~\citet{errard_feeney_2015}, assuming PICO measurements of the $TT$, $TE$, $EE$, $BB$, and $\phi\phi$ power spectra, with the latter inferred via the iterative $EB$ estimator.  The polarized dust and synchrotron levels match those measured by {\em Planck}~\citep{PlanckFG2015}, including spatial variations, and are cleaned using a parametric maximum-likelihood approach.  Fitting a $\Lambda$CDM$+r$ model, for both the PICO ``requirements'' and ``CBE'' configurations, we find $\sigma(\tau) = 0.002$.  With the exquisite control of systematics needed for the much smaller primordial $BB$ signal, these are not a concern for $EE$ (and hence $\tau$).

In temperature, the most important small-scale imprint is that sourced by the ``patchy'' kinematic Sunyaev-Zel'dovich (kSZ) effect, 
due to the peculiar velocities of free electron bubbles around ionizing sources (e.g., galaxies or quasars).  The total kSZ power 
spectrum receives contributions from both the patchy reionization signal and ``late-time'' sources, e.g., the intergalactic and 
intracluster media.  The reionization and late-time signals are expected to have comparable 
amplitudes~\citep{Shaw2012,MMS2012,Battaglia2013}.  With constraints on the late-time contribution from other 
information (e.g., cross-correlations), effective small-scale foreground removal, and with the primary CMB $TT$ power 
spectrum constrained by inference from the $EE$ power spectrum, it is possible to extract reionization constraints from the 
small-scale kSZ power spectrum~\citep{calabrese/etal/2014}. \comor{if pico is not providing kSZ constraints, only S3 does, there 
is no need to dwell on it at all, I think}

Fig.~\ref{fig:ReionizationPICO} presents forecasts for reionization constraints obtained from PICO's measurement of 
$\tau$ in combination with ground-based Stage-III (CMB-S3) constraints on the kSZ power spectrum (note that PICO will also 
provide some large-scale information on the kSZ power spectrum). \comor{what does the parenthesis mean?} 
Constraints from existing Planck data and observations at 
other wavelengths are also presented.  The PICO measurement of $\tau$ is essential for breaking degeneracies and allowing 
simultaneous precise constraints to be placed on both the mean redshift and duration of reionization.
\comor{Figures 3 has lines for `IGM opacity' and `source efficiency' but those are not explained, and not tied to star formation history 
in the text.}

\begin{figure}
\includegraphics[width=0.5\textwidth]{images/Reionization_Contours_zbar_delz_PICO_NEW.pdf}
\caption{\label{fig:ReionizationPICO} Summary of constraints on the mean redshift and duration of reionization. The forecasts show 68\% and 95\% confidence-level contours for PICO combined with CMB-S3 experiments and Planck combined with CMB-S3 experiments (dark blue and dashed blue, respectively). The solid black lines illustrate how the IGM opacity and source efficiency model parameters map onto this parameter space. The forecasted PICO constraints are compared to: current exclusion limits for the mean redshift of reionization from Planck, shown by the yellow bands \citealp{planck2018:parameters}; recent exclusion limits from the global 21 cm signal measured by EDGES, shown with the red band \citealp{edges2017}; exclusion limits from measurements of the Gunn-Peterson trough from fully absorbed Lyman$\alpha$ in quasar spectra, shown by the grey band \citealp{Fan2006}; exclusion limit on the duration of reionization from Planck and SPT data, shown by the green band \citealp{planck_reio:2016}.}
\end{figure}

In addition to these signals, reionization also leaves specific non-Gaussian signatures in the CMB.  In particular, patchy reionization 
induces non-trivial 4-point functions in both temperature~\citep{SmithFerraro2017} and polarization~\citep{DvorkinSmith2008}.  The 
temperature 4-point function can be used to separate reionization and late-time kSZ contributions.  Combinations of temperature and 
polarization data can be used to build quadratic estimators for reconstruction of the patchy $\tau$ field, analogous to CMB lensing 
reconstruction.  These estimators generally require high angular resolution, but also rely on foreground-cleaned CMB maps.  Thus, 
while PICO alone may not enable high S/N reconstructions, its high-frequency channels --- which have better than 2 arcmin 
resolution and observe at frequencies that have yet to be demonstrated from the ground --- will enable these estimators to be 
robustly applied to ground-based CMB data sets, a strong example of ground-space complementarity.
\comor{if pico is complementarity by {\it only} providing foreground maps at sufficiently high resolution, I think we should move 
this to the 'complementarity'. It is not a direct science goal or outcome.}

% JCH: I checked the Smith-Ferraro 4-pt estimator, and PICO does not have sufficient resolution to do this (see Fig. 3 of https://arxiv.org/pdf/1803.07036.pdf )

The Sections below need to be rearranged to match other Science Objective(s) from the STM, but to also relay the 
breadth of science reachable by PICO, even if those goals are not in the STM. 

{\bf Structure Formation via Gravitational Lensing}

Measurements of the CMB reveal structure imprinted not only at the early time of recombination, but also at nearly every significant ensuing epoch in cosmic history.  In particular the matter between us and the CMB last-scattering surface will deflect the path of CMB photons, a process known as gravitational lensing.  Although the lensing of the CMB is a weak signal, targeted statistical estimators enable its extraction.  The detection of the lensing signal has rapidly progressed, from the first detections in 2007-8 \citep{2007PhRvD..76d3510S, 2008PhRvD..78d3520H} to the recent $40\sigma$ measurement by the {\it Planck} team \cite{2018arXiv180706210P}.  When applied to a rich dataset such as that expected from the PICO satellite, these estimators will provide a map of all the matter in the Universe in projection, with the most sensitivity at redshift $z \simeq 2$ and down to scales of approximately ten arcminutes.  

Forecasts show that the power spectrum of lensing in the PICO CMB map can be detected at approximately 580$\sigma$ or 650$\sigma$ for the requirement or CBE configurations, respectively.  Such high-S/N measurements are more than an order of magnitude improvement over the current state of the art, obtained by the {\it Planck} team.  The legacy value of the PICO CMB lensing map is immense, as has already been seen with {\it Planck}.  It will enable studies of high-redshift structure formation (including the neutrino mass detection described in the previous section), constraints on the properties of quasars and other high-redshift astrophysics, and likely novel probes that have not yet been conceived.  Figure \ref{fig:lensingNoisePICO} shows per-mode noise curves for the reconstruction of CMB lensing from PICO.  

\begin{figure}
\includegraphics[width=0.8\textwidth]{images/lensingNoisePICO.pdf}
\caption{\label{fig:lensingNoisePICO} Lensing noise curves for various experimental configurations, showing the noise per mode of the reconstructed lensing field.  The  signal curve, i.e., the CMB lensing power spectrum, is shown in grey; mapping of the matter density field is possible on scales where the noise curves fall below this signal.  Shown in solid are the performance of the $EB$ polarization estimator, including iterated delensing; this estimator performs best on large angular scales.  On smaller angular scales the temperature ($TT$) estimator shows the best performance. The associated signal to noise ratio for the lensing power spectrum is 570 for the 'requirement' configuration and 650 for the 'CBE' configuration. \textbf{Update with foreground-cleaned polarization noise curve(s) to demonstrate robustness}}
\end{figure}

Lensing breaks the highly symmetric configuration of primordial density fluctuations appearing as $E$ modes on the sky, turning some of these into $B$ modes.  If not accounted for, this effect yields a noise floor on the large-scale $B$ modes that can be measured from the early Universe.  However, maps of $E$ modes and of the lensing field, both of which will be obtained with PICO data, can be used together to create a template for these lensed $B$ modes.  This template can then be subtracted from the measured $B$ modes to obtain improved performance, in a process known as ``delensing'' \citep{2004PhRvD..69d3005S,2012JCAP...06..014S}.  Forecasts show that up to 80\% of the lens-induced $B$ mode power can be removed in the 'requirement' configuration, with this number becoming 85\% for the current best estimate.  

\textbf{To be added:}
\textbf{- CMB halo lensing forecast (Jim, Jean-Baptiste)}
\textbf{- Cross-correlation forecast? (Marcel)}


{\bf Physics of Galaxy Formation via the Sunyaev-Zel'dovich (SZ) Effects}

Not all CMB photons propagate through the universe freely; about 6\% are Thomson-scattered by free electrons in the intergalactic medium (IGM) and intracluster medium (ICM). These scattering events leave a measurable imprint on CMB temperature fluctuations, and they contain a wealth of information from how structure grows to the thermodynamic history of baryons. A fraction of these photons are responsible for the Sunyaev--Zel'dovich effects~\citep{SZ1969,SZ1972}. The thermal SZ effect (tSZ) is the increase in energy of CMB photons due to scattering off hot electrons. This results in a spectral distortion
%, proportinal to the electron pressure,
 of the CMB blackbody that corresponds to a decrement in CMB temperature at frequencies below 217 GHz and an increment at frequencies above. The kSZ effect is the Doppler shift of CMB photons Thomson-scattering off free electrons that have a non-zero peculiar velocity with respect to the CMB rest frame. 
%This produces small shifts in the CMB temperature proportional to the radial velocity of the object and its optical depth.
The amplitudes of the tSZ and kSZ signals are proportional to the integrated electron pressure (tSZ) and momentum (kSZ) along the line of sight, respectively.  They thus contain information about the thermodynamic properties of the IGM and ICM.
%since their magnitudes are proportional to the integrated electron pressure (tSZ) and momentum (kSZ) along the line of sight.
The tSZ effect can be used to measure ensemble statistics of galaxy clusters, which contain cosmological information, as well as to provide uniform cluster samples for galaxy formation studies in dense enviroments.
%
%\item Cosmological parameters from the abundance of tSZ-detected clusters and statistics of component-separated tSZ maps.
%\item Thermodynamic properties of galaxies, groups, and clusters from combined tSZ and kSZ cross-correlation measurements.
%\item Measurements of peculiar velocities, which are powerful cosmological probes on large scales, through the kSZ effect.
%\item Patchy reionization which imprints the CMB through higher order moments of the kSZ effect.

{\bf Galaxy Clusters}

%Through the tSZ PICO will produce a large all sky catalog of 

Galaxy clusters found via the tSZ  effect provide a well-defined sample with a simple-to-model selection function. Sample of clusters such as these are easy to use for cosmological inferences and studies of galaxy evolution in dense environments. 
Points to still hit.
High z sample,
Numbers -- Nick and Jim should cross-check,
most massive cluster all over the whole sky,
Cosmology.

{\bf Compton-$y$ map and tSZ auto-power spectrum}

Describe NILC reconstrution briefly, mention important legacy value of the y-map (c.f. Planck).  Give an example cross-correlation S/N, probably LSST (compare to Planck).

\begin{figure}
\includegraphics[width=0.8\textwidth]{images/PICO_tSZ_PS_plot.pdf}
\caption{\label{fig:PICO_tSZ_PS} Constraints on the tSZ power spectrum from PICO and current data.  The black curve shows the simulated tSZ power spectrum signal.  The light green shaded region shows the error bars for PICO at each multipole, i.e., with no binning, as determined from NILC analysis of full-sky simulations.  The blue points show the current constraints from Planck, which have been averaged into broad multipole bins.  The orange and dark green points show the constraints from ACT and SPT, respectively, at a single multipole of $\ell=3000$.  The overall PICO $S/N = 1270$, nearly two orders of magnitude larger than current measurements.}
\end{figure}


\end{document}

%\begin{figure}[!htb]
%\centering
%\includegraphics[width=4cm]{images/example}
%\caption{example}
%\label{fig:im_3}
%\end{figure}
