\documentclass[PICOReport.tex]{subfiles}

\begin{document}

\begin{table}[tb]
\caption{Relative characteristics of ground, balloon, and space platforms for experiments in the CMB bands.\label{tab:comparison}}
\begingroup
%\openup 5pt
\newdimen\tblskip \tblskip=5pt
\nointerlineskip
\vskip -5mm
\footnotesize %\footnotesize
\setbox\tablebox=\vbox{
    \newdimen\digitwidth
    \setbox0=\hbox{\rm 0}
    \digitwidth=\wd0
    \catcode`*=\active
    \def*{\kern\digitwidth}
%
    \newdimen\signwidth
    \setbox0=\hbox{+}
    \signwidth=\wd0
    \catcode`!=\active
    \def!{\kern\signwidth}
%
\halign to \textwidth{
\hbox to 1.8in{#\leaderfil}\tabskip=0.6em plus 0.6em&
#\hfil&
#\hfil&
#\hfil\tabskip=0pt\cr
\noalign{\doubleline}
\omit\hfil{\bf Characteristic}\hfil&\omit\hfil{\bf Ground}\hfil&\omit\hfil{\bf Balloon}\hfil&\omit\hfil{\bf Space}\hfil\cr
\noalign{\vskip 3pt\hrule\vskip 5pt}
Sky coverage&Partial from single site&Partial from single flight&Full\cr
\noalign{\vskip 3pt}
Frequency coverage&70~GHz inaccessible,$^{a}$&70~GHz inaccessible,$^{a}$&Unrestricted\cr
\omit                       &$\nu \ge 300$\,GHz unusable,&otherwise, almost unlimited&\cr
\omit                       &limited atmospheric windows&&\cr
\noalign{\vskip 3pt}
Angular resolution at 150\,GHz\,$^{b}$& $1.\!'5$ with 6\,m telescope&$6'$ with 1.5\,m telescope&$6'$ with 1.5\,m telescope \cr
\noalign{\vskip 3pt}
Detector noise\,$^{c}$&$265\,\mu$K$_{\rm CMB}\sqrt{\rm s}$& $162\,\mu$K$_{\rm CMB}\sqrt{\rm s}$ & $38\,\mu$K$_{\rm CMB}\sqrt{\rm s}$ \cr
\noalign{\vskip 3pt}
Integration time  &  Unlimited, with interruptions & Weeks, continuous  & Several years, continuous \cr
\noalign{\vskip 3pt}
Repairability, Upgradeability & Good & None; multiple flights possible & None \cr
\noalign{\vskip 5pt\hrule\vskip 3pt}
} % close halign
\noindent {$^{a}$}~70\,GHz is the frequency at which large angular scale $B$-mode Galactic emissions have a minimum (Fig.~\ref{fig:pico-channels-and-fg}).
\qquad {$^{b}$}~We give representative telescope apertures. Significantly larger apertures for balloons and in space result in higher mass, volume, and cost.
\qquad {$^{c}$}~The noise-equivalent temperatures given are illustrative of general capabilities. Detailed comparisons depend on detector heat sink temperatures, bandwidths, and other factors that differ among specific implementations. Ground -- median detector noise at 95\,GHz from BICEP3~\citep{kang20182017}; balloon -- median detector noise at 94\,GHz from SPIDER~\citep{privatecommunication}; space -- 90\,GHz from PICO CBE.    
} % close vbox
\endPlancktable
\endgroup
\end{table}

Since the first \ac{CMB} measurements, more than 50 years ago, important observations have been made from the ground, from balloons, and from space. Each of the CMB satellites flown to date -- \cobe, \wmap, and \planck\ -- has relied on technologies and experience that were the result of sub-orbital efforts. PICO is no different. Examples include: the arrays of micro-fabricated, multi-color pixels and the multiplexed readout that are baselined for the PICO focal plane and that are a consequence of technical developments over the last decade (\S~\ref{sec:focal_plane}, \S~\ref{sec:detector_readout}); and the recent results from \planck \ and ground-based experiments that established the need for a multitude of frequency bands to characterize and control foregrounds.  A healthy sub-orbital program is essential for the success of PICO. 

The phenomenal success and the immense science outcomes of past space missions are a direct consequence of their relative advantages (Table~\ref{tab:comparison}). In every respect, with the exception of repairability and upgradeability, space has the advantage. 
When the entire sky is needed, as for measurements on the largest angular scales, space is by far the most suitable platform.  When broad frequency coverage is needed, space will be required to reach the ultimate limits set by astronomical foregrounds because ground-based observations are limited to a handful of atmospheric windows, all below 300~GHz. Balloons can provide useful information at higher frequencies, but their limited observing time limits \ac{SNR}. The stability offered in space can not be matched on any other platform, and it translates to superb control of systematic uncertainties. There is a consensus within the CMB community that for levels of $r \lesssim 0.001$ the challenges in the measurement are the ability to remove Galactic emissions (\S~\ref{sec:signal_separation}) and to control systematic uncertainties (\S~\ref{sec:systematics}); modern focal plane arrays, like the one employed by PICO, have ample raw sensitivity. The PICO goal of reaching $\sigma(r) = 5 \times 10^{-4}\,(5 \sigma) $ is beyond the reach of sub-orbital observations in the foreseeable future.  

%As Figs.~\ref{fig:clbb} and~\ref{fig:pico-channels-and-fg} demonstrate, Galactic emissions overwhelm the inflationary signal on large and intermediate angular scales ($\ell \leq 150$), and they are significant even at high $\ell$, potentially limiting the process of delensing that is necessary for reaching  $r\lesssim0.001$. The stability offered in space can not be matched on any other platform, and it translates to superb control of systematic uncertainties. There is a consensus within the CMB community that for levels of $r \lesssim 0.001$ the challenges in the measurement are the ability to control systematic uncertainties and to remove Galactic emissions; modern focal plane arrays like the one employed by PICO have ample raw sensitivity. The PICO goal of reaching $\sigma(r) = 5 \times 10^{-4}\,(5 \sigma) $ is beyond the reach of sub-orbital observations in the foreseeable future.  

%Since the first \ac{CMB} measurements, more than 50 years ago, important observations have been made from the ground, from balloons, and from space. Each of the CMB satellites flown to date -- COBE, WMAP, and \planck\ -- has relied crucially on technologies and techniques that were first proved on ground and balloon flights, making these also crucial to the success of PICO. The phenomenal success and the immense science outcomes of past space missions is a direct consequence of their relative advantages (Table~\ref{tab:comparison}). In every respect, with the exception of repairability and upgradeability, space has the advantage. 

%When the entire sky is needed, as for fluctuations on the largest angular scales, space is by far the most suitable platform.  When broad frequency coverage is needed, space will be required to reach the ultimate limits set by astronomical foregrounds. As Figs.~\ref{fig:clbb} and~\ref{fig:pico-channels-and-fg} demonstrate, Galactic emissions overwhelm the inflationary signal on large and intermediate angular scales ($\ell \leq 150$), and they are significant even at high $\ell$, potentially limiting the process of delensing that is necessary for reaching  $r\lesssim0.001$. The stability offered in space can not be matched on any other platform, and it translates to superb control of systematic uncertainties. There is a consensus within the CMB community that for levels of $r \lesssim 0.001$ the challenges in the measurement are the ability to control systematic uncertainties and to remove Galactic emissions; modern focal plane arrays like the one employed by PICO have ample raw sensitivity. The PICO goal of reaching $\sigma(r) = 5 \times 10^{-4}\,(5 \sigma) $ is beyond the reach of sub-orbital observations in the foreseeable future.  

For science requiring higher angular resolution, such as observations of galaxy clusters with $\sim1$ arcmin resolution at 150~GHz, the ground has an advantage. An appropriately large aperture on the ground will also provide high resolution information at lower frequencies, which may be important for separating Galactic emissions at high $\ell$. 

The relative advantages of a space mission used to come with higher costs relative to sub-orbital experiments. However, with the advent of massive ground-based efforts this balance shifts; the costs for a next generation ground-based CMB experiment planned for the next decade are squarely within the cost window of this Probe. A recommended plan for the next decade is therefore to pursue a space mission, and complement it with an aggressive ground program that will overlap in $\ell$ space, and will add science at the highest angular resolution, beyond the reach of a space mission. 


%When the entire sky is needed, as for fluctuations on the largest angular scales, space is necessary.  The difficulties of controlling systematic errors and foregrounds over the whole sky at a level significantly below what has been achieved by Planck are simply too great to overcome on the ground.  Sample variance limited measurements of the reionization bump ($2\leq \ell \leq 12$), whether for $\tau$ or for $r$, require a space mission.  Significant progress can surely be made from sub-orbital experiments on the recombination bump ($30 \leq \ell \leq 300$), and the $r \approx 10^{-3}$ goal of the ``ultimate'' ground-based experiment CMB-S4 looks to be both bold and achievable.  For $r$, the signal resulting from large-scale-structure lensing of CMB $E$-modes into $B$-modes must be measured and removed.  This requires observations on sub-degree scales over a wide frequency range that, especially at the lower frequencies, is a challenge from space.(WCJ:  I agree with this.  I think it should be indicated in Table 1, as a practical limit at low frequencies for a satellite)

%The PICO $r$ goal of $10^{-4}$ is beyond the reach of ground observations.  The limited frequency range observable from the ground is not enough to separate foregrounds to the necessary level, and at $10^{-4}$ there is no room to give up any advantage on systematics.  For science requiring higher angular resolution, however, such as observations of galaxy clusters at 1 arcmin resolution, the ground has a clear advantage.

Balloon observations have been exceedingly valuable in the past. They co-led discoveries of the temperature anisotropy and polarization, provided proving grounds for the technologies enabling the success of \cobe, \wmap, and \planck , and trained the scientists that then led NASA's space missions. Balloon missions will continue to play an important role through making measurements at frequency bands above 280~GHz -- currently there are no plans for any ground program to conduct observations at higher frequencies -- and, because balloon observations are free from the noise induced by atmospheric turbulence, balloons are well suited for probing the low $\ell$ multipoles. The balloon near-space environment is also the best available for elevating detector technologies to TRL6. 
%(d) Training:  balloon experiments continue to be an excellent arena for training the space leaders of tomorrow.
%With flights above 99\% of the atmosphere, balloon-borne observations are free from the noise induced by atmospheric turbulence, making them good platforms for observations of the low $\ell$ multipoles. The balloon near-space environment is the best available for elevating detector technologies to TRL6, and balloon-platforms continue to be an excellent arena for training the space leaders of tomorrow. 


%Balloon observations have been exceedingly valuable in the past. They co-led discoveries of the temperature anisotropy and polarization, provided proving grounds for the technologies enabling the success of COBE, WMAP and \planck , and trained the scientists that then led NASA's space missions. There are specific areas for which balloon missions can continue to play an important role, despite their inherently limited observing time. Balloon payload can access frequency bands above 280~GHz; currently there are no plans for any ground program to conduct observations at higher frequencies. These frequency bands will provide important, and perhaps critical information about polarized emission by  Galactic dust, a foreground that is currently known to limit knowledge of the CMB signals.  With flights above 99\% of the atmosphere, balloon-borne observations are free from the noise induced by atmospheric turbulence, making them good platforms for observations of the low $\ell$ multipoles, and for characterizing foregrounds on these very large angular scales. From a technology point of view, the near-space environment is the best available for elevating detector technologies to TRL6; and balloon-platforms continue to be an excellent arena for training the scientists of tomorrow. 

%Even the ultra-long-duration balloons offer only $10^2$ days per flight, 5--10\% of the duration of the Planck mission (depending on instrument), and 3\% of the duration of WMAP. Also, only one astrophysics payload has flown on this platform.  Both WMAP and Planck showed the essential role of repeated observations in nearly identical conditions in revealing and controlling systematic effects.  This will never be far more difficult with balloon experiments; reaching $10^{-4}$ will require vigorous exploitation of {\it every\/} possible advantage.

%\comblue{say something about other complementarities? }

\end{document}
