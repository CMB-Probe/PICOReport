\documentclass[PICOReport.tex]{subfiles}

\begin{document}


%{\it Should describe complementarity with other astrophysics surveys (both space and ground) and with sub-orbital CMB measurements.  For the astrophysical surveys, this is summary text of text that is presumably sprinkled elsewhere in the proposal; please cross-reference.}

\subsubsection{Complementarity with Astrophysical Surveys in the 2020s}

PICO has strong complementarity with forthcoming surveys. Here we summarize areas of synergy that have been mentioned in a number of earlier sections. 

There is no known way to achieve any cosmological constraint on the sum of the neutrino mass $\sigma(\sum m_\nu)<25\,\mathrm{meV}$ without improving \planck 's measurement of the optical depth $\tau$. In particular, this applies to all methods that rely on comparing low-redshift structures with the amplitude of the CMB at high redshift, such as galaxy clustering, weak lensing, or cluster counts. PICO therefore complements all efforts that probe the late time structure of the Universe; combining PICO with these low-redshift observations extends the scientific reach of all these experiments well beyond what they could achieve on their own.

%[Should add a sentence on ongoing work that attempts to get around $\tau$: still not possible to do better than 25meV even with LSST and CMB-S4.] 
%Improving $\tau$, and therefore $\sigma(\sum m_\nu)$, is only possible by improved observations of low-$\ell$ $E$ modes, which are only possible from space. With its improved $\tau$ measurement PICO will therefore directly improve neutrino mass constraints when combined with late-time probes, reaching $\sigma(\sum m_\nu)<15$ meV.  PICO therefore complements all efforts that probe the late time structure of the Universe to constrain the sum of neutrino masses, and combining PICO with these low-redshift observations extends the scientific reach of all these experiments well beyond what they could achieve on their own.
 
Reconstructing the CMB lensing $\phi$ map on very large angular scales, $L<20$, requires exquisite control of systematic uncertainties over a large sky fraction, with sufficient angular resolution to perform the lensing reconstruction, and with breadth in frequency band to robustly separate Galactic emissions (see Section~\ref{sec:signal_separation}). PICO will provide these, complementing ground-based CMB lensing reconstructions that typically observe a smaller sky fraction, with a smaller number of frequency bands, and without access to the largest angular scales. As discussed in page~\pageref{lensing}, PICO will robustly measure the lensing signal with a power spectrum \ac{SNR} larger than 10 {\it per mode} on very large scales. Such high-significance CMB lensing measurements on the very largest scales will be useful when combined with measurements of galaxy clustering from LSST, Euclid, and SPHEREx (if selected), \comor{include wfirst?} to search for local primordial non-Gaussianity via its scale-dependent effect on galaxy bias; see Section~\ref{sec:fundamental}.
% . In an idealized forecast, we find $\sigma(f_\mathrm{NL})\simeq 0.5$ for $L_\mathrm{min}^{\kappa\kappa,\kappa g,gg}=4$, and $\sigma(f_\mathrm{NL})\simeq 0.9$ for $L_\mathrm{min}^{\kappa\kappa,\kappa g,gg}=20$, assuming optimistic LSST galaxy clustering with $60\,\mathrm{arcmin}^{-2}$ galaxies and with high-redshift dropout galaxies. 
%This would be a notable improvement over the best current constraint $\sigma(f_\mathrm{NL})=5$ from {\it Planck}.  
%While such a measurement would ultimately be limited by limitations of the LSST data on the very largest scales, space based observations of galaxy clustering with Euclid or SPHEREx may enable further improvements.

\comor{what about wfirst? jwst? DESI? CIB?} 

\subsubsection{Complementarity with Sub-Orbital Measurements}

%% start of text %
\newcommand{\sizeA}{0.07\textwidth} %define space between columns below
\begin{table}%[h,t]
\begin{adjustbox}{width=1\textwidth,center}
\begin{tabular}{|c|@{\hspace{\sizeA}}c@{\hspace{\sizeA}}|@{\hspace{\sizeA}}c@{\hspace{\sizeA}}|c|}
\hline
\bf{Characteristic} & {\bf Ground} & {\bf Balloon} & {\bf Space}  \\ \hline
Sky coverage & Partial from single site & Partial from single flight & Full \\ \hline
\multirow{3}{*}{Frequency coverage} & 70~GHz inaccessible$^{a}$    & 70~GHz inaccessible$^{a}$       & \multirow{3}{*}{Unrestricted} \\  
                                                           &  $\nu \ge 300$~GHz unusable   &    otherwise, almost unlimited   &                                            \\  
                                                           &     limited atmospheric windows  &                                      &                                           \\ \hline
  Angular resolution at 150~GHz$^{b}$  & 1.5' with 6~m telescope & 6' with 1.5~m telescope & 6' with 1.5~m telescope \\ \hline
   Detector Noise                                    & $\ge xx$ microK rt(s);  & $\ge xx$ microK rt(s)  &  $ xx$ microK rt(s) \\  \hline
%\multirow{2}{*}{Detector Noise}          & $\ge xx$ microK rt(s);  & $\ge xx$ microK rt(s)  &  $ xx$ microK rt(s) \\  \hline
%                                                           & atmospheric 1/f noise requires subtraction &  & \\ \hline
Integration time                                   & Unlimited & Weeks to a Month & Continuous, for years \\ \hline
Accessibility, repairability                    & Good & None.  Multiple flights possible.& None \\
\hline
\multicolumn{4}{l}{$^{a}$ 70~GHz is the frequency at which large angular scale $B$-mode Galactic emissions have a minimum. } \\
\multicolumn{4}{l}{$^{b}$ We give representative approximate telescope aperture values. Significantly larger apertures for balloons and in space result in higher mass, volume, and cost.  }
\end{tabular}
\end{adjustbox}
\vspace{-0.13in}
\caption{ \small \setlength{\baselineskip}{0.95\baselineskip}
Relative characteristics of ground, balloon, and space platforms for experiments in the CMB bands.\label{tab:comparison} }
\vspace{-0.05in}
\end{table}

Since the first \ac{CMB} measurements, more than 50 years ago, important observations have been made from the ground, from balloons, and from space. Each of the CMB satellites flown to date - COBE, WMAP, and \planck - has relied crucially on technologies and techniques that were first proved on ground and balloon flights, making these also crucial to the success of PICO. The phenomenal success of, and the immense science outcomes produced by, past space missions is a direct consequence of their relative advantages, as listed in 
%The question for the future is what should the roles be of these three types of experimental approaches.
Table~\ref{tab:comparison}. In every respect, with the exception of repairability, space has the advantage. These advantages used to come with higher relative costs. However, with the advent of massive ground-based experiments this balance shifts; the costs for a CMB experiment planned for the next decade are squarely within the cost window of this Probe. We can thus point to the following general guidelines for the next decade. 

When the entire sky is needed, as for fluctuations on the largest angular scales, space is by far the most suitable platform, and for the search for the \ac{IGW} signal it is absolutely necessary. When broad frequency coverage is needed, space will be required to reach the ultimate limits set by astronomical foregrounds.  As Figures~\ref{fig:clbb} and~\ref{fig:pico-channels-and-fg} demonstrate, Galactic emission overwhelms the \ac{IGW} signal on the largest angular scales, and they are dominant even at high $\ell$, potentially limiting the process of delensing that is necessary for reaching levels of $r\lesssim0.001$. The stability offered in space can not be matched on any other platform and translates to superb control of systematic uncertainties. There is a broad consensus within the CMB community that for levels of $r \lesssim 0.001$ the challenges in the measurement are the ability to control systematic uncertainties and to remove Galactic emissions; modern focal plane arrays, like the one employed by PICO have ample raw sensitivity. The PICO $r$ goal of \comor{$10^{-4}$} is beyond the reach of ground observations.  However, for science requiring higher angular resolution, such as observations of galaxy clusters with $\sim1$ arcmin resolution at 150~GHz, the ground has a clear advantage. An appropriately large aperture on the ground will also provide high resolution information at lower frequencies, which may be important for separating Galactic emissions at high $\ell$. 
A recommended plan for the next decade is therefore to pursue a space mission, and complement it with an aggressive ground program that will overlap in $\ell$ space, and will add science at the highest angular resolution, beyond the reach of a space mission. 

%When the entire sky is needed, as for fluctuations on the largest angular scales, space is necessary.  The difficulties of controlling systematic errors and foregrounds over the whole sky at a level significantly below what has been achieved by Planck are simply too great to overcome on the ground.  Sample variance limited measurements of the reionization bump ($2\leq \ell \leq 12$), whether for $\tau$ or for $r$, require a space mission.  Significant progress can surely be made from sub-orbital experiments on the recombination bump ($30 \leq \ell \leq 300$), and the $r \approx 10^{-3}$ goal of the ``ultimate'' ground-based experiment CMB-S4 looks to be both bold and achievable.  For $r$, the signal resulting from large-scale-structure lensing of CMB $E$-modes into $B$-modes must be measured and removed.  This requires observations on sub-degree scales over a wide frequency range that, especially at the lower frequencies, is a challenge from space.(WCJ:  I agree with this.  I think it should be indicated in Table 1, as a practical limit at low frequencies for a satellite)

%The PICO $r$ goal of $10^{-4}$ is beyond the reach of ground observations.  The limited frequency range observable from the ground is not enough to separate foregrounds to the necessary level, and at $10^{-4}$ there is no room to give up any advantage on systematics.  For science requiring higher angular resolution, however, such as observations of galaxy clusters at 1 arcmin resolution, the ground has a clear advantage.

Balloon observations have been exceedingly valuable in the past. They co-lead discoveries of the temperature anisotropy and polarization, provided proving grounds for the technologies enabling the success of COBE, WMAP and \planck , and trained the scientists that then led NASA's space missions. There are specific areas for which balloon missions can continue to play an important role, despite their inherently limited observing time. Balloon payload can access frequency bands above 280~GHz; currently there are no plans for any ground program to conduct observations at higher frequencies. These frequency bands will provide important, and perhaps critical information about polarized emission by  Galactic dust, a foreground that is currently known to limit knowledge of the CMB signals.  With flights above 99\% of the atmosphere, balloon-borne observations are free from the noise induced by atmospheric turbulence, making them good platforms for observations of the low $\ell$ multipoles, and for characterizing foregrounds on these very large angular scales. From a technology point of view, the near-space environment is the best available for elevating detector technologies to TRL6; and balloon-platforms continue to be an excellent arena for training the scientists of tomorrow. 

%Even the ultra-long-duration balloons offer only $10^2$ days per flight, 5--10\% of the duration of the Planck mission (depending on instrument), and 3\% of the duration of WMAP. Also, only one astrophysics payload has flown on this platform.  Both WMAP and Planck showed the essential role of repeated observations in nearly identical conditions in revealing and controlling systematic effects.  This will never be far more difficult with balloon experiments; reaching $10^{-4}$ will require vigorous exploitation of {\it every\/} possible advantage.


\end{document}
