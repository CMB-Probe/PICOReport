
\documentclass[PICOReport.tex]{subfiles}

\begin{document}

%Enumerate the various signals in polarization. Use the frequency band + signals figure. The challenge is to dig out the faintest of all signals, the one due to $r$. This sets the tone for the entire 'signal decomposition' or 'component separation'.  Removing the galactic signal to unmask $r \lesssim 0.001$ is a challenge for all future experiments searching for $r$ at that level, and is a strong advantage of a space platform. The physics of galactic signals suggests complexities in their combined emission properties; the level of this complexity is not known. } 

%\comblue{Let's make clear that there are numerous signals that need to be separated; rather than a 'signal' and 'foregrounds'.} 
Confusion between CMB emission and astrophysical foreground emission of various origins is one of the major challenges that future observations of CMB polarization must face. PICO observes the sky around the peak of the CMB blackbody at about 150\,GHz. Besides the CMB, several astrophysical sources contribute significantly to the total sky brightness and polarization fluctuations around this frequency. 
Polarized radiation, in particular, arises primarily from the synchrotron emission of energetic electrons spiralling in the magnetic field of our own Galaxy, and from thermal emission from elongated interstellar dust grains. The combination of those two components dominates over primordial CMB polarization B-modes by several orders of magnitude, at all frequencies, in all sky directions -- this is the case even at high Galactic latitude, in regions of the sky where Galactic emission is at its minimum.
%
Disentangling all these emissions to detect CMB primordial B-modes is hence expected to be the hardest component separation task that must be performed to reach PICO's main science goals. Demonstrating its feasibility has thus been addressed in priority.

To separate the various sky emissions, one exploits the fact that they arise from different physical processes, which give them different colors. While the CMB emits as a function of frequency as the temperature derivative of a blackbody at 2.725\,K, the frequency scaling of synchrotron is close to a power law with spectral index $\alpha \simeq -1$ (in brightness units), 
\begin{equation}
I_{\rm sync} \propto \nu^{\alpha},
\end{equation}
and that of thermal dust close to a modified blackbody with power law emissivity
\begin{equation}
I_{\rm dust} \propto \nu^{\beta} B_\nu(T_{\rm dust}),
\end{equation}
where $\beta \simeq 1.6$, $T_{\rm dust} \simeq 19$\,K, and $B_\nu(T)$ is the Planck function.

If these emission laws were universal and their parameters well known a priori, and in the absence of any other emission processes, it would suffice to observe the sky at three different frequencies to separate these emissions. However, WMAP and Planck observations have shown that neither emission law is universal: spectral parameters vary with the region of emission. Also, while both emission laws are well-motivated phenomenological descriptions, they are not expected to be exact. At the level of accuracy required by PICO, one must learn from the data themselves, refine these models, and use sophisticated component separation techniques leaving residuals at the relative level of 0.01\%. As an additional complication, whether other subdominant processes can contaminate PICO CMB observations significantly is still not known. The so-called anomalous dust emission, an excess dust emission at frequencies between 10 and 100\,GHz, could be polarized at the sub-percent level. For these reasons, broad frequency coverage is required to reach PICO's goals. Observations at high frequency, in particular, are crucial to understand thermal dust emission, which is the dominant foreground at frequencies  between 100 and 200 GHz, where CMB sensitivity is the best. 

%%%%%%%%%%%%%%%%%%%%%%%%%%%%%%%%%%%
\subsubsection{The PICO component separation challenge}
%%%%%%%%%%%%%%%%%%%%%%%%%%%%%%%%%%%

The baseline design of PICO has 21 channels observing the sky in the 20\,GHz to 800\,GHz frequency range (Fig.~\ref{fig:pico-channels-and-fg}). By analysing how the total emission varies across frequency bands, one can infer the detailed emission properties of the various emission components, form linear combinations of the observations that maximise the contribution of a component of interest while minimising contamination by the others and by instrumental noise, and understand the properties of the foregrounds to evaluate potential residuals in the CMB B-mode map. Various such techniques have been successfully used in previous CMB observations such as those of the Planck mission. Building on this existing expertise, we have carried out map based simulations within a ``data challenge'' framework to assess the capacity of PICO to measure the main signal of interest (CMB primordial B-modes). In this process one group prepares sets of simulated maps for different models of foreground emission of varying complexity from optimistic to pessimistic, which are placed in a shared area. These are then re-analyzed by multiple individuals and groups employing various different component separation algorithms.

\begin{figure}
\includegraphics[width=0.49\textwidth,angle=0]{images/sensitivity_vs_frequency_Jun29th_2018_large.pdf}
\includegraphics[width=0.49\textwidth,angle=0]{images/sensitivity_vs_frequency_Jun29th_2018_2deg.pdf}
\caption{Polarization B modes of Galactic synchrotron and dust, compared to CMB polarization E-modes and B-modes of different origin, for two values of the tensor-to-scalar ration $r$. The location and sensitivity of the PICO frequency channels is shown as vertical bands. Left: integrated r.m.s. emission on the largest angular scales ($2\leq \ell \leq 10$), corresponding to the reionization peak; Right, integrated r.m.s. emission on $\simeq 2^\circ$ angular scales ($50 \leq \ell \leq 150$), corresponding to the expected recombination peak in CMB primordial B modes.}
\label{fig:pico-channels-and-fg}
\end{figure}

%%%%%%%%%%%%%%%%%%%%%%%%%%%%%%%%%%%
%\subsubsection{The PICO simulated observations}
%%%%%%%%%%%%%%%%%%%%%%%%%%%%%%%%%%%

Several different models of foreground emission have been used for the PICO data challenge, from simple Gaussian realizations of synchrotron and dust at the level observed in the BICEP2 field, scaling rigidly in frequency with a single modified blackbody, to models in which the spectral parameters of foregrounds can vary across the sky and along the line of sight, AME can be 2\% polarized, dust polarization can rotate slightly as a function of frequency by reason of projection effects, or the dust SED can depart from a simple modified blackbody. All foreground maps are generated at native HEALPix resolution {\tt nside=512}. They are generated using PySM and/or PSM codes.
%
The CMB is generated as realizations of lensed-$\lambda$CDM from the Planck simulations data set, with primordial B-modes at the $r=0.003$ level. PICO noise is simulated as Gaussian uniform on the sky and uncorrelated from pixel to pixel, at the appropriate levels for each of the 21 PICO bands. 

%%%%%%%%%%%%%%%%%%%%%%%%%%%%%%%%%%%
%\subsubsection{Component separation methods}
%%%%%%%%%%%%%%%%%%%%%%%%%%%%%%%%%%%

Observations from the challenge are analyzed with a variety of well-established component separation methods that rely on various approaches to model the data and to separate the various emissions. These methods can be classified in two broad categories: correlation methods, which exploit the fact that foreground emission is strongly correlated from frequency to frequency, but uncorrelated with the CMB, and parametric methods, which model the sky emission using specific (parametric) emission laws, and use spectral fits in independent pixels or sky regions to infer the amplitude and spectral parameters of each of the components in the sky. Correlation methods include the SEVEM algorithm, and variants of the Internal Linear Combination (ILC) algorithm, such as the needlet space ILC (NILC) and a version generalised to multidmensional components (GNILC). Parametric methods include the Commander algorithm and the X-forecast method.

%%%%%%%%%%%%%%%%%%%%%%%%%%%%%%%%%%%
\subsubsection{Results}
%%%%%%%%%%%%%%%%%%%%%%%%%%%%%%%%%%%

%%%%%%%%%%%%%%%%%%%%%%%%%%%%%%%%%%%
\subsubsection{Discussion}
%%%%%%%%%%%%%%%%%%%%%%%%%%%%%%%%%%%


\end{document}

%\begin{figure}[!htb]
%\centering
%\includegraphics[width=4cm]{images/example}
%\caption{example}
%\label{fig:im_3}
%\end{figure}
