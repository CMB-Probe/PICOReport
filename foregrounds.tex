
\documentclass[PICOReport.tex]{subfiles}

\begin{document}

Contamination of CMB observations by astrophysical foreground emission of various origins is one of the major challenges that future observations of CMB polarization must face. Polarized emission in the PICO frequency range is known to be dominated by synchrotron emission from the galactic interstellar medium in our own galaxy and by thermal emission from interstellar dust. Although much has been learnt about these polarized foregrounds with WMAP and Planck observations, their exact properties are not known at the level that would be needed to guarantee that they can be separated out from the CMB using known component separation techniques. In particular neither the exact form one should assume for the frequency scaling of their emission (if any), nor the variation of these emission laws across the sky or along the line of sight, nor the small-scale distribution of emission, are known at the required level of detail. Whether other processes emit at a level that can contaminate PICO observations significantly is also not known at present. 

The many frequency channels of PICO can be used to learn from the PICO observations themselves the properties of these foregrounds, and to identify the CMB contribution in the multifrequency data using its unique spectral signature. To be robust against our uncertainty about the foregrounds, we assess the feasibility of foreground cleaning for different models of foreground emission of varying complexity, from optimistic to pessimistic, and for various component separation methods. 

%%%%%%%%%%%%%%%%%%%%%%%%%%%%%%%%%%%
\subsubsection{PSM sky simulations at $\boldsymbol{N_{\rm side}=16}$}
%%%%%%%%%%%%%%%%%%%%%%%%%%%%%%%%%%%

The PSM (Planck Sky Model) \cite{delabrouille/etal:2013} is a simulation software developed by the \emph{Planck} Collaboration  to model sky emissions at submillimetre to centimetre wavelengths based on the state-of-the-art observations of the \emph{Planck} satellite mission and earlier surveys. We use the {\sc PSM} to simulate all-sky polarization maps at the 21 frequency bands of \emph{PICO} ($21$ to $800$\,GHz). The simulated components of emission include CMB E- and B-mode polarization, polarized Galactic synchrotron and thermal dust foreground emissions. The CMB template map is generated from CAMB \cite{Lewis/etal:2000} through CMB E- and B-mode power spectra by assuming an optical depth to reionization of $\tau=0.06$ and a tensor-to-scalar ratio of $r = 10^{-3}$, as well as gravitational lensing effects. Galactic dust Q and U polarization maps are generated from the \emph{Planck} {\sc GNILC} 
dust intensity all-sky map at $353$\,GHz , in which cosmic infrared background (CIB), CMB, and noise have been filtered out, and assuming average polarization fractions of $5$-$10$\%. Galactic synchrotron Q and U maps are based on the \emph{WMAP} polarization observations at $23$\,GHz~\cite{Miville-Deschenes/etal:2008}. 

The thermal dust Q and U template maps are extrapolated across \emph{PICO} frequencies through a modified blackbody energy spectrum having variable spectral index and temperature over the sky, using the Planck {\sc GNILC} dust spectral index and temperature maps for which ${\beta_d = 1.6\pm 0.1}$ and ${T_d=19.4 \pm 1.3}$\,K, respectively. The synchrotron Q and U template maps are extrapolated across \emph{PICO} frequencies through a curved power-law spectrum, with a variable spectral index over the sky of ${\beta_s = -3\pm 0.06}$ based on the synchrotron index map from \cite{Miville-Deschenes/etal:2008}, and a constant curvature of $C_s = 0.3$ \cite{Kogut/etal:2007} to account for different populations of cosmic ray electrons  and possible AME polarization \cite{dickinson/etal:2018}. The CMB, dust, and synchrotron component maps in each \emph{PICO} frequency are then co-added, and instrumental white noise is added to each sky map using the sensitivities per frequency quoted by \emph{PICO}.
The PSM simulations are generated at a pixel resolution of HEALPix \cite{gorski/etal:2005} $N_{\rm side}=16$ ($2 \leq \ell \leq 47$), therefore assuming spectral variations of the foregrounds across $N_{\rm side}=16$ pixels.

%%%%%%%%%%%%%%%%%%%%%%%%%%%%%%%%%%%
\subsubsection{\emph{PICO} forecasts with \textsc{Commander}}
%%%%%%%%%%%%%%%%%%%%%%%%%%%%%%%%%%%

We have applied the \textsc{Commander} algorithm \cite{eriksen/etal:2008} to the set of PSM $N_{\rm side}=16$ sky simulations for \emph{PICO}.
\textsc{Commander} is a Bayesian parametric fitting method which has been thoroughly used by the \emph{Planck} Collaboration for the separation of CMB and foreground components \cite{Planck_2015_X,Planck_2018_IV}. Using MCMC Gibbs sampling in each pixel, the \textsc{Commander} algorithm allows to fit simultaneously the amplitudes of CMB and foreground components, their spectral parameters, and the CMB E- and B-mode power spectra in a self-consistent Bayesian framework. A Blackwell-Rao estimator applied to the Gibbs samples allows in particular to reconstruct the statistical (chi-square) distribution of the CMB B-mode power spectrum at each multipole, and the posterior distribution of the tensor-to-scalar ratio \cite{Remazeilles/etal:2016,Remazeilles/etal:2018}. Such a Bayesian method performs end-to-end propagation of all foreground uncertainties towards the tensor-to-scalar ratio, while providing a chi-square goodness-of-fit in each pixel which allows to revise the parametric model or readjust the galactic mask a posteriori. The \emph{PICO} sky maps are processed by \textsc{Commander} using a Galactic mask leaving a $50$\% fraction of the sky, and forecasts on $r$ are computed using the range of low multipoles $2 \leq \ell \leq 47$.

First considering \emph{PICO} sky simulations \emph{without foregrounds}, but just CMB and noise in the \emph{PICO} frequency bands,  we recover the tensor-to-scalar ratio with ${\sigma(r = 10^{-3}) = 0.40 \times 10^{-3}}$ significance: this provides the minimum uncertainty on ${r = 10^{-3}}$ that can be achieved by \emph{PICO} from low multipoles $2 \leq \ell \leq 47$ in the absence of foregrounds on $50$\% of the sky. In the presence of foregrounds (full simulation), with variable spectral indices over the sky, the \textsc{Commander} results on $r$ are of similar quality due to the broad frequency range of \emph{PICO} ($21$ -$800$\,GHz), with ${\sigma(r = 10^{-3}) = 0.41 \times 10^{-3}}$ for \emph{PICOv2-1.4} and ${\sigma(r = 10^{-3}) = 0.36 \times 10^{-3}}$ for \emph{PICOv3} (increased sensitivities per channel) after foreground cleaning. Assuming now that $60$\% delensing in power has been achieved (modified PSM simulation with only $40$\% of the lensing B-mode power left in the CMB map realisation), the uncertainty on ${r=10^{-3}}$ is improved by more than $30$\%, with ${\sigma(r = 10^{-3}) = 0.24 \times 10^{-3}}$ for \emph{PICOv3}  after foreground cleaning and $60$\% delensing. Finally, we note that discarding low- and high-frequency bands, resulting in a descoped version of \emph{PICO} with a narrower frequency range of $43$-$462$\,GHz, the \textsc{Commander} results are degraded by $75$\%, with an uncertainty  increasing from 
${\sigma(r = 10^{-3}) = 0.4 \times 10^{-3}}$  to ${\sigma(r = 10^{-3}) = 0.7 \times 10^{-3}}$ after foreground cleaning. The main reason for the degradation is a lack of constraining power on the dust temperature variations over the sky when discarding frequencies above $462$ GHz, which translates into a degradation of the fitted CMB B-mode power spectrum by extrapolation towards central frequencies. 
\newline\textcolor{red}{[MR: I can add a figure on \textsc{Commander} $C_\ell^{BB}$ results with \emph{PICO}, if needed]}

We are working in parallel on \textsc{Commander2}, a new version of the algorithm operating in harmonic space, thus allowing to perform the parametric fit at full resolution and to further reduce the uncertainty on $r$ by the increase of information / modes.




\end{document}

%\begin{figure}[!htb]
%\centering
%\includegraphics[width=4cm]{images/example}
%\caption{example}
%\label{fig:im_3}
%\end{figure}
