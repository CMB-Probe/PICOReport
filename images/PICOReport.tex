\documentclass[12pt]{article}
\usepackage{amssymb,amsmath,times}
\usepackage[T1]{fontenc}
\usepackage{mathptmx}
\usepackage{xcolor}
\usepackage{graphicx}
\usepackage{fancyhdr}
\usepackage{multirow}
%\usepackage{cite}
\usepackage[numbers,sort&compress]{natbib}
\usepackage{color}
\usepackage{natbib}
\usepackage[nolist]{acronym}
% Soumen's packages
\usepackage{hyperref}
\usepackage[latin1]{inputenc}
\usepackage{grffile}
\usepackage{multirow}
\usepackage{adjustbox}
%subfiles
\usepackage{subfiles}
\usepackage[small,compact]{titlesec}
\usepackage[font=small]{caption}
\usepackage{multirow}
\usepackage{afterpage}

% Letter-size paper

\newlength{\pagewidthA}
\newlength{\pageheightA}
\setlength{\pagewidthA}{8.5in}
\setlength{\pageheightA}{11in}

% Double-page paper

\newlength{\pagewidthB}
\newlength{\pageheightB}
\setlength{\pagewidthB}{17in}
\setlength{\pageheightB}{11in}

\newlength{\stockwidth}
\newlength{\stockheight}

\usepackage{geometry}
\newcommand{\generatePageLayouts}{%
  \newgeometry{layoutwidth=\pagewidthA,layoutheight=\pageheightA, left=1in,right=1in,top=1in,bottom=1in} 
  \savegeometry{LayoutPageA}

  \newgeometry{layoutwidth=\pagewidthB,layoutheight=\pageheightB, left=1in,right=1in,top=1in,bottom=1in}
  \savegeometry{LayoutPageB}
}


\newcommand{\switchToLayoutPageA}{%
  % switch page size first:
  \pdfpagewidth=\pagewidthA \pdfpageheight=\pageheightA % for PDF output
  \paperwidth=\pagewidthA \paperheight=\pageheightA     % for TikZ
  \stockwidth=\pagewidthA \stockheight=\pageheightA % hyperref (memoir)?!
  \loadgeometry{LayoutPageA} % note; \loadgeometry may reset paperwidth/h!
}

\newcommand{\switchToLayoutPageB}{%
  % switch page size first:
  \pdfpagewidth=\pagewidthB \pdfpageheight=\pageheightB % for PDF output
  \paperwidth=\pagewidthB \paperheight=\pageheightB     % for TikZ
  \stockwidth=\pagewidthB \stockheight=\pageheightB % hyperref (memoir)?!
  \loadgeometry{LayoutPageB} % note; \loadgeometry may reset paperwidth/h!
}

% define formatting
%\pagestyle{empty}
\parindent=0pt
\topmargin=0.in \headheight=0in \headsep=-0.1in \textheight=9.2in
\textwidth=6.5in \oddsidemargin=0in

\def\Neff{{N_{\rm eff}}}
\newcommand{\hi}{H{\sc i}~}
\newcommand{\HI}{H{\sc i}}



% define spacings
\def\p{\smallskip}
\def\sp{\vspace{0.15in}}
\def\spa{\vspace{0.3in}}
\def\spaa{\vspace{0.5in}}
\def\spaaa{\vspace{0.7in}}

% define shorthands for latex commands
\def\bei{\begin{itemize}}
\def\eei{\end{itemize}}

% define commonly used symbols
\def\et{{\it et al.\ }}
\def\degr{$^{\circ}$}
\def\arcsec{$^{\prime\prime}$}
\def\pp{\pi}
\def\Neff{{N_{\rm eff}}}
\newcommand{\hi}{H{\sc i}~}
\newcommand{\HI}{H{\sc i}}


% define various names
\def\usk{ }
\def\max{MAX}
\def\maxima{MAXIMA}
\def\boom{BOOMERanG}
\def\arch{Archeops}
\def\maxboom{\maxima/\boom}
\def\planck{{\it Planck}}
\def\combat{COMBAT}
\def\cmb{CMB}
\def\cmba{CMBA}
\def\tic{Ticra}
\def\codef{CODE5}
\def\forecast{FORECAST}
\def\maxipol{MAXIPOL}
\def\hwp{HWP}
\def\ahwp{AHWP}
\def\wmap{{\it WMAP}}
\def\cobe{{\it COBE}}
\def\igb{IGB}
\def\apex{APEX}
\def\ebex{EBEX}
\def\squid{SQUID}
\def\ld{LD}
\def\ldii{LD-II}
\def\blast{BLAST}
\def\pb{\sc polarbear}
\def\pbsa{{\sc polarbear}/SA}
\def\spttg{SPT3G}
\def\ebextw{EBEX2013}
\def\ebexsk{EBEX-IDS}
\def\litebird{LiteBIRD}
\def\bicep{BKA}
\def\biceptwo{BICEP2}
\def\dfmux{DFMux}
\def\xsixf{$\times$64}
\def\xones{$\times$16}
\newcommand{\core}{\textit{\negthinspace CORE\/}}

% for systematics section
\newcommand{\suffix}{pdf} % for pdflatex
\newcommand{\pico}{PICO}
\newcommand{\prang}{\ensuremath{\alpha}}% Polarisation Rotation Angle
\newcommand{\arcmin}{\ensuremath{'}}
\newcommand{\degree}{\ensuremath{^\circ}}
\newcommand{\fsky}{f_{\rm sky}}
\newcommand{\EFH}[1]{\textcolor{red}{$\dagger${[#1]}$\dagger$}}


%define physics and cosmological notations
\def\het{$^{3}$He}
\def\hef{$^{4}$He}
\def\lnt{lN$_{2}$}
\def\wn{cm$^{-1}$}
\def\omeg{$\Omega$}
\def\omegb{$\Omega_{b}$}
\def\hubble{$H_{0}$}
\def\lamb{$\Lambda$}
\def\cl{$C_{\ell}$}
\def\micron{$\mu$m}
\def\microk{$\mu{\mbox{K}}$}
\def\microkrtsec{$\mu{\mbox{K}}\sqrt{\mbox{sec}}$}
\def\microkamin{$\mu{\mbox{K}}\cdot \mbox{arcmin}$}
\def\microkprthz{$\mu{\mbox{K}}/\sqrt{\mbox{Hz}}$}
\def\wattrthz{${\mbox{Watt}}\sqrt{\mbox{Hz}}$}
\def\voltprthz{${\mbox{Volt}}/\sqrt{\mbox{Hz}}$}
\def\sintheta{\mbox{$\sin\theta$}}
\def\bceti{$\beta$-ceti}
\def\etad{$\eta$-draconis}
\def\ruo2{RuO$_{2}$}
\def\tdot{$\dot{\theta}$}
\def\taub{$\tau_{b}$}
\def\degsq{deg$^2$}

% define polarization symbols parameters
\def\It{$I_{t}$}  
\def\sq{$Q$}
\def\su{$U$}
\def\dsu{$\Delta U$}
\def\dsq{$\Delta Q$}
\def\TT{$C_l^{\rm TT}$}
\def\TE{$C_l^{\rm TE}$}
\def\EE{$C_l^{\rm EE}$}
\def\BB{$C_l^{\rm BB}$}

% define math and vectors

\def\mathrelfun#1#2{\lower3.6pt\vbox{\baselineskip0pt\lineskip.9pt
  \ialign{$\mathsurround=0pt#1\hfil##\hfil$\crcr#2\crcr\sim\crcr}}}
\def\simlt{\mathrel{\mathpalette\mathrelfun <}}
\def\simgt{\mathrel{\mathpalette\mathrelfun >}}

\def\hatx{{\bf \hat n}}
\def\hatnprime{{\bf \hat n'}}
\def\hatnone{{\bf \hat n}_1}
\def\hatntwo{{\bf \hat n}_2}
\def\hatni{{\bf \hat n}_i}
\def\hatnj{{\bf \hat n}_j}
\def\vecx{{\bf x}}
\def\veck{{\bf k}}
\def\hatx{{\bf \hat x}}
\def\hatk{{\bf \hat k}}
\def\hatz{{\bf \hat z}}
\def\VEV#1{{\left\langle #1 \right\rangle}}
\def\cP{{\cal P}}
\def\noise{{\rm noise}}
\def\pix{{\rm pix}}
\def\map{{\rm map}}
\long\def\comment#1{}


%------------------------------------------------------------------------------------------------------
% Tables by Charles Lawrence
%------------------------------------------------------------------------------------------------------
\newbox\tablebox    \newdimen\tablewidth
\def\leaderfil{\leaders\hbox to 5pt{\hss.\hss}\hfil}
\def\endPICOtable{\tablewidth=\wd\tablebox
    $$\hss\copy\tablebox\hss$$
    \vskip-\lastskip\vskip -2pt}
\def\tablenote#1 #2\par{\begingroup \parindent=0.8em
    \abovedisplayshortskip=0pt\belowdisplayshortskip=0pt
    \noindent
    $$\hss\vbox{\hsize\tablewidth \hangindent=\parindent \hangafter=1 \noindent
    \hbox to \parindent{$^#1$\hss}\strut#2\strut\par}\hss$$
    \endgroup}
\def\doubleline{\vskip 3pt\hrule \vskip 1.5pt \hrule \vskip 5pt}
%-------------------------------------------------------------------------------------------------------



% \newcommand{\beq}{\begin{equation}}
% \newcommand{\eeq}{\end{equation}}
% \newcommand{\bea}{\begin{eqnarray}}
% \newcommand{\eea}{\end{eqnarray}}
\newcommand\PRL{{\it Phys.~Rev.~Lett.}}
\newcommand\prl{{\it Phys.~Rev.~Lett.}}
\newcommand\ApJ{{\it Ap.~J.}}
\newcommand\apj{{\it Ap.~J.}}
\newcommand\ApJL{{\it Ap.~J.~Lett.}}
\newcommand\apjl{{\it Ap.~J.~Lett.}}
\newcommand\ApJS{{\it Ap.~J.~Suppl.}}
\newcommand\apjs{{\it Ap.~J.~Suppl.}}
\newcommand\PR{{\it Phys.~Rev.}}
\newcommand\PL{{\it Phys.~Lett.}}
\newcommand\MNRAS{{\it MNRAS}}
\newcommand\mnras{{\it MNRAS}}
\newcommand\MNRASL{{\it MNRAS\ Lett.}}
\newcommand\AnA{{\it Astron.~Astrophys.}}
\newcommand\BAAS{{\it Bull.~Am.~Astron.~Soc.}}
\newcommand\NP{{\it Nucl.~Phys.}}
\newcommand\RMP{{\it Rev.~Mod.~Phys.}}
\newcommand\ARAA{{\it ARAA}}
\newcommand\araa{{\it ARAA}}
\newcommand\prd{{\it Phys.~Rev.~D.}}
\newcommand\plb{{\it Phys.~Lett.~B.}}
\newcommand\ao{{\it Appl.~Optics}}
\newcommand\aap{{\it Astron.~Astrophys.}}
\newcommand\aaps{{\it Astron.~Astrophys.~Suppl.}}
\newcommand\pasp{{\it Proc.~Ast.~Soc.~Pac.}}
\newcommand\josa{{\it J.~Opt.~Soc.~Am.}}
\newcommand\phr{{\it Phys. Reports}}
\newcommand\aj{{\it Astronomical Journal}}
\newcommand\jcap{{\it JCAP}}
\newcommand\apss{{\it ApSS}}
\newcommand\nat{{\it Nature}}
\newcommand\procspie{{\it Proceedings of SPIE}}
\newcommand\pasj{{\it Publications of the Astronomical Society of Japan}}
\newcommand\ssr{{\it Space Science Reviews}}
\newcommand\physrep{{\it Physics Reports}}
\newcommand\aapr{{\it Astronomy and Astrophysics Review}}
\newcommand\nar{{\it New Astronomy Reviews}}


\newcommand{\comred}[1]{\textcolor{red}{#1}}
\newcommand{\comblue}[1]{\textcolor{blue}{#1}}
\newcommand{\comgreen}[1]{\textcolor{green}{#1}}
\newcommand{\comor}[1]{\textcolor{orange}{#1}}

\newcommand{\sored}[1]{\comred{\sout{#1}}}
\newcommand{\soor}[1]{\comor{\sout{#1}}}

% Let's you define a command for both text and math mode. 
\newcommand{\wisk}[1]{{\ifmmode{#1}\else{$#1$}\fi}}






\setlength{\floatsep}{0.5\floatsep}
\setlength{\textfloatsep}{0.5\textfloatsep}
\setlength{\intextsep}{0.5\intextsep}
\setlength{\floatsep}{0.5\floatsep}
\setlength{\dblfloatsep}{0.5\dblfloatsep}
\setlength{\dbltextfloatsep}{0.5\dbltextfloatsep}



\begin{document}

% generate page layouts first based on layoutwidth as page size;
  % don't switch actual page sizes yet:
  \generatePageLayouts{}

\bibliographystyle{unsrtnat}

\setlength{\baselineskip}{0.96\baselineskip} %% measured, 5.0 lines/inch.  Can go to 0.96\baselineskip
\setlength{\parskip}{1.\parskip}
 \switchToLayoutPageA{}

%\tableofcontents

\setcounter{page}{0}
\setcounter{figure}{0}

\LARGE{ \centerline{\bf{The Probe of Inflation and Cosmic Origins}}}
\vspace{0.5in}
\Large{ \centerline{A Space Mission Study Report}}
\Large{ \centerline{December, 2018 }}
\vspace{0.5in}
\parindent = 0pt
\large{Principal Investigator:} \\
\large{Steering Committee:} \\
\large{Executive Committee:} \\
\large{Contributors:} \\
\large{Endorsers:} \\

\normalsize

\parindent = 15pt

\newpage
%\twocolumn

%test figure in main file
%


\section{Executive Summary (2 pg, Hanany)} 
\subfile{executive.tex}

%\vspace{-0.05in}

48 remaining pages are distributed 29/19: 29 pages for science (including foregrounds and systematics), 
19 for instrument, technology, mission, management and cost.

%\vspace{-0.18in}

\section{Science}

%\vspace{-0.05in}

\subsection{Introduction (1.5 pgs)}

%\vspace{-0.05in}

%{\it NASA suggested table of contents says Science Intro or Landscape section should include: State of the Art in the Field ; Compelling Outstanding Questions; Needed Capabilities for Progress. }

\subfile{landscape.tex}

%\vspace{-0.18in}

\subsection{Science Objectives (17.5 pgs) } 

%\vspace{-0.05in}

The Science Traceability Matrix can be found in Table~\ref{tab:STM}.

%  The PICO Science Traceability Matrix (2pg, Hanany\&Trangsrud) will be inserted around here.  It is an 11x17 foldout, so it counts as 2 pages, which leaves 15.5 to all the rest in 2.2. Currently allocating 15 pages.  

% [TJP] ------------------------
% [TJP] The Science Traceability Matrix goes here. The table is in stm.tex, but the following LaTeX invocation makes it appear on a double-wide page
\afterpage{%
  % switch to LayoutPageB (includes switching page size)
  \switchToLayoutPageB{}
    \input stm.tex
   \clearpage
% start with LayoutPageA (includes switching page size)
\switchToLayoutPageA{}
}
% [TJP] end --------------------

%    \vspace{12pt}
%     FOR EACH OF THE BELOW SUBSECTIONS:
%    \begin{itemize}
%    \item Introduce and elaborate on the applicable PICO ``Science Objectives" from the STM table (what do they mean and why are they important)
%    \item Observations/Measurements that enable PICO to accomplish each Science Objective (tell the data analysis story that connects the Observations column of the STM to the Science Objective column)
%    \item Contextualize relative to sub-orbital and other space missions. {\it Emphasize where capabilities are unique to space.}
%    \item Science yield estimate (be quantitative. how well will PICO do at Baseline/Required performance? at Current Best Estimate performance?)
%    \item Include a summary plot or table which demonstrates PICO's performance against the Science Objective as written (e.g. how it discriminates between different theories) 
%    \item Perceived science impact. (The impact isn't reducing sigma on a parameter.  It is about what we will learn about nature.)
%    \end{itemize}

\subsubsection{Fundamental Physics (6 pgs)}
\label{sec:fundamentalsci}
%\vspace{-0.05in}

\subfile{fundamentalsci.tex}

%\vspace{-0.05in}

% ------------

\subsubsection{Cosmic Structure Formation and Evolution (4 pgs)}

\subfile{extragalacticsci.tex}

% ------------

\subsubsection{Galactic Structure and Star Formation (3 pgs)}

\subfile{galacticsci.tex}

% ------------

\subsection{Legacy Surveys (2 pgs)} 

{\it Describe science that we get for free. }

\subfile{legacysci.tex}

% ------------

\subsection{Complementarity with Other Surveys and with Sub-Orbital Measurements (1 pg)} 

\subfile{complementarity_sh_qi.tex}

% ------------

\subsection{Signal Separation (4 pages)}

\subfile{foregrounds.tex}

% ------------

\subsection{Systematic Errors (3 pgs)}

\subfile{systematics.tex}

% ------------

\subsection{Measurement Requirements (2 pgs)}

\subfile{requirements.tex}

\newpage

% ------------

\section{Instrument (6 pgs, Hanany \& Transgrud)}

\subfile{instrument.tex}

% ------------

\section{Mission (5 pgs, Trangsrud)}

\subfile{mission.tex}

% ------------

\section{Technology Maturation (4 pgs, O'Brient \& Trangsrud)}

\subfile{technology.tex}

% ------------

\section{Management, Risk, Heritage, and Cost (4 pgs, Trangsrud)}

\subfile{cost.tex}

% ------------

\newpage

\bibliography{mybib}


\begin{acronym}
    %A
    \acro{ACS}{attitude control system}
    \acro{ADC}{analog-to-digital converters}
    \acro{ADS}{attitude determination software}
    \acro{AHWP}{achromatic half-wave plate}
    \acro{AMC}{Advanced Motion Controls}
    \acro{AME}{anomalous microwave emission}
    \acro{ARC}{anti-reflection coatings}
    \acro{ATA}{advanced technology attachment}
    %B
    \acro{BAO}{baryon acoustic oscillations}
    \acro{BRC}{bolometer readout crates}
    \acro{BLAST}{Balloon-borne Large-Aperture Submillimeter Telescope}
    %C
    \acro{CANbus}{controller area network bus}
    \acro{CBE}{current best estimate}
    \acro{CIB}{cosmic infrared background}
    \acro{CMB}{cosmic microwave background}
    \acro{CMM}{coordinate measurement machine}
    \acro{CSBF}{Columbia Scientific Balloon Facility}
    \acro{CCD}{charge coupled device}
    %D
    \acro{DAC}{digital-to-analog converters}
    \acro{DASI}{Degree~Angular~Scale~Interferometer}
    \acro{dGPS}{differential global positioning system}
    \acro{DfMUX}{digital~frequency~domain~multiplexer}
    \acro{DLFOV}{diffraction limited field of view}
    \acro{DSP}{digital signal processing}
    %E
    \acro{EBEX}{E~and~B~Experiment}
    \acro{EBEX2013}{EBEX2013}
    \acro{ELIS}{EBEX low inductance striplines}
    \acro{ETC}{EBEX test cryostat}
    %F
    \acro{FDM}{frequency domain multiplexing}
    \acro{FPGA}{field programmable gate array}
    \acro{FCP}{flight control program}
    \acro{FOV}{field of view}
    \acro{FWHM}{full width half maximum}
    %G
    \acro{GPS}{global positioning system}
    %H
    \acro{HDPE}{high density polyethylene}
    \acro{HIM}{high index materials}
    \acro{HWP}{half-wave plate} 
    %I
    \acro{IA}{integrated attitude}
    \acro{ICM}{intercluster medium}
    \acro{IGM}{intergalactic medium}
    \acro{IGW}{inflationary gravity wave} 
    \acro{ILC}{independent linear combination}
    \acro{IP}{instrumental polarization} 
    \acro{ISM}{interstellar medium}
    %J
    \acro{JSON}{JavaScript Object Notation}
    %L
    \acro{LDB}{long duration balloon}
    \acro{LED}{light emitting diode}
    \acro{LC}{inductor and capacitor}
    \acro{LCS}{liquid cooling system}
    \acro{LZH}{Lazer Zentrum Hannover}
%M
    \acro{MCP}{multi-color pixel}
    \acro{MSM}{millimeter and sub-millimeter}    
    \acro{MLR}{multilayer reflective}
    \acro{MAXIMA}{Millimeter~Anisotropy~eXperiment~IMaging~Array}
    %N
    \acro{NASA}{National Aeronautics and Space Administration}
    \acro{NDF}{neutral density filter}
    %P
    \acro{PCB}{printed circuit board}
    \acro{PE}{polyethylene}
%    \acro{PTFE}{polytetrafluoroethylene}
    \acro{PME}{polarization modulation efficiency}
    \acro{PSF}{point spread function}
    \acro{PV}{pressure vessel}
    \acro{PWM}{pulse width modulation}
    %R
    \acro{RMS}{root mean square}
%S
    \acro{SED}{spectral energy distribution}
    \acro{SLR}{single layer reflective}
    \acro{SMB}{superconducting magnetic bearing}
    \acro{SNR}{signal-to-noise ratio}
    \acro{SOs}{science objectives}
    \acro{SO}{science objective}
    \acro{SQUID}{superconducting quantum interference device}
    \acro{SQL}{structured query language}
    \acro{STARS}{star tracking attitude reconstruction software}
    \acro{SZ}{Sunyaev--Zeldovich}
    \acro{SWS}{sub-wavelength structures}
%T
    \acro{tSZ}{thermal Sunyaev--Zeldovich}
    \acro{TES}{transition-edge-sensor}
    \acro{TDRSS}{tracking and data relay satellites}
   \acro{TM}{transformation matrix}
   \acro{TRL}{Technology Readiness Level}
% U
    \acro{UHMWPE}{ultra high molecular weight polyethylene}   
    \acro{UMN}{University of Minnesota}
    
\end{acronym}


\end{document}

%\begin{figure}[!htb]
%\centering
%\includegraphics[width=4cm]{images/example0}
%\caption{example0}
%\label{fig:im_1}
%\end{figure}

