
\documentclass[PICOReport.tex]{subfiles}

\begin{document}

%\vskip 10pt

Recent theoretical developments and measurements of the \ac{CMB} 
have uncovered tremendous potential for new discoveries over the next 10-20 years. The new discoveries are promising to be 
no less revolutionary then those attained to date. Many of the potential new discoveries are based on deeper
measurements of the spatial pattern of the CMB's polarization. \comor{would like to make this broader to not shortchange 
T-science, but still connect to E, B in the next paragraph}.

The angular power spectra of sky-based $Q$ and $U$ polarization Stokes parameters are commonly recast in terms
of curl-free $E$ mode and gradient-free $B$ mode patterns. $E$ modes are generated by either scalar, such as density, perturbations
in the early Universe or by tensor, such as gravitational wave, perturbations. $B$ modes are only generated through tensor
perturbations. The Probe of Inflation and Cosmic Origins (PICO) is an imaging polarimeter designed to survey the entire 
sky at frequencies between 21 and 800 GHz with 57 times the polarization sensitivity of the {\it Planck} mission, a sensitivity 
surpassing any other current or planned CMB instrument. 

Fluctuations of the space-time metric during the epoch of inflation, near the Planck time, have generated gravitational waves 
that embed a unique B-mode signature on the polarization of the CMB. A detection of the signal "would be a watershed discovery", 
a quote from the 2010 decadal panel report~\citep{nwnh}, as it would be our first signal from the epoch of quantum gravity 
at the beginning of the Universe. The signal would also give strong clues 
about the nature of inflation, as the B-mode signal is proportional to the energy scale of inflation through a parameter 
commonly labeled $r$, the tensor-to-scalar ratio. 
The combination of data from \planck~ and the BICEP/Keck Array give the strongest constraint to date 
$r<0.06\,\, (95\%)$~\citep{rlimit}.  This limit has already ruled out several 
models for the inflaton potential~\citep{planck2018inflation}. 
%An intense campaign is ongoing by sub-orbital instruments to reduce the limit on $r$ and perhaps detect the 
%inflationary B-mode signal.  
But the measurements have also 
revealed that emission within our own galaxy is a source of confusion that must be separated with high fidelity 
before definitive discovery, or stronger upper limits, can be claimed~\citep{whichplanck}. 
PICO has the frequency coverage and sensitivity to measure and separate sources of foreground confusion and is thus 
poised to detect or place unprecedented constraints on the physics of inflation. 
\comor{SO2 in the STEM is about measuring ns and nrun. should we mention them here?} 

\comor{suggest to insert words about star formation history here. This will introduce tau, which we need in the next paragraph} 
A few hundred million years after the Big Bang, the neutral hydrogen gas permeating the Universe was reionized by photons emitted by the first luminous sources to have formed.  The nature of these sources (e.g., star-forming galaxies or high-redshift quasars) and the exact history of this epoch are key missing links in our cosmological knowledge.  Various measurements have indicated that reionization concluded by $z approx 6$, but its onset at higher redshift is poorly constrained.  PICO will yield a breakthrough in this context via a cosmic-variance-limited measurement of $\tau$, the optical depth to reionization, which can only be directly measured in large-scale CMB polarization fluctuations.  \planck found $\tau = 0.054 \pm 0.007$, but PICO will provide $\tau$ with uncertainty $\sigma (\tau) = 0.002$, limited only by cosmic variance~\footnote{The cosmic variance limit is the statistical limit arising from observing a single Universe.}  The only proven method to date for measuring this signal, which requires exquisite control of systematics and foreground contamination, is the space-based CMB platform, as realized in PICO.

%Few hundred million years after the bang the Universe transitions into the `reionization epoch' during which the 
%first stars begin to ionize the Universe. \comor{ the big mysteries are ... and they are answerable through a measurement of 
%$\tau$. Planck gave $\tau = 0.054 \pm 0.007$, but PICO will provide $\tau$ with uncertainty $\sigma (\tau) = 0.002$ limited only by 
%cosmic variance~\footnote{The cosmic variance limit is the statistical limit arising from observing a single Universe.}, thus doing
%these wonderful things ... and add this sentence? - The only proven method to date for measuring this signal, which requires 
%exquisite control of systematics 
%and foreground contamination, is the space-based CMB platform, as realized in PICO.}

Lensing of the CMB photons by structures as they traverse the Universe provides a projected map of all the matter 
in the universe from the epoch of decoupling until today.  The non-zero mass of neutrinos affects the clustering of 
matter and thus can be inferred from maps of the projected matter distribution. The quantity that can specifically be 
inferred is the sum of the neutrino masses.  The current constraint from the combination of \planck (including CMB lensing) and large-scale structure data is $\sum m_{\nu} < 0.12$ eV (95\% C.L.).  This is approaching the minimum summed mass allowed in the inverted neutrino hierarchy ($\approx 0.1$ eV) and is within a factor of two of the minimal mass allowed in the normal hierarchy ($\approx 0.06$ eV).  A detection thus appears imminent.  %\comor{ suggest to say something about current constraints from planck, then continue to what is necessary for improvement} 
However, the precision of determining the neutrino mass scale, using the CMB or {\it any} other cosmological probe, is limited by knowledge of $\tau$, due to the strong degeneracy between $\tau$ and the amplitude of matter fluctuations.  %\comor{remove following sentences from intro? start here} 
% This limitation arises because neutrino mass constraints effectively rely on a comparison of the amplitude of fluctuations at $z=1100$, as inferred from the primary CMB power spectra, to the amplitude at $z \sim 0$, as inferred from large-scale structure probes.  However, the primary CMB only probes a degenerate combination of the amplitude and $\tau$.  
%The exception to this statement is the large-scale E-mode polarization signal sourced by Thomson 
%scattering of CMB photons during the epoch of reionization, which directly measures $\tau$.  
%\comor{end remove sentences from before} 
A direct measurement of $\tau$ via the large-scale E-mode polarization signal is thus required in order to break this degeneracy and enable a detection of the sum of the neutrino masses.  The current uncertainty from \planck, $\sigma(\tau) \approx 0.007$, will already limit neutrino mass constraints from cosmological experiments in the next five years; in order to go beyond this, a 
cosmic-variance-limited measurement with $\sigma(\tau) = 0.002$ must be achieved.  Due to its multifrequency capabilities, all-sky coverage, and excellent control of systematics, PICO is the ideal experiment to achieve this goal.

In addition, the larger sky coverage improves the statistics of both the primary spectra and the lensing maps.  
This information can be combined to give new insights into the nature of dark matter and potential interactions 
with the Standard Model. 
\comor{the last two sentences are about dark matter; should we have a different paragraph on dm? }

\comor{can we recombine the next two paragraphs? what is the state of the art with neff? what is the necessary progress? 
how is pico poised to transform the field?}
The standard model of particle physics posits 3 neutrino families, but it also allows for additional light, relativistic particles, if 
they existed early enough during the evolution of the Universe. The discovery of such particles would revolutionize 
our understanding of the particle content of the Universe, 
and may alleviate the recent tension in different measurements of the Hubble constant~\ref{??}.  We count the total number 
light particles thermalized in the early universe using $\Neff$, which is sensitive to the freeze-out temperature and 
the spin of the particle. The E-mode polarization angular power spectrum is sensitive to the precise value of $\Neff$.  

Light particles thermalized in the early universe leave a universal contribution to $\Neff$ that is sensitive to the 
freeze-out temperature and then spin of the particle.  A mission like PICO holds the promise to reach 
back to times when then temperature of the universe was orders of magnitude hotter than we have probe today.  
Such a measurement would shed light on the history of the universe at those very early times and can address 
important questions about the particles and forces in the Standard Model and Beyond.

%Measuring CMB polarization at small angular scales will bring many of the questions about the universe into sharp focus.  Of particular interest is the projected improvements to the number of relativistic species, $\Neff$.  Light particles thermalized in the early universe leave a universal contribution to $\Neff$ that is sensitive to the freeze-out temperature and then spin of the particle.  A mission like PICO holds the promise to reach back to times when then temperature of the universe was orders of magnitude hotter than we have probe today.  Such a measurement would shed light on the history of the universe at those very early times and can address important questions about the particles and forces in the Standard Model and Beyond.

The CMB also offers a unique window into the {\it thermal} history of the universe, from the time of reheating through today.  
It is during these eras that the matter and radiation that fill the universe were produced and evolved to form the structures 
observed at low redshifts.  Measurements of the CMB on small angular scales are sensitive to the many components 
that make up the universe including the baryons, cosmic neutrinos, dark matter, and a wide variety of particles 
motived by extensions of the Standard Model.  The history of the universe prior to a few seconds is still largely 
unexplored observationally and potentially hides important clues to the nature of the fundamental laws and our cosmic origins.
\comor{what specifically does this paragraph refer to? is it an introduction to the next paragraph?} 

\comor{the paragraph below covers a lot of ground, but the anticipated impact is not clear. do we want to mention
cluster counts? the impact of source counts? Gianfranco thinks pico is unique, but this is not clear.  Need to say what PICO will do for these topics.}
Secondary anisotropies in the CMB provide information on the growth and evolution of structure in our universe. 
CMB lensing, the thermal and kinematic Sunyaev-Zel'dovich (SZ) effects, and extragalactic point sources all 
contribute significantly to the CMB intensity fluctuations on small angular scales (note that lensing is also 
present in polarization fluctuations). The projected mass map reconstructed from CMB lensing can be correlated with tracers of 
large-scale structure to tomographically probe the growth of structure.  The thermal SZ effect provides a map of 
the integrated free electron pressure along the line of sight, and the peaks of this map trace the locations of 
all galaxy clusters in the universe. In addition to $\tau$, the epoch of reionizaton imprints information in the 
statistical moments of the kinematic SZ signal; extracting this signature will provide information on the nature 
of the sources responsible for reionization.  Combining thermal and kinematic SZ measurements of galaxies 
will yield important information about the thermodynamics of galaxy formation, as well as the precise location 
of the ``missing baryons''. Extragalactic point sources are beacons for active galactic nuclei (in the radio) and 
dust emission from vigorously star-forming galaxies at $z \sim 2$ and earlier (in the far-IR).


\comor{need to add words about galactic science} 

\end{document}

%\begin{figure}[!htb]
%\centering
%\includegraphics[width=4cm]{images/example}
%\caption{example}
%\label{fig:im_4}
%\end{figure}
