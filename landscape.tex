
\documentclass[PICOReport.tex]{subfiles}

\begin{document}

%\vskip 10pt

Theoretical developments and measurements of the \ac{CMB} over the last decade 
have uncovered tremendous potential for new discoveries over the next 10-20 years. The new discoveries are promising to be 
no less revolutionary then those attained with the CMB to date. Many of the potential new discoveries are based on 
measurements of the polarization properties of the \ac{CMB}.  (INTRODUCE E, B modes, and PICO)

Fluctuations of the space-time metric during the epoch of inflation, near the Planck time, have generated gravitational waves that embed a 
unique `B-mode' signature on the polarization of the CMB. A detection of the signal "would be a watershed discovery", a quote
from the 2010 decadal panel report, as it would be our first signal from the epoch of quantum gravity at the beginning of the Universe. 
The signal would also give strong clues about the nature of inflation. 
An intense hunt for the signal is ongoing by sub-orbital instruments. These searches have already ruled out several 
models for the inflaton potential. But the measurements, by the \planck\ space mission and other sub-orbital experiments, have also 
revealed that emission by our own galaxy is a source of confusion that must be separated with high fidelity before definitive discovery, or 
stronger upper limits, can be claimed. PICO is poised to detect or place unprecedented constraints the inflationary-induced 
B-mode signal. (WHAT IS THE VALUE OF r TODAY; SAY WHERE YOU ARE GOING)

(SAY SOMETHING ABOUT CONSTRAINING MODELS OF REIONIZATION, INTRODUCE TAU; NEED TAU IN THE 
NEXT PARAGRAPH - Nick / Colin?)

Lensing of the CMB photons by structures as they traverse the Universe provides a projected map of all the matter 
in the universe from the epoch of decoupling until today. 
The non-zero mass of neutrinos affects the clustering of matter and thus can in principle be inferred from maps of the 
projected matter. However, the precision of determining the neutrino mass scale, using the CMB or {\it any} other 
cosmological probe, is limited by knowledge of the optical depth to reioniziation, $\tau$. (EXPLAIN WHY IN 1-2 SENTENCES; 
GIVE THE PLANCK CONSTRAINT; SAY WHAT'S NEEDED TO MAKE PROGRESS - Dan) 
which is measured on the largest angular scales, a unique strength of a satellite like PICO.  

In addition, the larger sky coverage improves the statistics of both the primary spectra and the lensing maps.  
This information can be combined to give new insights into the nature of dark matter and potential interactions 
with the Standard Model. (THE LAST TWO SENTENCES SEEM TO BE ABOUT DM; SHOULD WE HAVE A DIFFERENT 
PARAGRAPH?)

The standard model of particle physics posits 3 neutrino families, but it also allows for additional light, relativistic particles, if 
they existed early enough during the evolution of the Universe. The discovery of such particles would revolutionize our understanding
of the particle content of the Universe, 
and may alleviate the recent tension in different measurements of the Hubble constant~\ref{??}.  We count the total number 
light particles thermalized in the early universe using $\Neff$, which is sensitive to the freeze-out temperature and 
the spin of the particle. The `E-mode' polarization angular power spectrum is sensitive to the precise value of $\Neff$ PICO 
will pl

Light particles thermalized in the early universe leave a universal contribution to $\Neff$ that is sensitive to the freeze-out temperature and then spin of the particle.  A mission like PICO holds the promise to reach back to times when then temperature of the universe was orders of magnitude hotter than we have probe today.  Such a measurement would shed light on the history of the universe at those very early times and can address important questions about the particles and forces in the Standard Model and Beyond.

%Measuring CMB polarization at small angular scales will bring many of the questions about the universe into sharp focus.  Of particular interest is the projected improvements to the number of relativistic species, $\Neff$.  Light particles thermalized in the early universe leave a universal contribution to $\Neff$ that is sensitive to the freeze-out temperature and then spin of the particle.  A mission like PICO holds the promise to reach back to times when then temperature of the universe was orders of magnitude hotter than we have probe today.  Such a measurement would shed light on the history of the universe at those very early times and can address important questions about the particles and forces in the Standard Model and Beyond.


Measuring CMB polarization at small angular scales will bring many of the questions about the universe into sharp focus.  Of particular interest is the projected improvements to the number of relativistic species, $\Neff$.  Light particles thermalized in the early universe leave a universal contribution to $\Neff$ that is sensitive to the freeze-out temperature and then spin of the particle.  A mission like PICO holds the promise to reach back to times when then temperature of the universe was orders of magnitude hotter than we have probe today.  Such a measurement would shed light on the history of the universe at those very early times and can address important questions about the particles and forces in the Standard Model and Beyond. 


%Measuring the CMB polarization on the full sky offers additional opportunities for the measurement of neutrino mass, properties of the dark matter and more.  Lensing of the CMB provides a projected map of all the matter in the universe from reionization until today.  The non-zero mass of neutrinos affects the clustering of matter and can, in principal, be measured from clustering observed in these lensing maps.  Our knowledge of the neutrino mass scale is limited by our knowledge of the optical depth to reioniziation, $\tau$, which is measured on the largest angular scales, a unique strength of a satellite like PICO.  In addition, the larger sky coverage improves the statistics of both the primary spectra and the lensing maps.  This information can be combined to give new insights into the nature of dark matter and potential interactions with the Standard Model.


The CMB also offers a unique window into the {\it thermal} history of the universe, from the time of reheating through today.  It is during these eras that the matter and radiation that fill the universe were produced and evolved to form the structures observed at low-redshifts.  Measurements of the CMB on small angular scales are sensitive to the many components that make up the universe including the baryons, cosmic neutrinos, dark matter, and a wide variety of particles motived by extensions of the Standard Model.  The history of the universe prior to a few seconds is still largely unexplored observationally and potentially hides important clues to the nature of the fundamental laws and our cosmic origins.


CMB secondary anisotropies provides information into the growth and evolution of structure in our universe. CMB lensing, the thermal and kinetic Sunyaev-Zeldovich effects (SZ), and extragalatic point sources all contribute significantly to the CMB intensity fluctuations on small angular scales. The projected mass map from CMB lensing correlated with tracers of large-scale structure tomographically probes the how stucture grows. The thermal SZ provides a map of the integrated free electron pressure along the line of sight, and the peaks of this map trace all the galaxy clusters in the universe. In addition to $\tau$, The epoch of reionizaton inprints information in the statistical moments of the kinetic SZ signal. Extragalactis point sources are becons for active galactic nucluei (in the radio) and dust emission from vigorously star-forming galxies around redshift two and ealier.

\end{document}

%\begin{figure}[!htb]
%\centering
%\includegraphics[width=4cm]{images/example}
%\caption{example}
%\label{fig:im_4}
%\end{figure}
