
\documentclass[PICOReport.tex]{subfiles}

\begin{document}

\vskip 10pt
The CMB also offers a unique window into the {\it thermal} history of the universe, from the time of reheating through today.  It is during these eras that the matter and radiation that fill the universe were produced and evolved to form the structures observed at low-redshifts.  Measurements of the CMB on small angular scales are sensitive to the many components that make up the universe including the baryons, cosmic neutrinos, dark matter, and a wide variety of particles motived by extensions of the Standard Model.  The history of the universe prior to a few seconds is still largely unexplored observationally and potentially hides important clues to the nature of the fundamental laws and our cosmic origins.

Measuring CMB polarization at small angular scales will bring many of the questions about the universe into sharp focus.  Of particular interest is the projected improvements to the number of relativistic species, $\Neff$.  Light particles thermalized in the early universe leave a universal contribution to $\Neff$ that is sensitive to the freeze-out temperature and then spin of the particle.  A mission like PICO holds the promise to reach back to times when then temperature of the universe was orders of magnitude hotter than we have probe today.  Such a measurement would shed light on the history of the universe at those very early times and can address important questions about the particles and forces in the Standard Model and Beyond.

Measuring the CMB polarization on the full sky offers additional opportunities for the measurement of neutrino mass, properties of the dark matter and more.  Lensing of the CMB provides a projected map of all the matter in the universe from reionization until today.  The non-zero mass of neutrinos affects the clustering of matter and can, in principal, be measured from clustering observed in these lensing maps.  Our knowledge of the neutrino mass scale is limited by our knowledge of the optical depth to reioniziation, $\tau$, which is measured on the largest angular scales, a unique strength of a satellite like PICO.  In addition, the larger sky coverage improves the statistics of both the primary spectra and the lensing maps.  This information can be combined to give new insights into the nature of dark matter and potential interactions with the Standard Model.

CMB secondary anisotropies provides information into the growth and evolution of structure in our universe. CMB lensing, the thermal and kinetic Sunyaev-Zeldovich effects (SZ), and extragalatic point sources all contribute significantly to the CMB intensity fluctuations on small angular scales. The projected mass map from CMB lensing correlated with tracers of large-scale structure tomographically probes the how stucture grows. The thermal SZ provides a map of the integrated free electron pressure along the line of sight, and the peaks of this map trace all the galaxy clusters in the universe. In addition to $\tau$, The epoch of reionizaton inprints information in the statistical moments of the kinetic SZ signal. Extragalactis point sources are becons for active galactic nucluei (in the radio) and dust emission from vigorously star-forming galxies around redshift two and ealier.

\end{document}

%\begin{figure}[!htb]
%\centering
%\includegraphics[width=4cm]{images/example}
%\caption{example}
%\label{fig:im_4}
%\end{figure}
