\documentclass[12pt]{article}
\usepackage{amssymb,amsmath,times}
\usepackage{color}
\usepackage{graphicx}
\usepackage{fancyhdr}
\usepackage{multirow}
\usepackage{cite}
\usepackage{color}
\usepackage{longtable}

\usepackage{natbib}
\usepackage[pdftex,
 pdfauthor={PICO Systematics Working Group},
 pdftitle={PICO Systematic Error Report},
 pdfdisplaydoctitle=True]{hyperref}

\title{Probe of Inflation and Cosmic Origins (PICO) Systematic Errors and their Mitigation} 

\author{PICO Systematic Working Group: Brendan Crill, Ranajoy Banerji, \\ Colin Bischoff, Julian Borrill, 
Jacques Delabrouille, Joy Didier, Shaul Hanany, Eric Hivon, \\Brad Johnson, Paolo Natoli, Maurizio Tomasi, Andrea Zonca, et al.}
% define formatting
%\pagestyle{empty}
\parindent=0pt
\topmargin=0.in \headheight=0in \headsep=-0.1in \textheight=9.2in
\textwidth=6.5in \oddsidemargin=0in

\usepackage{array}
\usepackage[table]{xcolor}   


\newcolumntype{L}[1]{>{\raggedright\let\newline\\\arraybackslash\hspace{0pt}}m{#1}}
\newcolumntype{K}[1]{>{\raggedright\centering\arraybackslash}m{#1}}


\newcommand\PRL{{\it Phys.~Rev.~Lett.}}
\newcommand\prl{{\it Phys.~Rev.~Lett.}}
\newcommand\ApJ{{\it Ap.~J.}}
\newcommand\apj{{\it Ap.~J.}}
\newcommand\ApJL{{\it Ap.~J.~Lett.}}
\newcommand\apjl{{\it Ap.~J.~Lett.}}
\newcommand\ApJS{{\it Ap.~J.~Suppl.}}
\newcommand\apjs{{\it Ap.~J.~Suppl.}}
\newcommand\PR{{\it Phys.~Rev.}}
\newcommand\PL{{\it Phys.~Lett.}}
\newcommand\MNRAS{{\it MNRAS}}
\newcommand\mnras{{\it MNRAS}}
\newcommand\MNRASL{{\it MNRAS\ Lett.}}
\newcommand\AnA{{\it Astron.~Astrophys.}}
\newcommand\BAAS{{\it Bull.~Am.~Astron.~Soc.}}
\newcommand\NP{{\it Nucl.~Phys.}}
\newcommand\RMP{{\it Rev.~Mod.~Phys.}}
\newcommand\ARAA{{\it ARAA}}
\newcommand\prd{{\it Phys.~Rev.~D.}}
\newcommand\plb{{\it Phys.~Lett.~B.}}
\newcommand\ao{{\it Appl.~Optics}}
\newcommand\aap{{\it Astron.~Astrophys.}}
\newcommand\aaps{{\it Astron.~Astrophys.~Suppl.}}
\newcommand\pasp{{\it Proc.~Ast.~Soc.~Pac.}}
\newcommand\josa{{\it J.~Opt.~Soc.~Am.}}
\newcommand\phr{{\it Phys. Reports}}
\newcommand\aj{{\it Astronomical Journal}}
\newcommand\jcap{{\it JCAP}}
\newcommand\apss{{\it ApSS}}

\begin{document}
\bibliographystyle{unsrt}

  \maketitle 

\section{Introduction}

Here we list systematic errors that impact the polarization science of the PICO mission concept, describe methods in the literature for mitigating these systematics, and describe specific work we did for the PICO study.  In many cases, we attempted to translate the need for mitigating systematics by spacecraft/mission/instrument design or by measurement into rough requirements for a mission.

We relied a lot on previous thinking for the CORE and Litebird concepts.

\section{Review of Major Systematics}

\begin{table}[h!]
\centering
\scriptsize
 \begin{tabular}{p{3.3cm} p{0.5cm} p{4.2cm} p{4.2cm} p{4.2cm}}
 \hline
\textbf{Name} & \textbf{Risk}&\textbf{Description} & \textbf{State-of-the-art} & \textbf{Additional Mitigation Needed} \\
 \hline
\textbf{Leakage}& &\\
Polarization Angle Calibration\dotfill& 
5&
Uncertainty in polarization calibration leaks E$\to$B.
& 
Knowledge of astrophysical calibrators to 0.3$^\circ$\citet{Aumont+2018}; ground measurement to 0.9$^\circ$ reconstruction to 0.2$^\circ$ using $TB$ and $EB$ demonstrated by \planck\citet{Planck_Lowell}
& 
See Sect.~\ref{sec:angle} for discussion.
\\
 Bandpass Mismatch\dotfill&
 4& 
 Edges and shapes of the the spectral filters vary from detector to detector. leaks T $\to$ P, P $\to$ P if the source's bandpass differs from calibrator's bandpass\cite{Hoang_2017} & Precise bandpass measurement\cite{Pajot_2010};
SRoll algorithm\cite{Planck_Lowell}; filtering technique\cite{CORE_systematics};  additional component solution (see Banerji\& Delabrouille in prep). &
State-of-the-art meet requirements.
   \\
Beam mismatch\dotfill& 
4&
Beam shapes differ between detectors that are combined to reconstruct polarization; leaks T $\to$ P, P $\to$ P
& See Sect.~\ref{sec:angle} & State-of-the-art meet requirements.\\
Time Response Accuracy and Stability\dotfill&
4&
Uncertainty of detector in time constants (measurement errors, time variability) biases polarization angle, pointing and beam size. In a constant spin-rate mission (\pico) is degenerate with the beam shape. leaks T$\to$P, P$\to$P&
On-orbit reconstruction of time response to 0.1\% across a wide signal band\cite{planck2013_vii}, residuals corrected as part of beam and map-making algorithm\cite{Planck_Lowell}.
& State-of-the-art meet requirements.
\\
Readout Cross-talk\dotfill& 
4&
Power in one detector leaks into other detectors
&
\planck's high-impedence bolometers with crosstalk measured at the level of 10$^{-3}$ did not impact CMB polarization science\cite{Planck_Lowell}.  Cross-talk of low-impedence bolometers measured at 0.3\%\cite{BICEP2_II}.
&
State-of-the-art meets requirements.
\\
Chromatic beam shape\dotfill&
4&
Beam shape is a function of source SED: measured using a planet, used to build a window function to correct CMB power spectrum.
&
\planck\ simulations and parameterization as part of the likelihood.
&
Should be further investigated in Phase A of a mission using physical optics simulations.  
\\

Gain mismatch\dotfill&
3&
Relative gain between detectors that are combined to reconstruct polarization; error leaks T$\to$P &
mission-average relative calibration demonstrated to 10$^{-4}$ to 10$^{-5}$ level \cite{Planck_Lowell}
&
State-of-the-art measurement of mission-average gain meets requirements; Sect.~\ref{sec:gain} describes effects of stability in time in relative gains.  
\\


Cross-polarization\dotfill&
3&
Q$\to$U rotation by the optical elements of the instrument.
&
Degenerate with polarization gain calibration.
&
State-of-the-art meets requirements.
\\
\hline 
\textbf{Stability} & & \\
Gain Stability\dotfill& 
5&
Time-variation of detector gain due to time variability of detector heat sink temperature variations and optical loading.
& 
Reconstruction of time variability of gain to 0.2\% in \planck\cite{Planck_Lowell}.
&
See Sect.~\ref{sec:gain}; Gain fluctuations in \pico\ can be calibrated to $<$10$^{-4}$ on time scales of 40 hrs. on the dipole.
\\
Pointing jitter\dotfill&
3&
Random pointing error mixes T, E and B at small angular scale
&
Pointing reconstruction in \planck\ to 0.8 and 1.9 arcsec in-scan and cross-scan \cite{planck2016_l}
&
State-of-the-art meets requirements.
\\

\hline
\textbf{Straylight}& & \\
Far Sidelobes\dotfill& 
5&
Pickup of Galactic signals at large angles from the main beam axis; Spillover can be highly polarized.
& 
\planck\ validated straylight model in anechoic chamber to -80~dBi\cite{Tauber2010}.
&
Design of optical system and baffling, informed by telescope straylight simulations. See Sect.~\ref{sec:fsl} for a study of beams calculated with a physical optics code for the \pico\ telescope and simulated Galactic pickup during the reference mission.\\
 \hline
\textbf{Other} \\
Residual correlated cosmic ray hits\dotfill&
3 &
detectors experience correlated cosmic ray hits below detection threshold resulting in misestimated noise covariance.
&
\planck/HFI found the 5\% percent noise correlation due to this effect did not impact results\cite{Planck_Lowell}. 
&
State-of-the-art detector design to reduce cosmic ray cross-section; State-of-the-art analysis techniques (accounting for correlated noise) meet requirements.
\\
\hline
 \end{tabular}
\caption{\label{tbl:SystematicsList} Systematic errors expected in \pico's measurement of CMB polarization.}
 \end{table}

\begin{center}
 \begin{longtable}{K{3cm} L{4.3cm} L{4.3cm} L{4.3cm}}
 \hline
\textbf{Name} & \textbf{Description} & \textbf{State-of-the-art} & \textbf{Additional Possible Mitigation} \\
 \hline
 \endhead
  \rowcolor{lightgray}
&  \textbf{Pair Differencing} & & \\
 Bandpass Mismatch & 
 Edges and shapes of the the spectral filters vary from detector to detector. leaks T $\to$ P, P $\to$ P leakage if the source's bandpass differs from calibrator's bandpass\cite{Hoang_2017} & Precise bandpass measurement\cite{Pajot_2010};
SRoll algorithm\cite{Planck_Lowell}; filtering technique\cite{CORE_systematics};   &
polarization modulation; full I/Q/U maps for individual detectors mitigates; \textbf{Current techniques may be adequate}  \\
Beam mismatch &
\\
Gain mismatch &
\\
\hline 
  \rowcolor{lightgray}
&  \textbf{Optics} & & \\
Instrumental Polarization Sky T$\to$P & 
\\
Instrumental emission & 
\\
Cross-polarization &
\\
Sidelobes: Reflector Spillover & & &
See Sect.~\ref{sec:fsl}\\
Sidelobes: Stray reflections & & &
See Sect.~\ref{sec:fsl}
\\
Sidelobes: Diffraction & & &
See Sect.~\ref{sec:fsl}
\\
Beam ghosting & & &
See Sect.~\ref{sec:fsl}
\\
Scattering & & &
See Sect.~\ref{sec:fsl}
\\
Pointing systematic error
\\
Pointing jitter
\\
Chromatic beam shape
\\
\hline
  \rowcolor{lightgray}
&  \textbf{Detectors} & & \\
Time Response Accuracy and Stability  &
\\
Readout Cross-talk &
\\
Polarization Angle & & &
See Sect.~\ref{sec:angle}
\\
Gain Stability & & &
See Sect.~\ref{sec:gain}
\\
Detector Non-linearity &
\\
Residual correlated cosmic ray hits &
\\
ADC non-linearity &
\\
 \hline
 \rowcolor{white}
 \caption{\label{tbl:TechnologyGaps} \textbf{Technologies to be developed for exo-Earth direct imaging and characterization.}}
 \end{longtable}
 \end{center}

\section{PICO Reference Mission}

Here we describe the baseline mission concept used to simulate systematic errors.

\subsection{Simulation Tools}

We used toast, quickpol, and other tools.

\section{Highest Priority Systematics}

Here are more details about the three that worried us the most.

\subsection{Far Sidelobes}
\label{sec:fsl}

\subsection{Polarization Angle Calibration}
\label{sec:angle}

\subsection{Time-variable Gain Mismatch}
\label{sec:gain}

\section{Other Systematics}

\subsection{Pointing Accuracy and Reconstruction}

\section{Relation to Other Projects}

What possible overlap with S4.

\newpage

\bibliography{PICO_systematics_memo}

\end{document}


