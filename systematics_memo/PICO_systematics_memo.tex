\documentclass[12pt]{article}
\usepackage{amssymb,amsmath,times}
\usepackage{color}
\usepackage{graphicx}
\usepackage{fancyhdr}
\usepackage{multirow}
\usepackage{cite}
\usepackage{color}
\usepackage{natbib}
\usepackage[pdftex,
 pdfauthor={PICO Systematics Working Group},
 pdftitle={PICO Systematic Error Report},
 pdfdisplaydoctitle=True]{hyperref}

\title{Systematics Error Study for the PICO Probe Study} 

\author{PICO Systematics Working Group: Brendan Crill, etc, etc.}
% define formatting
%\pagestyle{empty}
\parindent=0pt
\topmargin=0.in \headheight=0in \headsep=-0.1in \textheight=9.2in
\textwidth=6.5in \oddsidemargin=0in

\begin{document}
  \maketitle 

\section{Introduction}

Here we list systematic errors that impact the polarization science of the PICO mission concept, describe methods in the literature for mitigating these systematics, and describe specific work we did for the PICO study.  In many cases, we attempted to translate the need for mitigating systematics by spacecraft/mission/instrument design or by measurement into rough requirements for a mission.

We relied a lot on previous thinking for the CORE and Litebird concepts.

\section{Review of Major Systematics Encountered in Previous Missions}

Specifically Planck and any ground-based relevant work.

\section{PICO Reference Mission}

Here we describe the baseline mission concept used to simulate systematic errors.

\subsection{Simulation Tools}

We used toast, quickpol, and other tools.

\section{Highest Priority Systematics}

Include the table of systematic errors from the wiki.

Here are more details about the three that worried us the most.

\subsection{Far Sidelobes}

\subsection{Polarization Angle Calibration}

\subsection{Time-variable Gain Mismatch}

\subsection{Lower Priority Systematics}

\section{Other Systematics}

\subsection{Pointing Accuracy and Reconstruction}

\section{Relation to Other Projects}

What possible overlap with S4.

\bibliographystyle{unsrt}
\end{document}


