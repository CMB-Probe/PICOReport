\documentclass[PICOReport.tex]{subfiles}

%\newcolumntype{L}[1]{>{\raggedright\let\newline\\\arraybackslash\hspace{0pt}}m{#1}}
%\newcolumntype{K}[1]{>{\raggedright\centering\arraybackslash}m{#1}}

\begin{document}

Having flown the \wmap\ and \planck\ space missions and fielded numerous sub-orbital experiments to measure polarization, the mm/sub-mm wavelength community has gained extensive experience with systematic uncertainties that occur in various experimental configurations. A rich literature investigates the types of systematic errors due to the environment, the instrumentation, observation strategies, and data analysis that could confound polarization measurements by creating a bias or an increased variance \cite{hu03,shimon2008,yadav2010,Griffiths2014,LFI_systematics,Kaplan2002,Miller2009,Pagano2009,SO_sys_optical,SO_sys_detector, bicep_systematics,SPIDER_systematics}.%Teams have used the accumulated experience to incorporate technological solutions during the design phase, and to optimize data analysis techniques to identify and compensate for systematic errors. 
%As a consequence, all of the cosmological results reported to date are noise, not systematic uncertainty limited. In several cases systematic effects have been identified in the data, but then suppressed to levels below the noise. 

Just as requirements on signal separation %(\S~\ref{sec:signal_separation}) 
are determined by the need to reach the faint inflationary signal, so are the requirements on control of systematic uncertainties. Since an inflationary $BB$ power spectrum with $r = 5 \times 10^{-4}$ has a peak signal level of 7~nK, systematic effects need to be controlled to a level of 1~nK. It has long been recognized that exquisite control of systematic uncertainties will be required from any experiment attempting to reach levels of $r \lesssim 1\times 10^{-3}$, and it is widely accepted that the stability provided aboard a space platform makes it best suited to control systematic uncertainties compared to other platforms. This is one of the most compelling reasons to observe from space.  As \wmap\ and \planck~ demonstrated, an L2 orbit offers excellent thermal stability, as well as flexibility in the choice of scan strategy.  

Sources of systematic effects and their ultimate degree of severity are a function of the instrument implementation, the spacecraft scan strategy, and mitigation methods developed during the data analysis phase. Thus, a proper assessment requires end-to-end simulation of the mission. Such a simulation should include realistic non-idealities of the spacecraft, telescope, and instrument, and fold in data post-processing and analysis techniques. Developing such a simulation is a significant undertaking, which took years  for the \planck\ mission, and was beyond the scope of this study. We have instead opted to: (1)~implement design features within PICO that would provide strong data redundancy and enable cross-checks during the data analysis~(\S~\ref{sec:systematics_key}); and (2)~enumerate the sources of possible systematic errors, assess their effects, and investigate three that were deemed the highest priority~(\S~\ref{sec:systematics_list}--\S~\ref{sec:fsl}). 

\subsubsection{Potential Systematic Effects}
\label{sec:systematics_list}


The systematic effects faced by PICO can be grouped into three broad categories: (1)~coupling between signals; (2)~stability; and (3)~stray light. For the first category, the most important are the intensity coupling into polarization (both $E$ and $B$) and $E$ coupling into $B$. This is because $T$ (denoting intensity) is approximately ten times stronger than $E$, which is approximately ten times stronger than $B$. The systematic effects are listed in Table \ref{tbl:SystematicsList2col} and were prioritized for further study using a priority level incorporating a PICO Systematics Working Group's assessment of how mission-limiting the effect is, how well these effects are understood by the community, and whether mitigation techniques exist.  

We used simulations to investigate the following three effects that had the highest priority: error in the absolute calibration of polarization angle; error in the relative calibration between orthogonally oriented detectors; and the effect of the telescope sidelobes. We adapted tools developed for \planck~\cite{plank2015_xii_focalplane} and in the context of a European-led mission concept~\citep{core_systematics}. To understand the severity of the effects, we analyzed each in isolation, and in most cases without complicating effects such as inclusion of foreground-separation steps. More detailed studies of the combination of effects and the inclusion of a foreground-separation step are important but are left to the future. 

%\input tables/table2.2.tex
\input tables/table2.2_wide.tex


\subsubsection{Absolute Polarization Angle Calibration}
\label{sec:angle}

In PICO, each of the Stokes $Q$ and $U$ parameters along any line of sight is evaluated through having sensitivity to two orthogonal polarization states. The relative designation of $Q$ and $U$ is derived from having sensitivity to pairs of polarization orientations that are 45\degree\ apart (\S\,\ref{sec:polarimetry}). A systematic error in the implementation (or estimation) of these angles by an amount $\alpha$ causes signals in $Q$ and $U$, and thus in $E$ and $B$, to mix. Because the CMB $E$-mode is much larger than $B$, mixing between $E$ and $B$ leads to the generation of a spurious $BB$ angular power spectrum that mirrors the shape of the $EE$ spectrum~(Fig.~\ref{fig:rot_bb_tb_eb}). The level of spurious $BB$ is proportional to $\alpha^{2} \times EE$. At angular multipoles $\ell \lesssim 100$ a systematic error $\alpha\approx 10\arcmin$ will result in a spurious $BB$ level that is approximately equivalent to $r = 1\times10^{-4}$~\citep{shimon2008,Aumont+2018}.  The mixing of $E$ and $B$ also leads to spurious cross-spectra $EB$ and $TB$, which respectively mimic the $EE$ and $TE$ spectra. 

This systematic error is most usefully split into two contributions: an overall `absolute' error in the assumed instrument's sensitivity to polarization orientations relative to fixed sky coordinates; and a `relative' rotation error between various pairs of detectors. For PICO, the relative rotation of the detectors will be measured to $0.1\arcmin$ by comparing the measured polarization signals between many independent detectors and pairs. However, directly measuring the overall rotation in flight -- which is the process of calibrating the polarization angles --  is challenging, since there are no sufficiently well calibrated polarized astronomical sources. For example, \citet{Aumont+2018} showed that for the best characterized source -- the Crab Nebula -- the current uncertainty of $0.33$\degree\ on the polarization orientation limits measurements to $r \sim 0.01$. 

PICO will overcome this potential source of error 
%through using its high \ac{SNR} measurement of the polarization to identify and reduce it below relevant levels 
in data analysis. \citet{yadav2010} showed that because the $T$ and $E$ signals are much stronger than $B$, an experiment that searches for a specific level of cosmological $BB$ will have high \ac{SNR} for detecting spurious $EB$ and $TB$ cross-spectra arising from error in the calibration of the polarization angle. Applying their method to the PICO baseline specifications we find a constraint of $\alpha<0.2\arcmin$ and $0.6\arcmin$ $(3\sigma)$ using the $EB$ and $TB$ spectra, respectively, and thus a suppression of this systematic effect to negligible levels~(Fig.~\ref{fig:rot_bb_tb_eb}). The constraints quoted include a delensing level of 73\%, which is the PICO forecast including foreground separation. 


\subsubsection{Differential Gain}
\label{sec:gain_stability}

Photometric calibration is the process of converting the raw output of each detector -- typically given in digital readout units -- to physical units via a calibration factor $C(t)$, which is a function of time. One straightforward way for PICO to derive $Q$ and $U$ is through differencing detectors that are sensitive to two orthogonal polarization states $A$ and $B$. A systematic error in the determination of either $A$ or $B$ calibration factors will translate to a biased $Q$ or $U$. We investigated whether the anticipated error on $C_{A,B} (t)$ is adequate for PICO's requirements on measuring the inflationary signal.  
%However, statistical or systematic errors in the determination of $G_{A}(t)$ and $G_{B}(t)$ translate to increased uncertainties, or biased results, respectively.  We investigated whether the anticipated error on $G(t)$ is adequate for PICO's requirements on measuring the faint inflationary signal.  

We assume that the inflationary signal will be extracted from data in the primary CMB bands between 60 and 300~GHz.  Detectors in these bands will be calibrated using measurements of the CMB dipole, a signal that will be measured once per minute as the telescope scans the sky (\S~\ref{sec:survey_design}).  We evaluated the combined impact of the scan strategy and white- and $1/f$ noise in the estimation of $C(t)$.
%\footnote{We use software tools developed for and validated in the context of analysis of the \planck/LFI data.} 
The simulation included signals from the anisotropy of the CMB, including the dipole and $BB$ lensing. Full details of the simulation pipeline are available in the PICO website~\citep{picoweb_dipole}. Figure~\ref{fig:rot_bb_tb_eb} demonstrates that the power spectrum due to error in $C_{A,B} (T)$ is much lower than the PICO requirement of $\sigma(r) = 1\times 10^{-4}$. 
\begin{figure}[thb]
\centerline{
\includegraphics[width=0.375\textwidth]{images/PICO_rotate_eb6_v3.\suffix} 
\hspace{0.3in}
\includegraphics[width=0.41\textwidth]{images/calibration_spectrum_BB.pdf} }
\vspace{-0.1in}
\caption{\captiontext
Two of the initially-estimated highest priority systematic effects for PICO can be suppressed to low levels relative to requirements; here we show inflationary signals with  $r = 1\times 10^{-4}\,\, \mbox{and} \,\, 1\times 10^{-3}$ (solid and dash orange, respectively), and $BB$ lensing (black, theory on the left; realization on right). {\bf Left:} The residual spurious $BB$ spectrum due to 0.2\arcmin\ mis-calibration of PICO's angles of polarization sensitivity (solid blue) has the shape of the $EE$ spectrum, and is small compared to the requirement for $\ell<200$ and compared to the baseline statistical noise level (grey dash). {\bf Right:} Simulated residual $BB$ power after accounting for calibration drifts (solid blue). 
\label{fig:rot_bb_tb_eb} }
\vspace{-0.1in}
\end{figure}

\subsubsection{Far Sidelobes}
\label{sec:fsl}

Differences between the assumed and actual antenna pattern of the detectors will give rise to systematic errors. Such differences are particularly hard to detect in the `far-sidelobes' where the antenna pattern is below the noise level. Unknown far-sidelobe response can couple to bright Galactic signals when the telescope points tens of degrees away from the Galactic plane, and cause spurious signals. To evaluate PICO's susceptibility to this systematic effect we 
%used the same physical optics software as used by \planck\ to 
computed PICO's $4\pi$~sr antenna response for four 155~GHz detectors located at the center of the focal plane. We simulated the time domain response of the detectors as they scan the sky over a year of PICO observations. We convolved their antenna response with a full-sky Galactic emission model~\citep{thorne2018_pysm}, reconstructed maps of $I$, $Q$, and $U$, and calculated the resulting $BB$ angular power spectrum when using a \planck\ Galactic mask excluding 60\% of the sky~\citep{planck_2013_xv}. 

The largest sidelobe in the antenna response is at a level of $-80$~dB from the main lobe. We find that if that sidelobe is known to a level of $-95$~dB (\ac{SNR}$>$20), or further suppressed to that level, the contamination from the sidelobe is a factor of ten below the requirement of $\sigma(r) = 5 \times 10^{-4}$. This suppression can be achieved by adding baffles and through ground-based measurements. \planck 's ground-based measurements mapped the antenna response to levels between $-90$ and $-100$~dB from the main lobe~\citep{planck_sidelobes_IEEE}. The combination of measurements and modeling will be used to remove sidelobe pickup during data analysis. 
 

\subsubsection{Additional Key Findings}
\label{sec:systematics_key}

Properly modeling, engineering for, and controlling systematic effects are key for the success of any experimental endeavor striving to achieve $\sigma(r) \lesssim 1 \times 10^{-3}$. Based on extensive community experience with both hardware and analysis of data we make the following points. \\
$\bullet$ \hspace{0.1in}  Relative to other platforms, a space-based mission provides the most thermally stable platform, and thus the prerequisite for improved control of systematic effects. PICO's orbit at L2 is among the most thermally stable of possible orbits. \\
$\bullet$ \hspace{0.1in} PICO's sky scan pattern gives strong data redundancy, which enables numerous cross-checks. Each of the 12,996 detectors makes independent maps of the $I,\,Q$, and $U$ Stokes parameters enabling many comparisons within and across frequency bands, within and across sections of the focal plane, and within and across bolometers that have either the same or different polarization sensitivities. Half the sky is scanned every two weeks, and the entire sky is scanned in 6 months. Thus combinations of maps constructed at different times during of the mission will be differenced to search for residual time-dependent systematic effects. \\
$\bullet$ \hspace{0.1in}  The scan pattern gives almost continuous scans of planets and large amplitude ($\geq 4$~mK) CMB dipole signals~\citep{picoweb_dipole}. These features result in continuous, high \ac{SNR} calibration and antenna-pattern characterization. In comparison, \planck\ observed each of the planets with only a 6 month cadence and had nearly 100~days/year during which the dipole calibration signals were below 4~mK, at times dipping below 1~mK. \\
$\bullet$ \hspace{0.1in}  We showed that two of the highest priority systematic effects can be controlled to levels that are small compared to requirements. More analysis and planning is required to address systematic uncertainties arising from the far-sidelobe response of the telescope. 

We strongly recommend that further support be provided for further analysis of systematic effects, their combinations, and their coupling with foreground separation. Specifically, support for suborbital efforts is essential to continue the development of means to identify systematic effects, and to develop new techniques to mitigate them. We also endorse support for the development of a complete end-to-end software simulation facility, which is the most robust way to quantify mission trade-offs under the influence of a combination of systematic effects that are coupled to the task of signal separation.  

\end{document}


