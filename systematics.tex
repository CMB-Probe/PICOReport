\documentclass[PICOReport.tex]{subfiles}

%\newcolumntype{L}[1]{>{\raggedright\let\newline\\\arraybackslash\hspace{0pt}}m{#1}}
%\newcolumntype{K}[1]{>{\raggedright\centering\arraybackslash}m{#1}}

\begin{document}

In developing the theoretical framework for measuring the faint CMB
polarization signal, the CMB community has long recognized the impacts
of systematic errors due to instrumentation and data analysis.   A rich literature investigates the types of
systematic errors that confound the polarization measurement and ways
to mitigate them. (Cite Hu(20013)).
% Hu, W., Hedman, M. M., & Zaldarriaga, M. (2003). Benchmark
% parameters for CMB polarization experiments. Physical Review D,
% 67(4), 043004. http://doi.org/10.1103/PhysRevD.67.043004
Each on-orbit measurement of CMB polarization was limited by
sysetmatic errors until an in-depth study of the systematics was
performed and the post-processing data analysis removed them as well
as possible (cite Bennet et al (2013), Planck intermediate XLVI).
% Planck collaboration, et al. (2016). Planck intermediate results - XLVI. Reduction of large-scale systematic effects in HFI polarization maps and estimation of the reionization optical depth. Astronomy and Astrophysics, 596, A107. http://doi.org/10.1051/0004-6361/201628890
% Bennett, C. L., Larson, D., Weiland, J. L., Jarosik, N., Hinshaw, G., Odegard, N., et al. (2013). NINE-YEAR WILKINSON MICROWAVE ANISOTROPY PROBE( WMAP) OBSERVATIONS: FINAL MAPS AND RESULTS. The Astrophysical Journal Supplement Series, 208(2), 20. http://doi.org/10.1088/0067-0049/208/2/20
Additionally, recently proposed CMB missions, such as LiteBird and CORE, have placed
systematic error mitigation at the forefront of the case for their
mission (cite Natoli et al Wallis et al, some litebird paper).
% Wallis, C. G. R., Brown, M. L., Battye, R. A., & Delabrouille,
% J. (2017). Optimal scan strategies for future CMB satellite
% experiments. Monthly Notices of the Royal Astronomical Society,
% 466(1), 425–442. http://doi.org/10.1093/mnras/stw2577

\textbf{Note: Calculate a defensible level of acceptable systematics.
For example, integrate r $10^{-3}$ and $10^{-4}$ power from ell 2-10, 10-50,(do a signal to
noise calculation), then state what map rms that corresponds to.  }

A CMB mission aiming for the unprescedented sensitivity of PICO must control
systematic errors to avoid bias or an increased variance of the
science measurement.   
Systematics must be controlled or corrected to a level that enables the PICO
science goals (better than 1 nanoKelvin (\textbf{Note: make sure this is a
  correct statement - also be careful about reducing a complicated
  concept such as systematic errors to a single map rms number, could also
  quote this as a $\delta$r error}) in the map).
Mitigation of systematic errors is the most important reason (along
with the availability of broad wavelength coverage) to perform a
measurement of the CMB polarization from a space telescope; Compared
with a ground-based, sub-orbital, or even a space mission in 
low-Earth orbit, the L2 environment offers excellent stability as well
as the ability to observe large fractions of the sky on many time
scales without interference from the Sun, Earth, or Moon.
This redundancy of observations allows the checking of consisitency of
results and an improved ability to correct systematic errors in
post-processing analysis.

During the course of the PICO Study, a systematics working group
examined systematic errors affecting PICO,
Most  systematic errors can be mitigated by careful design and engineering of 
the spacecraft and instrument, and the use of present-day state-of-the-art
technology and data analysis tools.   
However, some systematic errors may
limit the precision of the B-mode measurement and the group studied
these in further detail. 
The work was based on the experience of the group's involvement with
past missions, in particular Planck, and in recent detailed studies on
the CORE and LiteBird  concepts

End-to-end simulation of the experiment is an essential tool,
including realistic instabilities and non-idealities of the spacecraft,
telescope, instrument and folding in data post-processing techniques
used to mitigate the effects.  Systematics are coupled with the
spacecraft scan strategy, and the details of the 
data analysis pipeline.  During the study, the PICO team used 
 simulation and analysis tools developed for the Planck mission\cite{plank2015_xii_focalplane} and 
the CORE mission concept, adapting them for PICO.  These tools allowed
a deeper examination of several key systematic errors. 

\subsubsection{List of Systematics}
The systematic errors face by PICO can be categorized into three broad categories 
1) Intensity-to-polarization leakage, 2) stability, and 3)
straylight.    These were prioritized for further study based on the
team's assessment of how well these systematics are understood by the
community, whether mitigation techniques exist - either in instrument
design or in data analysis.

In many cases, these systematic errors

\begin{table}[h!]
\centering
\scriptsize
 \begin{tabular}{p{4.1cm} p{4.1cm} p{4.1cm} p{4.1cm}}
 \hline
\textbf{Name} & \textbf{Description} & \textbf{State-of-the-art} & \textbf{Additional Possible Mitigation} \\
 \hline
\textbf{Leakage} & &\\
 Bandpass Mismatch & 
 Edges and shapes of the the spectral filters vary from detector to detector. leaks T $\to$ P, P $\to$ P leakage if the source's bandpass differs from calibrator's bandpass\cite{Hoang_2017} & Precise bandpass measurement\cite{Pajot_2010};
SRoll algorithm\cite{Planck_Lowell}; filtering technique\cite{CORE_systematics};   &
polarization modulation; full I/Q/U maps for individual detectors
                                                                                     mitigates;
                                                                                     additional
                                                                                     component
                                                                                     solution
                                                                                     (see
                                                                                     Banerji\&
                                                                                              Delabrouille
                                                                                               (in
                                                                                             prep)
                                                                                               \textbf{Current
                                                                                     techniques may be adequate}  \\
Beam mismatch & & See Sect.~\ref{sec:angle}
\\
Gain mismatch &
\\
Time Response Accuracy and Stability  &
\\
Readout Cross-talk & &
\\
Polarization Angle & & &
See Sect.~\ref{sec:angle}
\\

Cross-polarization &
\\
Chromatic beam shape
\\
\hline 
\textbf{Stability} & & \\

Pointing jitter
\\

Gain Stability & & &
See Sect.~\ref{sec:gain}
\\
\hline
\textbf{Straylight} & & \\
Far Sidelobes& & &
See Sect.~\ref{sec:fsl}\\
 \hline
\textbf{Other} \\
Residual correlated cosmic ray hits &
\\
\hline
 \end{tabular}
\caption{\label{tbl:SystematicsList} Systematic errors expected to affect PICO.}
 \end{table}

\subsubsection{Absolute polarization angle calibration}
\label{sec:angle}
%An error in the angle of polarization sensitivity of the
%polarization-sensive detectors  mixes E-mode and B-mode polarization
%and can thus confuse a measurement of primoridal gravitational wave
%signature. The polarization direction must be calibrated to high
%precision.  The relative angle between polarimeters can be determined
%to very good precision but an absolute angle must be calibrated
%ref). The ground-based calibration of Planck/HFI (Rosset et al 
%2010) achieved better than 1$^\circ$.  PICO requires XXX$^\circ$
%precision which can instead by detemied by .
\textbf{Eric H to write up}

\subsubsection{Gain Stability}
\label{sec:gain}
\textbf{Maurizio to write up}

\subsubsection{Far Sidelobe Pickup}
\label{sec:fsl}
...

\subsubsection{Key Findings}
Understanding and controlling the effects of systematic errors in a
next-generation CMB probe is critical.

The raw sensitivity of the instrument should include enough margin
that data subsets can independently archieve the science goals.
This allows testing of the results in the data analysis and additional
data cuts, if needed.

In a PICO mission's phase A, a complete end-to-end system-level
simulation software facility would be developed to assist the team in setting 
requirements and conducting trades between subsystem requirements while
realistically accounting for post-processing mitigation.  Any future
CMB mission is likely to have similar orbit  
and scan characteristics to those of PICO, thus there is an opportunity for NASA and
the CMB community to invest in further development of this capability now.


\end{document}

%\begin{figure}[!htb]
%\centering
%\includegraphics[width=4cm]{images/example}
%\caption{example}
%\label{fig:im_3}
%\end{figure}
