\documentclass[PICOReport.tex]{subfiles}

%\newcolumntype{L}[1]{>{\raggedright\let\newline\\\arraybackslash\hspace{0pt}}m{#1}}
%\newcolumntype{K}[1]{>{\raggedright\centering\arraybackslash}m{#1}}

\begin{document}

In developing the theoretical framework for measuring the faint CMB
polarization signal, the CMB community has long recognized the impacts
of systematic errors due to instrumentation and data analysis.   A rich literature investigates the types of
systematic errors that confound the polarization measurement and ways
to mitigate them\cite{hu03}. 
Each on-orbit measurement of CMB polarization was limited by
sysetmatic errors until an in-depth study of the systematics was
performed and the post-processing data analysis removed them as well
as possible\cite{Bennett13,planck2016_xlvi}. 
Additionally, recently proposed CMB missions, such as LiteBird and CORE, have placed
systematic error mitigation at the forefront of the case for their
mission\cite{hazumi2012,wallis2017,Natoli2018}.

\textbf{Note: Calculate a defensible level of acceptable systematics.
For example, integrate r $10^{-3}$ and $10^{-4}$ power from ell 2-10, 10-50,(do a signal to
noise calculation), then state what map rms that corresponds to.  }

A CMB mission aiming for the unprescedented sensitivity of PICO must control
systematic errors to avoid bias or an increased variance of the
science measurement.   
Systematics must be controlled or corrected to a level that enables the PICO
science goals (better than 1 nanoKelvin (\textbf{Note: make sure this is a
  correct statement - also be careful about reducing a complicated
  concept such as systematic errors to a single map rms number, could also
  quote this as a $\delta$r error}) in the map).
Mitigation of systematic errors is the most important reason (along
with the availability of broad wavelength coverage) to perform a
measurement of the CMB polarization from a space telescope; Compared
with a ground-based, sub-orbital, or even a space mission in 
low-Earth orbit, the L2 environment offers excellent stability as well
as the ability to observe large fractions of the sky on many time
scales without interference from the Sun, Earth, or Moon.
This redundancy of observations allows the checking of consisitency of
results and an improved ability to correct systematic errors in
post-processing analysis.

During the course of the PICO Study, a systematics working group
examined systematic errors affecting PICO,
Most  systematic errors can be mitigated by careful design and engineering of 
the spacecraft and instrument, and the use of present-day state-of-the-art
technology and data analysis tools.   
However, some systematic errors may
limit the precision of the B-mode measurement and the group studied
these in further detail. 
The work was based on the experience of the group's involvement with
past missions, in particular Planck, and in recent detailed studies on
the CORE and LiteBird  concepts

End-to-end simulation of the experiment is an essential tool,
including realistic instabilities and non-idealities of the spacecraft,
telescope, instrument and folding in data post-processing techniques
used to mitigate the effects.  Systematics are coupled with the
spacecraft scan strategy, and the details of the 
data analysis pipeline.  During the study, the PICO team used 
 simulation and analysis tools developed for the Planck mission\cite{plank2015_xii_focalplane} and 
the \core mission concept, adapting them for PICO.  These tools allowed
a deeper examination of several key systematic errors. 

\subsubsection{List of Systematics}
The systematic errors face by PICO can be categorized into three broad categories 
1) Intensity-to-polarization leakage, 2) stability, and 3)
straylight.    These were prioritized for further study based on the
team's assessment of how well these systematics are understood by the
community, whether mitigation techniques exist - either in instrument
design or in data analysis.

\begin{table}[h!]
\centering
\scriptsize
 \begin{tabular}{p{4.1cm} p{4.1cm} p{4.1cm} p{4.1cm}}
 \hline
\textbf{Name} & \textbf{Description} & \textbf{State-of-the-art} & \textbf{Additional Possible Mitigation} \\
 \hline
\textbf{Leakage} & &\\
 Bandpass Mismatch & 
 Edges and shapes of the the spectral filters vary from detector to detector. leaks T $\to$ P, P $\to$ P leakage if the source's bandpass differs from calibrator's bandpass\cite{Hoang_2017} & Precise bandpass measurement\cite{Pajot_2010};
SRoll algorithm\cite{Planck_Lowell}; filtering technique\cite{CORE_systematics};   &
polarization modulation; full I/Q/U maps for individual detectors
                                                                                     mitigates;
                                                                                     additional
                                                                                     component
                                                                                     solution
                                                                                     (see
                                                                                     Banerji\&
                                                                                              Delabrouille
                                                                                               (in
                                                                                             prep)
                                                                                               \textbf{Current
                                                                                     techniques may be adequate}  \\
Beam mismatch & & See Sect.~\ref{sec:angle}
\\
Gain mismatch &
\\
Time Response Accuracy and Stability  &
\\
Readout Cross-talk & &
\\
Polarization Angle & & &
See Sect.~\ref{sec:angle}
\\

Cross-polarization &
\\
Chromatic beam shape
\\
\hline 
\textbf{Stability} & & \\

Pointing jitter
\\

Gain Stability & & &
See Sect.~\ref{sec:gain}
\\
\hline
\textbf{Straylight} & & \\
Far Sidelobes& & &
See Sect.~\ref{sec:fsl}\\
 \hline
\textbf{Other} \\
Residual correlated cosmic ray hits &
\\
\hline
 \end{tabular}
\caption{\label{tbl:SystematicsList} Systematic errors expected to affect PICO.}
 \end{table}

\subsubsection{Absolute polarization angle calibration}
\label{sec:angle}

The rotation of the CMB polarization can have different causes,
including 1. a birefringent primordial Universe, or a Faraday rotation
due a primordial magnetic field \citep{Pogosian+2018}, 2. birefringent
foregrounds, or interaction with the Galactic magnetic field,
3. systematic effects in the instrument, and in particular an error on
the actual direction of polarization measured by each detector.  
While the first two sources create a rotation that may depend on scale,
position and/or frequency, the latter is expected to mostly depend on
the detector considered. 

A rotation {\prang} of the direction of polarization mixes the $Q$ and $U$ Stokes parameters via
$Q\pm iU \longrightarrow e^{\mp i 2 \prang} (Q\pm iU)$
and affects the power spectra via (assuming the rotation to be independent on scale and location)
\begin{subequations}
\begin{align}
C^{TT}_\ell &\longrightarrow & C^{TT}_\ell                                             &= & C^{TT}_\ell \\
C^{TE}_\ell &\longrightarrow & \cos 2\prang\  C^{TE}_\ell                                &\sim & \left(1 - 2\prang^2\right)\ C^{TE}_\ell \\
C^{EE}_\ell &\longrightarrow & \cos^2 2\prang\  C^{EE}_\ell + \sin^2 2\prang\  C^{BB}_\ell &\sim & C^{EE}_\ell - 4\prang^2\ \left(C^{EE}_\ell - C^{BB}_\ell\right) \\
C^{BB}_\ell &\longrightarrow & \sin^2 2\prang\  C^{EE}_\ell + \cos^2 2\prang\  C^{BB}_\ell &\sim & C^{BB}_\ell + 4\prang^2\ \left(C^{EE}_\ell - C^{BB}_\ell\right)\\
C^{TB}_\ell &\longrightarrow & \sin 2\prang\  C^{TE}_\ell                                &\sim & 2\prang\  C^{TE}_\ell \\
C^{EB}_\ell &\longrightarrow & \sin 2\prang \cos 2\prang \left(C^{EE}_\ell -  C^{BB}_\ell\right)  &\sim & 2\prang\ \left(C^{EE}_\ell -  C^{BB}_\ell\right)
\end{align}
\end{subequations}
as illustrated in Fig.~\ref{fig:rot_bb_tb_eb}.

%------------------------------------------------------------------------------------------
\begin{figure}[htb]
\includegraphics[width=0.8\textwidth]{images/PICO_rotate_eb2_v0.\suffix}
\caption{\label{fig:rot_bb_tb_eb} Effect of a rotation of the angle of polarization, assuming the Planck 2018 $\Lambda$-CDM best fit model \citep{Planck2018_VI} and expected \pico performances, with a perfect delensing \EFH{(the beam related systematics effects are still arbitrary; to be improved or removed)}.}
\end{figure}
%------------------------------------------------------------------------------------------
%
%------------------------------------------------------------------------------------------
\begin{figure}[htb]
\includegraphics[width=0.40\textwidth]{images/PICO_sens2_v0_F5p0_f0p5_n0p62_k4_a1p0_2_4000.\suffix}
\includegraphics[width=0.40\textwidth]{images/PICO_sens2_v0_F15p0_f0p5_n0p62_k4_a1p0_2_4000.\suffix}
\caption{\label{fig:rot_sens_0} Upper panels: signal to noise ratio of the polarization angle {\prang} measurement
by $EB$ (blue lines), $TB$ (green lines) and $BB$ (red lines), assuming either no delensing (solid lines) 
or perfect delensing (dashes); the shaded area is $|\prang|/\sigma_\prang < 3$.
Lower panels: degradation on measurement of $r$, for $r=10^{-2},\ 10^{-3},\ 10^{-4}$ (magenta, orange and cyan lines, respectively),
either with no delensing (solid lines) or perfect delensing (dashes).
The underlying cosmology is Planck 2018 $\Lambda$-CDM model (with $\tau = 0.054$), and assuming a polarized noise of rms = $0.62 \mu K.\arcmin$ and power spectrum $(1 + (\ell_{\rm knee}/\ell)^n)$ with $\ell_{\rm knee}=4$ and $n=1$, with the analysis done on the multipole range $[2,4000]$ over a sky fraction $\fsky=0.5$. The beam FWHM$=5\arcmin$ on the \emph{lhs} and $15\arcmin$
on the \emph{rhs} panels.}
\end{figure}
%
% \begin{figure*}[htb]
% \includegraphics[width=0.5\textwidth]{fig_efh/PICO_sens2_v0_F15.0_f0.5_n0.62_k4_a1.0_2_4000.\suffix}
% \caption{\label{fig:rot_sens_1} Same as Fig.~\ref{fig:rot_sens_0}, with a FWHM=$15\arcmin$.}
% \end{figure*}
%
\begin{figure}[htb]
\includegraphics[width=0.40\textwidth]{images/PICO_sens2_v0_F5p0_f0p5_n0p62_k4_a1p0_20_4000.\suffix}
\includegraphics[width=0.40\textwidth]{images/PICO_sens2_v0_F5p0_f0p5_n1p86_k4_a1p0_2_4000.\suffix}
\caption{\label{fig:rot_sens_2} Same as Fig.~\ref{fig:rot_sens_0}, left panels, reducing the multipole range $[20,4000]$ (\emph{lhs}) or with a noise rms multiplied by 3 (\emph{rhs}).}
\end{figure}
% %
% \begin{figure*}[htb]
% \caption{\label{fig:rot_sens_3} Same as Fig.~\ref{fig:rot_sens_0}, with a noise rms multiplied by 3.}
% \end{figure*}
% %------------------------------------------------------------------------------------------

In Planck, the ground measurements of the detectors orientation had an error of $\pm 0.9\degree$ (rel.) $\pm 0.3\degree$ (abs.) \citep{Rosset+2010}.

The most recent constraints on cosmological birefringence (or systematic rotation) was set in \citet{Planck2016_XLIX}, looking for residual signal in $TB$ and $EB$ spectra, but are dominated by the uncertainties on the detector orientations.

In \pico, the relative rotation of the detectors, could be measured with a good accuracy (a few $0.1\arcmin$ ?, \EFH{refs?}) on the CMB, but the overall rotation is difficult to determine.
Known polarized sources, such as the Crab Nebula, could be used to do that but \citet{Aumont+2018} show that the current uncertainty of $0.33\degree = 20\arcmin$ on the Crab polarization orientation, obtained when combining all the available measurements, 
would not the measurement of tensorial $B$ modes below $r \sim 0.01$ (assuming everything else to be nominal), far from \pico's target.

Figures \ref{fig:rot_sens_0} and \ref{fig:rot_sens_2} show how the measurement of $r$ by PICO is degraded because of an overall rotation of polarization, and how $TB$ and $EB$ can be used to monitor precisely this rotation, assuming that the only source of polarization rotation is instrumental.
These results are obtained assuming the spectra to have a Gaussian likelihood, with a variance $\propto 1/\fsky$, and ignoring the foreground contributions.



$TB$ and $EB$ spectra can detect and measure a global polarisation rotation at levels ($~0.1 \arcmin$) well below those affecting $r$ measurements in $BB$ ($> 1 \arcmin$).
However, in doing such studies, one can not ignore other important aspects of the measurements of CMB polarization:
For instance, one has to consider how the delensing will affect the $B$ maps and their final noise levels.
At large scales, the interaction with foregrounds makes masking necessary, which creates some extra $E$ to $B$ leakage, and complicates the estimation of $B$ modes and of their likelihood. It also necessitates a proper component separation technique, especially in presence of finite and probably mismatched spectral bandwith of the detectors.
The time correlation of the noise (with a $1/f$ spectral shape), will also complicate the analysis.
At intermediate and small scales, many other systematics, like those related to beams, will also have to be modeled.



\subsubsection{Gain Stability}
\label{sec:gain}
\textbf{Maurizio to write up}

\subsubsection{Far Sidelobe Pickup}
\label{sec:fsl}
...

\subsubsection{Key Findings}
Understanding and controlling the effects of systematic errors in a
next-generation CMB probe is critical.

The raw sensitivity of the instrument should include enough margin
that data subsets can independently archieve the science goals.
This allows testing of the results in the data analysis and additional
data cuts, if needed.

In a PICO mission's phase A, a complete end-to-end system-level
simulation software facility would be developed to assist the team in setting 
requirements and conducting trades between subsystem requirements while
realistically accounting for post-processing mitigation.  Any future
CMB mission is likely to have similar orbit  
and scan characteristics to those of PICO, thus there is an opportunity for NASA and
the CMB community to invest in further development of this capability now.


\end{document}

%\begin{figure}[!htb]
%\centering
%\includegraphics[width=4cm]{images/example}
%\caption{example}
%\label{fig:im_3}
%\end{figure}
