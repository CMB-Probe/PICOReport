\documentclass[PICOReport.tex]{subfiles}

%\newcolumntype{L}[1]{>{\raggedright\let\newline\\\arraybackslash\hspace{0pt}}m{#1}}
%\newcolumntype{K}[1]{>{\raggedright\centering\arraybackslash}m{#1}}

\begin{document}

A CMB mission aiming for the unprescedented sensitivity of PICO must control
systematic errors to avoid bias or an increased variance of the
science measurement.   
Systematics must be controlled or corrected to a level that enables the PICO
science goals.
Mitigation of systematic errors is the most important reason (along with the broad
wavelength band) to perform a measurement of the CMB polarization from a space telescope,.
Compared with a ground-based, sub-orbital, or even a space mission in
low-Earth orbit, the L2 environment offers excellent stability as well
as the ability to observe large fractions of the sky on many time
scales without interference from the Sun, Earth, or Moon.
This redundancy of observations allows the checking of consisitency of
results and an improved ability to correct systematic errors in
post-processing analysis.

During the course of the PICO Study, a systematics working group
examined systematic errors affecting PICO, based on the experience of 
past missions, in particular Planck.  
Most  systematic errors can be mitigated by careful design and engineering of 
the spacecraft and instrument, and the use of present-day state-of-the-art
technology and data analysis tools.   
However, some systematic errors do still present a risk that they may
limit the measurement and the group studied these in further detail.

End-to-end simulation of the experiment is an essential tool,
including realistic instabilities and non-idealities of the spacecraft,
telescope, instrument and folding in data post-processing techniques
used to mitigate the effects.  During the study, the PICO team used 
simulation and analysis tools developed for the Planck mission\cite{plank2015_xii_focalplane} and for
the CORE mission concept, adapting them for PICO.  While not fully
developed, these tools allowed a deeper examination of the most
worrisome systematic errors.

\subsubsection{List of Systematics}
The systematic errors face by PICO can be categorized into three broad categories 
1) Intensity-to-polarization leakage, 2) stability, and 3)
straylight.  

\begin{table}[h!]
\centering
\scriptsize
 \begin{tabular}{p{4.1cm} p{4.1cm} p{4.1cm} p{4.1cm}}
 \hline
\textbf{Name} & \textbf{Description} & \textbf{State-of-the-art} & \textbf{Additional Possible Mitigation} \\
 \hline
\textbf{Leakage} & & \\
 Bandpass Mismatch & 
 Edges and shapes of the the spectral filters vary from detector to detector. leaks T $\to$ P, P $\to$ P leakage if the source's bandpass differs from calibrator's bandpass\cite{Hoang_2017} & Precise bandpass measurement\cite{Pajot_2010};
SRoll algorithm\cite{Planck_Lowell}; filtering technique\cite{CORE_systematics};   &
polarization modulation; full I/Q/U maps for individual detectors mitigates; \textbf{Current techniques may be adequate}  \\
Beam mismatch & & See Sect.~\ref{sec:angle}
\\
Gain mismatch &
\\
Time Response Accuracy and Stability  &
\\
Readout Cross-talk & &
\\
Polarization Angle & & &
See Sect.~\ref{sec:angle}
\\

Cross-polarization &
\\
Chromatic beam shape
\\
\hline 
\textbf{Stability} & & \\

Pointing jitter
\\

Gain Stability & & &
See Sect.~\ref{sec:gain}
\\
\hline
\textbf{Straylight} & & \\
Far Sidelobes& & &
See Sect.~\ref{sec:fsl}\\
 \hline
\textbf{Other} \\
Residual correlated cosmic ray hits &
\\
\hline
 \end{tabular}
\caption{\label{tbl:SystematicsList} Systematic errors expected to affect PICO.}
 \end{table}

\subsubsection{Absolute polarization angle calibration}
\label{sec:angle}
...

\subsubsection{Gain Stability}
\label{sec:gain}
...

\subsubsection{Far Sidelobe Pickup}
\label{sec:fsl}
...

\subsubsection{Key Findings}
Understanding and controlling the effects of systematic errors in a
next-generation CMB probe is critical.

The raw sensitivity of the instrument should include enough margin
that data subsets can independently archieve the science goals.
This allows testing of the results in the data analysis, some margin
for data cuts as needed, and  

In a PICO mission's phase A, a complete end-to-end system-level
simulation facility would be developed to assist the team to set
requirements and conduct trades between subsystem requirements while
realistically accounting for post-processing mitigation.  Any future
CMB mission is likely to have similar orbit  
and scan characteristics to those of PICO, and there is an opportunity for NASA and
the CMB community to begin to develop this capability now.


\end{document}

%\begin{figure}[!htb]
%\centering
%\includegraphics[width=4cm]{images/example}
%\caption{example}
%\label{fig:im_3}
%\end{figure}
