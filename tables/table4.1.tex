%% Table 4.1

\begin{table}
\caption{\captiontext PICO carries margin on key mission parameters. Maximum Expected Value (MEV) includes contingency.}\label{tab:mission_parameters}
\begingroup
%\openup 5pt
\newdimen\tblskip \tblskip=5pt
\nointerlineskip
\vskip -3mm
\footnotesize %\footnotesize
\setbox\tablebox=\vbox{
    \newdimen\digitwidth
    \setbox0=\hbox{\rm 0}
    \digitwidth=\wd0
    \catcode`*=\active
    \def*{\kern\digitwidth}
%
    \newdimen\signwidth
    \setbox0=\hbox{+}
    \signwidth=\wd0
    \catcode`!=\active
    \def!{\kern\signwidth}
%
\halign{
\hbox to 1.5in{#\leaderfil}\tabskip=0.6em&
\vtop{\hsize 1.5in\raggedright\hangafter=1\hangindent=1em\noindent\strut#\strut\par}\tabskip=0pt\cr
\noalign{\doubleline}
Orbit type&Sun-Earth L2 Quasi-Halo\cr
Mission class&Class B\cr
Mission duration&5 years\cr
Propellant (hydrazine)&213\,kg (77\,\% tank fill)\cr
Launch mass (MEV)&2147\,kg (3195\,kg capability)\cr
Max power (MEV)&1320\,W (with 125\,\% margin on available solar array area)\cr
Onboard data storage&4.6\,Tb (3 days of compressed data, enabling retransmission)\cr
Survey implementation&Instrument on spin table\cr
Attitude control&Zero-momentum 3-axis stabilized\cr
\noalign{\vskip 5pt\hrule\vskip 3pt}
} % close halign
} % close vbox
\endPlancktable
\endgroup
\end{table}

