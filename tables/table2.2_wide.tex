\begin{table}[h!]
\scriptsize
%\footnotesize
{\centering
\caption{\captiontext
Enumeration of potential systematic errors anticipated in \pico's measurements, their assessed priority level,
%for affecting measurement of the inflationary signal,
their effects on the measurements, and subsections with further discussion for effects with priority level~5.
%Risk level rises with value. Effects with risk level 5 have been analyzed in more detail.
\label{tbl:SystematicsList2col}
}
\vspace{-2mm}
 \begin{tabular}{@{}p{4.5cm} @{}c p{1.5cm} @{~}c@{} p{0.5cm} @{}p{3.0cm} @{}c p{1.5cm} @{~}c@{}}
 %\hline
 \cline{1-4} \cline{6-9}
 \noalign{ \vskip 1pt}
\multicolumn{1}{c}{\textbf{Name}} & \textbf{Priority$^a$}&\multicolumn{1}{c}{\textbf{Effect$^b$}}& &  &\multicolumn{1}{c}{\textbf{Name}}&\textbf{Priority$^a$}&\multicolumn{1}{c}{\textbf{Effect$^b$}}\\
 %\hline
 \cline{1-4} \cline{6-9}
  \noalign{ \vskip 1pt}
\textbf{Coupling of Signals}& & & & &   \textbf{Stability} & & \\
\quad Polarization angle calibration\dotfill&
5&
$E{\to}B$ &
\S~\ref{sec:angle} & &
% 2nd half table
\quad Gain stability\dotfill&
5&
$T{\to}P$, $E{\to} B$
&
\S~\ref{sec:gain_stability}
\\
\quad Bandpass mismatch\dotfill&
 4&
$T{\to}P$, $E{\to}B$ & & &
% 2nd half table
\quad Pointing jitter\dotfill&
3&
$T{\to}P$, $E{\to}B$
   \\
%\cline{6-9}
 \noalign{ \vskip 1pt}
\quad Beam mismatch\dotfill&
4&
$T{\to}P$, $E{\to}B$ & &
& %Sect.~\ref{sec:angle}
% 2nd half table
\textbf{Straylight}& &
\\
\quad Time response accuracy, stability\dotfill&
4&
$T{\to}P$, $E{\to}B$ & & &
% 2nd half table
\quad Far sidelobes\dotfill&
5&
spurious $P$
&
\S~\ref{sec:fsl}
\\
%\cline{6-9}
\noalign{ \vskip 1pt}
\quad Readout cross-talk\dotfill&
4&
spurious $P$ & & &
% 2nd half table
\textbf{Other}
\\
\quad Chromatic beam shape\dotfill&
4&
spurious $P$ & & &
% 2nd half table
\multirow{2}{3.3cm}{\quad Residual correlated noise \\\quad($1/f$, cosmic ray hits)\dotfill}&
3 &
\multirow{2}{1.4cm}{increased variance}
\\
\quad Gain mismatch\dotfill&
3&
$T{\to}P$ & & &
\\
\quad Cross-polarization\dotfill&
3&
$E{\to}B$
\\
\cline{1-4}
\cline{6-9}
\end{tabular}
\vskip 3pt
} % end centering
 \noindent
 \footnotesize
 {$^{a}$}~Level 5 indicates a highly significant, design-driving effect; it may have limited past measurements, or is not well understood.  Level 4 is an effect that is either known to be large but is understood reasonably well, or is a smaller effect that requires precise modeling.  In Level 3 we expect the effect to be small, but it is not sufficiently well understood and detailed modeling will be done during a Phase A study. Level 2 indicates a well-understood or minimal effect that may not need modeling, and Level 1 is for an effect that is not significant and  does not need modeling.\qquad
 {$^{b}$}~$T \rightarrow P $ denotes coupling of the intensity signal (labeled as $T$ to denote temperature) into polarization, which would generally be both $E$ and $B$. Similar meaning holds for $E \rightarrow B$.\par
%\vspace{-0.1in}
\end{table}


%% Caption text before Friday Jan 11, 5 pm.
% \caption{\captiontext
% Enumeration of potential systematic errors anticipated in \pico's measurements together with their assessed risk level,
% %for affecting measurement of the inflationary signal,
% their effects on the measurements, and in few cases subsections with further discussion. The symbol $T \rightarrow P $ denotes coupling of the intensity signal (labeled as $T$ to denote temperature) into polarization, which would generally be both $E$ and $B$. Similar meaning holds for $E \rightarrow B$. Risk level rises with value. Effects with risk level 5 have been analyzed in more detail.
% Level 5 indicates a highly significant, design-driving effect; it may have limited past measurements, or isn't well understood.  Level 4 is an effect that is either known to be large but is understood reasonably well, or is a smaller effect that requires precise modeling.  In Level 3 we expect the effect to be small, but it is not sufficiently well understood and detailed modeling will be done during Phase A study. Level 2 indicates a well-understood or minimal effect that may not need modeling, and Level 1 is for an effect that is not significant and  doesn't need modeling.
% %Level 5 indicates a highly significant, design-driving effect; it may have limited past measurements, and/or isn't well understood.  Level 4 indicates an effect that is either known to be large but is understood reasonably well, or a smaller effect that requires precise modeling.  Level 3 indicates that we expect the effect to be small, but it is not sufficiently well understood and detailed modeling will be done during Phase~A study. We used simulations to investigate effects with risk level 5.
% \label{tbl:SystematicsList2col}
% }

%Level 5 indicates a highly significant, design-driving effect; it may have limited past measurements, and/or isn't well understood.  Level 4 indicates an effect that is either known to be large but is understood reasonably well, or a smaller effect that requires precise modeling.  Level 3 indicates that we expect the effect to be small, but it is not sufficiently well understood and detailed modeling will be done during Phase~A study. We used simulations to investigate effects with risk level 5.
