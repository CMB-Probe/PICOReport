\begin{table}[h!]
\hspace{-0.1in}
\centering
\scriptsize
 \begin{tabular}{p{4.2cm} p{0.5cm} p{1.4cm} p{1.0cm} p{0.04cm} p{2.8cm} p{0.5cm} p{1.4cm} p{1.0cm}}
 %\hline
 \cline{1-4} \cline{6-9}
\textbf{Name} & \textbf{Risk}&\textbf{Effect} & &  & \textbf{Name} & \textbf{Risk}&\textbf{Effect} \\
 %\hline
 \cline{1-4} \cline{6-9}
\textbf{Coupling of Signals}& & & & &   \textbf{Stability} & & \\    
Polarization angle calibration\dotfill&
5&
$E{\to}B$ &
\S~\ref{sec:angle}. & &
% 2nd half table
Gain stability\dotfill&
5&
$T{\to}$P, $E{\to} B$
&
\S~\ref{sec:gain_stability}.
\\
 Bandpass mismatch\dotfill&
 4&
$T{\to}$P, $E{\to}B$ & & &
% 2nd half table
Pointing jitter\dotfill&
3&
$T{\to}$P, $E{\to}B$
   \\
\cline{6-9}
Beam mismatch\dotfill&
4&
$T{\to}$P, $E{\to}B$ & &
& %Sect.~\ref{sec:angle}
% 2nd half table
\textbf{Straylight}& & 
\\
Time response accuracy and stability\dotfill&
4&
$T{\to}$P, $E{\to}B$ & & & 
% 2nd half table
Far sidelobes\dotfill&
5&
spurious P
&
\S~\ref{sec:fsl}.
\\
\cline{6-9}
Readout cross-talk\dotfill&
4&
spurious P & & &
% 2nd half table
\textbf{Other} 
\\
Chromatic beam shape\dotfill&
4&
spurious P & & &
% 2nd half table
\multirow{2}{3.3cm}{Residual correlated noise (1/f, cosmic ray hits)\dotfill}&
3 &
\multirow{2}{1.4cm}{increased variance}
\\
Gain mismatch\dotfill&
3&
$T{\to}$P
\\
Cross-polarization\dotfill&
3&
$E{\to}B$
\\
\cline{1-4}
\cline{6-9}
 \end{tabular}

\hspace{-0.0in}

\caption{\captiontext
Enumeration of potential systematic errors anticipated in \pico's measurements together with their assessed risk level,
%for affecting measurement of the inflationary signal, 
their effect on the measurements, and in few cases subsections with further discussion. The symbol $T \rightarrow P $ denotes coupling of the intensity (labeled as $T$ to denote temperature) signal into polarization, which would generally be both $E$ and $B$. Similar meaning holds for $E \rightarrow B$. Risk level rises with value. Effects with risk level 5 have been analyzed in more detail.
%Level 5 indicates a highly significant, design-driving effect; it may have limited past measurements, and/or isn't well understood.  Level 4 indicates an effect that is either known to be large but is understood reasonably well, or a smaller effect that requires precise modeling.  Level 3 indicates that we expect the effect to be small, but it is not sufficiently well understood and detailed modeling will be done during Phase~A study. We used simulations to investigate effects with risk level 5.
\label{tbl:SystematicsList2col} }
\hspace{-0.0in}
\end{table}
