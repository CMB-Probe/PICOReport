\documentclass[PICOReport.tex]{subfiles}

\begin{document}

%With a cosmic-variance limited full-sky survey of all of T,E, and B, PICO will harvest essentially all of what CMB temperature and polarization anisotropies can tell us. Although PICO has many other science objectives, this alone is already worth a probe-class space mission.

The current cosmological model, as encoded by $\Lambda$CDM, provides a good fit to most current data. A host of cosmological observations including the CMB fit within the model that consists of only six parameters~\citep{Planck2018_I}. But the model is phenomenological and it leaves fundamental questions open. Premier among them is the unknown content of the majority of the Universe. Approximately 95\% of the Universe appears to be composed of dark matter and dark energy of unknown nature, both of which are necessary to explain observations at scales ranging from that of a galaxy to that of the Hubble volume. Yet, there are no detection of dark matter particles, and as for dark energy, it even lacks a compelling theoretical motivation.

In this context, tension between measurements of any $\Lambda$CDM parameter obtained by different probes compel additional stringent tests and investigation of alternatives to the prevailing paradigm. Examples of emerging tensions are: the $3.6\sigma$ discrepancy between the CMB- and local-Universe-anchored supernovae-based measurements of the Hubble constant~\citep{Aghanim:2018eyx,Riess2018}; the identification of lack of power in the low $\ell$ modes of the $TT$ power spectrum that has a probability of less than $10^{-3}$ of occurrence in standard $\Lambda$CDM~\citep{TT_correlations}; and the $\sim 2\sigma$ tension in measurements of the amplitude of late time perturbations $\sigma_{8}$ between 
%$\sigma_8$-$\Omega_{\rm m}$ plane from 
the \planck\ CMB $TT$, $TE$, and $EE$ power spectra and those from cosmic shear surveys~\citep{Joudaki+2017, Abbott+2018, Hikage+2018, vanUitert+2018}.  A similar level  of tension for $\sigma_{8}$ ($\sim2\sigma$) arises when comparing \planck\ CMB spectra and cluster counts from \planck\ and other surveys~\citep{Planck, Bocquet+2018}. Such tensions, while perhaps only indicating the presence of systematic effects in the measurements, may in fact point toward new physics. One way to search for new physics is to better constrain the current measurements and to test for extensions beyond the base six-parameter set. 

Given PICO's baseline noise and angular resolution, and an input set of $N$ fiducial $\Lambda$CDM parameters, it is straightforward to calculate the uncertainty with which PICO will constrain this set~\citep{core_parameter}. A figure of merit (FOM) that quantifies the strength of the constraint is the volume of the uncertainty region in the $N$-dimensional space. We use the same analytical approach and FOM that have also been used in other studies~\citep{core_parameter,Wang2008,pdg2018,Namikawa2010}.\footnote{The FOM is determined by the covariance of the Fisher information matrix, ${\rm FoM} = \left( \det \left[ \mbox{cov} ( p_i)  \right] \right)^{-1/2},\,\, i=1, ..., N$, where $p$ is the parameter set.} This FOM is defined such that a larger value linearly corresponds to {\it smaller} volume and thus to smaller parameter errors. 
%\comblue{what else are we including? delensing? foreground separation? Can we point to other papers in which this specific code and FOM have been used? what 'cmb' information is used? only spectra?} \comred{search for FOM for dark energy, neutrino mass}

The six-parameter $\Lambda$CDM model that is constrained by current measurements includes the baryon density, the dark matter density, the amplitude and spectral index of a power-law spectrum of initial perturbations, the angular scale of acoustic oscillations, and the optical depth to reionization. To this set we added: $N_{\rm eff}$ the effective number of light relics~(\S~\ref{sec:relics_neutrinos}); two dark energy parameters $w_{0}$ and $w_{a}$\footnote{$w(a) = w_{0} + (1-a)w_{a}$.}; the tensor to scalar ratio $r$~(\S~\ref{sec:inflation}); the sum of neutrino masses $\sum m_{\nu}$~(\S~\ref{sec:relics_neutrinos}); and the amplitude of correlated CDM isocurvature perturbations. For this 12-parameter set and for the PICO baseline configuration we find that the FOM is $9.5\times10^{9}$ larger than that calculated for \planck . For the CBE the value increases to $30\times10^{9}$. Excluding an inflationary signal by fixing $r=0$, the values are $5\times10^{6}$ and $7\times10^{6}$, respectively~\citep{picoweb_lcdm}.  Even stronger improvements will be obtained when the PICO CMB data will be combined with data sets available in the next decade, including weak lensing, BAO, and cluster of galaxies. 
 
These improvements will test $\Lambda$CDM so stringently that it is hard to imagine it surviving such a scrutiny if it is not fundamentally correct. If tensions deepen to become discrepancies, it would be equally exciting if a new cosmological model emerged. 

%\comor{now here}
% Table /textcolor{red}{XXX} lists the improvement obtained by PICO alone over Planck 2018 results when such extensions are considered. The improvement is quantified with a figure of merit (FoM) that is proportional to the inverse of the volume of the error box
%\begin{equation}
%{\rm FoM} = (\det \left[ \mbox{cov} \left\{ \omega_b, \omega_c, \theta, \tau, A_s, n_s, p_i \right\} \right])^{-1/2},
%\end{equation}
%where $p_i$ are the extra parameters being considered.


\end{document}

%The improvement of the FoM \emph{with CMB alone}, by a factor between $3$ and $10$ billions (when $r$ is included in the extra parameters) or close to five million (excluding $r$) provides a test of $\Lambda$CDM that, is so stringent that it is hard to imagine that $\Lambda$CDM can survive such a scrutiny if it is not fundamentally correct. This is true in particular considering that in parallel to the improvement of CMB constraints, other probes such as weak lensing, BAO, clusters of galaxies (the later with PICO itself), will also experiment similar tightening of their own error boxes based on largely independent observables.

%PICO, in addition, will also improve constraints on other extensions of $\Lambda$CDM, such as interacting dark matter (with constraints on scattering cross sections), modifications of the laws of gravitation, primordial non-Gaussianity, etc. The history of the development of physical cosmology has taught us that discoveries often come from unexpected directions. PICO is ideally suited to look at the surprises that the Universe still has to reveal.

%\begin{table*}
%\begin{center}\footnotesize
%\begin{tabular}{cccc}
%\hline \hline
%    Model     & PICO v4.0 & PICO v4.1 &  Planck18   \\                   
%\hline
%$\Lambda$CDM + $\neff$ +$\alpha_1$+$w_0$+$w_a$+$\mnu$ & $7.4\times10^{6}$ & $4.8\times10^{6}$& $1$  \\
%$\Lambda$CDM + $\neff$ +$\alpha_1$+$w_0$+$w_a$+$r$+$\mnu$ & $2.7\times10^{10}$ & $9.5\times10^{9}$& $1$  \\
%\hline
%\hline
%\end{tabular}
%%\caption{}
%\label{tabFoM}
%\end{center}
%\end{table*}



%%%%%%%%%%%%%%%%%%%%%%%%%%%%%%%
