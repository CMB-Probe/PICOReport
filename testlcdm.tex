\documentclass[PICOReport.tex]{subfiles}

\begin{document}

%With a cosmic-variance limited full-sky survey of all of T,E, and B, PICO will harvest essentially all of what CMB temperature and polarization anisotropies can tell us. Although PICO has many other science objectives, this alone is already worth a probe-class space mission.

The current paradigm for the structure of the Universe, as encoded by \ac{LCDM}, provides a reasonably good phenomenological fit to current CMB and other cosmological observations. The combination of most data in hand can be fit with six parameters \comor{that include the baryon density, the dark matter density, the ..., }~\citep{??}. However, fundamental open questions remain, and premier among them is the unknown content of the Universe. Approximately 95\% of the Universe appears to be composed of dark matter and energy of unknown nature, both of which are necessary to explain cosmological observations at scales ranging from that of a galaxy to that of our present Hubble volume. However, no direct detection of dark matter particles has been made, and as for dark energy, it even lacks a compelling theoretical model.

In this context, tensions between $\Lambda$CDM parameters obtained by various probes compel additional stringent tests and investigation of possible alternatives. Examples of emerging tensions are the $3.5\sigma$ \comred{check} discrepancy between the CMB- and supernovae-based measurements of the Hubble constant~\citep{??}; and the $2.5\sigma??$ \comred{check} discrepancy between \planck\ and weak lensing surveys in the  measurement of the amplitude of late time perturbations $\sigma_8$. Such tensions, while perhaps only indicating systematic effects in measurements, may in fact be pointing toward new physics and new insights. 

One way to search for new physics is to test for natural extensions of the six-parameter \ac{LCDM} that include parameters for e.g. dark energy, neutrino masses, the number of relativistic species, and inflation.

\comor{now here}
 Table /textcolor{red}{XXX} lists the improvement obtained by PICO alone over Planck 2018 results when such extensions are considered. The improvement is quantified with a figure of merit (FoM) that is proportional to the inverse of the volume of the error box
\begin{equation}
{\rm FoM} = (\det \left[ \mbox{cov} \left\{ \omega_b, \omega_c, \theta, \tau, A_s, n_s, p_i \right\} \right])^{-1/2},
\end{equation}
where $p_i$ are the extra parameters being considered.

The improvement of the FoM \emph{with CMB alone}, by a factor of more than ten billion (when $r$ is included in the extra parameters) or close to ten million (excluring $r$) provides a test of $\Lambda$CDM that, is so stringent that it is hard to imagine that $\Lambda$CDM can survive such a scrutiny if it is not fundamentally correct. This is true in particular considering that in parallel to the improvement of CMB constraints, other probes such as weak lensing, BAO, clusters of galaxies (the later with PICO itself), will also experiment similar tightening of their own error boxes based on largely independent obsevrables.

PICO, in addition, will also improve constraints on other extensions of $\Lambda$CDM, such as interacting dark matter (with constraints on scattering cross sections), modifications of the laws of gravitation, primordial non-Gaussianity, etc. The history of the development of physical cosmology has taought us that discoveries often come from unexpected directions. PICO is ideally suited to look at the surprises that the Universe still has to reveal.



\end{document}

%%%%%%%%%%%%%%%%%%%%%%%%%%%%%%%
