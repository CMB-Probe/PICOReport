
\documentclass[PICOReport.tex]{subfiles}
 
\begin{document}

The combination of PICO's diverse science goals are achievable using a single instrument that executes 
a single, continuous observing pattern of the entire sky, and that integrates noise down to unprecedented levels in 
multiple frequency bands between 20 and 800 GHz. 

$\bullet$ {\bf Depth} \hspace{0.1in} We quantify survey depth in terms of the RMS fluctuations that would give
a signal-to-noise ratio of 1 on a sky pixel that is 1~arcminute on a side. To reach the limits on the IGW signal 
requires a depth of 0.8~\microkamin.  
 (r, foregrounds, systematics)

$\bullet$ {\bf Sky Coverage} \hspace{0.1in} There are several science goals driving a full sky survey for PICO. The 
term 'full sky' refers to the entire area of sky available after separating other astrophysical sources of confusion. In 
practice this implies an area of 50-70\% of the full sky for probing non-Galactic signals, and the rest of the sky
for achieving the Galactic science goals. \\
(1) Probing the optical depth to the epoch of reionization (STM SO5) requires full sky 
coverage as the signal peaks in the EE power spectrum on angular scales of 20 to 90 degrees. Measuring 
this optical depth to limits imposed by the statics of the small number of available $\ell$ modes is crucial 
for minimizing the error on the neutrino mass measurement. \\   
(2) If $r \ne 0 $, the BB power spectrum due to IGW (STM SO1) also has a local maximum in the same
range of angular scales (20 to 90 degrees). For $r \simgt 0.001$ (CHECK)
this local maximum is at a higher level than the BB lensing spectrum, making this range of scales appealing 
to survey, as there is no need to separate the signatures of two cosmological signals. \\
(3) The PICO constraint on $N_{eff}$ (STM SO4) requires a determination of the EE power spectrum to limits
imposed by the statics of available $\ell$ modes. Full sky coverage is required to achieve this limit.  
(4) PICO's survey of the Galactic plane and regions outside of it is essential to achieving its Galactic structure 
and star formation science goals (SO6, 7, 8). 

$\bullet$ {\bf Frequency Bands} \hspace{0.1in} The multitude of astrophysical signals that PICO will characterize determine
the frequency range and number of sub-bands that PICO uses. The IGW signal peaks 
in the frequency range between 30 and 300 GHz. However, Galactic signals, which are themselves signals PICO strives to 
characterize, are a source of confusion for the IGW. The Galactic signals and the IGW are separable using their 
spectral signature, and simulations indicate that 21 bands, each with $\sim$25\% bandwidth, and that are spread across 
the range of 20 - 800~GHz can achieve the separation at a level of fidelity required by PICO. 

Characterizing the Galactic signals, specifically the make up of Galactic dust (SO7), requires spectral characterization 
of galactic dust in frequencies between 100 and 800~GHz. \comblue{Aren't there synchrotron questions that 
are answerable with spectral information?} 

$\bullet$ {\bf Resolution} \hspace{0.1in} (foregrounds, lensing, $N_eff$, )

$\bullet$ {\bf Sky Scan Pattern} \hspace{0.1in} (uniform coverage, and systematic mitigation)


 
 
\end{document}

%\begin{figure}[!htb]
%\centering
%\includegraphics[width=4cm]{images/example}
%\caption{example}
%\label{fig:im_3}
%\end{figure} 

