
\documentclass[PICOReport.tex]{subfiles}
 
\begin{document}

The set of physical parameters and observables that derive from the PICO science objectives place 
requirements on the depth of the mission, the fraction of sky the instrument scans, the frequency range the 
instrument probes and the number of frequency bands, the angular resolution provided by the reflectors, and
the specific pattern with which PICO will observe the sky. We discuss each of these aspects. \\
%combination of PICO's diverse science goals are achievable using a single instrument that executes one, continuous, simple observing pattern of the entire sky. This pattern integrates noise down to unprecedented levels, and provides for multiple checks of possible systematic errors in the data analysis. \\
%
$\bullet$ {\bf Depth} \hspace{0.1in} We quantify survey depth in terms of the RMS fluctuations that would give
a signal-to-noise ratio of 1 on a sky pixel that is 1~arcminute on a side. Depth in any frequency band 
is determined by detector sensitivity, the number of detectors in the focal plane, the sky area covered, and the 
duration of the mission.  The science objective driving 
the depth requirement is SO1, the search for the IGW signal which 
requires a combined depth of 0.87~\microkamin. This requirement is a combination of the low-level of the signal, the need
to separate the various signals detected in each band, and the need to detect and subtract systematic effects 
to anticipated levels.  The \ac{CBE} value is 0.61~\microkamin\ coming from a realistic estimate of detector noise, and 
giving 40\% margin on mission performance. \\
%The required depth assumes full sky coverage, which is d. We expect to only use 50-60\% of the full sky, which do not include the brightest emission from the Milky Way, for the data analysis leading to a constraint on $r$. But as discussed below, other science objectives, require a scan of the Galaxy, leading to the requirement  
%To achieve this combined depth we implement the focal plane of detectors listed inTable~\ref{tab:reqs} The requirement on performance includes the following factors  a factor of 1.5 degradation in noise relative to our baseline expectations. 
%
$\bullet$ {\bf Sky Coverage} \hspace{0.1in} There are several science goals driving a full sky survey for PICO. The 
term `full sky' refers to the entire area of sky available after separating other astrophysical sources of confusion. In 
practice this implies an area of 50-70\% of the full sky for probing non-Galactic signals, and the rest of the sky
for achieving the Galactic science goals. \\
(1) Probing the optical depth to the epoch of reionization (STM SO5) requires full sky 
coverage as the signal peaks in the $EE$ power spectrum on angular scales of 20 to 90 degrees. Measuring 
this optical depth to limits imposed by the statistics of the small number of available $\ell$ modes is crucial 
for minimizing the error on the neutrino mass measurement. \\   
(2) If $r \ne 0 $, the $BB$ power spectrum due to IGW (STM SO1) has local maxima on large angular scales
(20 to 90 degrees, $ 2 \leq \ell \leq 10$), and around 1~degree ($ \ell \simeq 80$). 
A detection would strongly benefit from confirmation at {\it both} angular 
scales -- a goal that is beyond the capabilities of ground-based instruments -- and, for the $\ell = 80$ peak, 
{\it in several independent patches of the sky}, a goal PICO will achieve, but that is currently not planned for 
any next decade instrument.  \\
(3) The PICO constraint on $N_{eff}$ (STM SO4) requires a determination of the $EE$ power spectrum to limits
imposed by the statics of available $\ell$ modes. Full sky coverage is required to achieve this limit.  \\
(4) Achieving the neutrino mass limits (STM SO3), giving two independent $4\sigma$ constraints on the minimal sum of 58~meV, requires
a lensing map, and cluster counts from as large a sky fraction as possible. \\
(4) PICO's survey of the Galactic plane and regions outside of it is essential to achieving its Galactic structure 
and star formation science goals (SO6, 7). \\
%
$\bullet$ {\bf Frequency Bands} \hspace{0.1in} The multitude of astrophysical signals that PICO will characterize 
determine the frequency range and number of bands that the mission uses. The \ac{IGW} signal peaks 
in the frequency range between 30 and 300 GHz. However, Galactic signals, which are themselves signals PICO strives to 
characterize, are a source of confusion for the IGW. The Galactic signals and the IGW are separable using their 
spectral signature. Simulations indicate that 21 bands, each with $\sim$25\% bandwidth, that are spread across 
the range of 20 - 800~GHz can achieve the separation at the level of fidelity required by PICO. 

Characterizing the Galactic signals, specifically the make up of Galactic dust (SO6), requires spectral characterization 
of Galactic dust in frequencies between 100 and 800~GHz. \\ 
%\comblue{Aren't there synchrotron questions that are answerable with spectral information?} \\
%
$\bullet$ {\bf Resolution} \hspace{0.1in} 
Several science objectives require an aperture of 1.5~m and the resolution per frequency listed in Table~\ref{tab:STM}. To reach $\sigma(r) = 1\cdot10^{-4}$ we will need to `delens' the $E$- and $B$-mode maps, as describe in Sections~\ref{sec:fundamentalsci} and~\ref{sec:extragalacticsci}. Delensing is enabled with a map that has a native resolution of 2-3 arcminutes at frequencies between ~100 and ~300~GHz. Similar resolution is required to achieve the constraints on the number of light relics (SO4), which will be extracted from the $EE$ power spectrum at multipoles $100 \lesssim \ell \lesssim 2500$.  The process of delensing may be affected by other signals, primarily the signal due to Galactic dust. It is thus required to map Galactic dust to at least the same resolution as at ~300~GHz. Higher resolution is mandated by SO6 and 7, which require resolution of 1 arcminute at 800~GHz.  We have thus chosen to implement diffracted limited resolution between 20 and 800~GHz. \\
%
$\bullet$ {\bf Sky Scan Pattern} \hspace{0.1in} 
Control of polarization systematics uncertainties at anticipated levels is enabled by (1) making $I$, $Q$, and $U$ stokes parameters maps of the entire sky from each independent detector; and (2) by enabling sub-percent absolute gain calibration of the detectors through observations of the CMB dipole.  With these requirements we chose a sky scan pattern that enables each detector to scan a given pixel of the sky in multitude of directions, satisfying requirement (1). The scan we chose also gives strong CMB dipole signals in every rotation of the spacecraft throughout the lifetime of the mission, satisfying requirement (2). 
% need to be more quantitative. 

 
\end{document}

%\begin{figure}[!htb]
%\centering
%\includegraphics[width=4cm]{images/example}
%\caption{example}
%\label{fig:im_3}
%\end{figure} 

