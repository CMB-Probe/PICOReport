\documentclass[PICOReport.tex]{subfiles}

\begin{document}

Recent theoretical developments and measurements of the \ac{CMB} have uncovered tremendous potential for new exciting discoveries over the next 10 years. The new discoveries, to be delivered by the Probe of Inflation and Cosmic Origins (PICO), are promising to be revolutionary, affecting physics, astrophysics, and cosmology on the most fundamental levels. 

PICO is an imaging polarimeter that will scan the sky for 5 years with 21 frequency bands spread between 21 and 800~GHz. It will produce 10 independent full sky surveys of intensity and polarization with a final combined-map noise level that make it equivalent to 3250 \planck\ missions for the baseline required specifications, and estimated to actually perform as 6400 \planck\ missions. It will produce the first ever full sky polarization maps at frequencies above 350~GHz, and it will have diffraction limited resolution, giving it a resolution of 1' at 800~GHz. 

With these unprecedented capabilities, which are unmatched by any other existing or proposed platform, PICO could detect the signature of an inflationary epoch near the big bang. If the signal is not detected, it will probe physics at the quantum gravity scale, constrain broad classes of inflationary models, and exclude models with a mass scale larger than the Planck mass at more than $5\sigma$. The combination of data with LSST could rule out slow-roll single-field inflation, which will mark a landmark transition in studies of inflation. 

The mission will have a deep impact on particle physics by measuring the minimal expected sum of the neutrino masses with high confidence reaching $4\sigma$, in two independent methods. The measurements will either detect or strongly constrain deviations from the standard model of particle physics by counting the number of light particles in the early universe at an energy range that is up to 400 times higher than available today. The data will constrain dark matter candidates by pushing \planck\ constraints on the dark matter cross section by a factor of 25, and at low energy scales that are not accessible to direct detection experiments. The data will probe the existence of cosmic fields that could give rise to cosmic birefringence. 

PICO's will transform our knowledge of the structure and evolution of the universe. It will measure the redshift to the reionization epoch, strongly constraining physical models describing when and how the first luminous objects formed. It will make a map of the projected matter throughout the volume of the universe with a signal-to-noise ratio exceeding 500, more than 10 times the signal-to-noise ratio of \planck. Magnetic fields thread galaxies and affect their structure and evolution, but the origins of these magnetic fields is a hotly debated question. PICO will resolve the question of whether galactic magnetic fields have been seeded by primordial magnetic fields of cosmic origin. By discovering 50,000 proto-clusters with redshift up to 4.5, and  4500 strongly lensed galaxies with redshift up to 5, PICO will enable a unique view into early galaxy and cluster evolution. These counts are factors of 100 to 1000 larger than available with catalogs today, and the window PICO provides is entirely unique and not available to any other space mission. 


The potential new discoveries are based on deeper measurements of the spatial pattern of the CMB's polarization\comor{would like to make this broader to not shortchange T-science, but still connect to $E$, $B$ in the next paragraph}.

\comor{broad science, unique mission, nothing better in the foreseable future, complementing and enriching other science in the next decade, comparatively cheap, within cost, using existing technologies, relying on extensive community experience both on the ground and in space}

PICO's data will enrich and complement other astrophysical surveys in the next decade.

We note that if there {\it is} a detection of the \ac{IGW} signal with $r=0.001$, PICO will make it with high significance in multiple independent patches of the sky. 


A detection would strongly benefit from confirmation at {\it both} angular 
scales -- a goal that is beyond the capabilities of ground-based instruments -- {\it and}, for the $\ell = 80$ peak, 
in several independent patches of the sky -- a goal that is currently not planned for any next decade instrument. 

\end{document}

%see Fig.~\ref{fig:im_1}

%\begin{figure}[!htb]
%\centering
%\includegraphics[width=4cm]{images/example}
%\caption{example}
%\label{fig:im_1}
%\end{figure}
